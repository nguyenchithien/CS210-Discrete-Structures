\documentclass[../RachelNotes.tex]{subfiles}

\begin{document}

    \chapter{Introduction \& Logic}

    At the beginning of our course, we introduce some little word
    problems and games as a way to get you to start thinking
    methodically about the types of problems we will face.
    The important parts of Chapter 1 of our textbook is the
    propositions, predicates, and implications, so we will start
    here with our review, and come back to the sequences later on.

    \section{Propositions}

        \begin{introNOHEAD}
            \paragraph{Proposition:}
                A proposition is a type of statement that is either
                \textit{true or false}. These statements are standalone,
                and don't take any input. If you're familiar with programming,
                they are essentially like a boolean variable.

                \begin{center}
                    ``4 is greater than 5." \\
                    ``Pratik is taking Discrete Math." \\
                    ``The moon is bigger than the sun." \\
                \end{center} ~\\
                
                Note that a proposition \textit{doesn't have to be true}
                -- the moon is not bigger than the sun -- but it is still
                a proposition because we can identify it as either true or false.

            \paragraph{Propositional variables:}
                When we want to work with propositions, we will usually
                come up with a propositional variable. These are like
                variables in algebra, but instead of containing numbers,
                they contain propositional statements.
        \end{introNOHEAD}

        While working with propositions, we will write 'em both in
        mathematical terms, like:

        \begin{center} $p$ is the proposition, ``length and width are equal." \end{center}

        As well as propositions that are just English statements, like:

        \begin{center} $q$ is the proposition, ``Mariam likes cats." \end{center}

        Either way, we can build statements out of these!



        \subsection{Logic operators and compound statements}

        If you've spent some time programming, you've probably run across
        if statements that contain ``and", ``or", and ``not" in the statements.
        In C++, this would look like:

        \begin{center}
        \texttt{
        if ( a $>$ 0 \&\& a $<$ 10 )
        }
        \end{center}

        For Discrete Math, we will also have these logic operators, but
        they look different.

        \begin{center}
            \begin{tabular}{ c c c }
                \textbf{English} & \textbf{C++} & \textbf{Logic}
                \\ \hline
                AND & \texttt{\&\&} & $\land$
                \\
                OR & \texttt{||} & $\lor$
                \\
                NOT & \texttt{!} & $\neg$
            \end{tabular}
        \end{center}

        When we use these operators to combine two (or more) propositions,
        the result is a \textbf{compound statement}, like this: 

        \begin{mdframed}
            ~\\
            $p$ is the proposition, ``Brandon is allergic to cats" \\
            $q$ is the proposition, ``Brandon owns a cat" ~\\ ~\\
            $p \land q$: ``Brandon is allergic to cats, and Brandon owns a cat." \\
            $p \lor q$: ``Brandon is allergic to cats, or Brandon owns a cat." \\
            $p \land \neg q$: ``Brandon is allergic to cats, and Brandon \textit{does not} own a cat."
        \end{mdframed}



        \subsection{Negations}
        
        Using negations, we can turn a propositional statement into its
        logical opposite... This means that if a statement (or compound statement)
        is \textit{true}, then the opposite is \textit{false}, and if the
        statement is \textit{false}, then the opposite is \textit{true}. \\

        If we create the proposition
        $p$ is ``the printer is functioning."

        Then the negation, $\neg p$, is ``the printer is \textbf{not} functioning."

        We can also negate compound statements, which follow some rules...

        \begin{center}
            The negation of $p \land q$ is written as $\neg(p \land q)$, \\
                and is \textit{logically equivalent} to $\neg p \lor \neg q$. ~\\ ~\\

            The negation of $p \lor q$ is written as $\neg(p \lor q)$, \\
                and is \textit{logically equivalent} to $\neg p \land \neg q$.
        \end{center}

        When two statements are \textbf{logically equivalent}, it means that
        their outcomes will be the same based on the values of their
        propositional variables - in this case, $p$ and $q$.

        \paragraph{Negating relational operators} ~\\
        You might not have thought about it before, but the negation of
        something containing a $>$ might not be what you think:
        The logical opposite of $>$ is $\leq$, and the logical opposite of $<$ is $\geq$.

        \begin{mdframed}
            $a$ is the proposition, $x > 2$. \tab
            $\neg a \equiv x \leq 2$.
            ~\\~\\
            $b$ is the proposition, $y < 10$. \tab{}
            $\neg b \equiv y \geq 10$.
        \end{mdframed}

        And perhaps more intuitively, the negation of $=$ is going to be $\neq$.
        
        \begin{mdframed}
            $c$ is the proposition, $z = 5$. \tab
            $\neg c \equiv z \neq 5$.
        \end{mdframed}

        \newpage
        Just to drive the point home a little more, let's assume we're
        writing a program and using an \textit{if, else} statement to
        check some variable.
    

\begin{figure}[h]
\centering
\begin{subfigure}{.5\textwidth}
    \centering
\begin{mdframed}
\begin{verbatim}
if ( age < 18 )
{
    print( "Can't vote" );
}
else
{
    print( "Can vote" );
}
\end{verbatim}
\end{mdframed}
\end{subfigure}%
\begin{subfigure}{.5\textwidth}
    \centering
    
    \begin{addmargin}[1em]{0em}
        If the user enters some number less than 18, they will see
        the ``Can't vote" message pop up. However, the else statement
        will catch any other cases - in this case, if age is equal to 18,
        or if it is greater than 18.
    \end{addmargin}
    
\end{subfigure}
\end{figure}

    \subsection{Contradictions and Tautologies}

        We can also end up writing propositional statements that always
        result to being \textit{true} no matter what, or always
        result to being \textit{false} no matter what. For the first case,
        if a statement is always true, then it is a \textbf{tautology}.
        On the other hand, if the statement is always false, then
        it is a \textbf{contradiction}. ~\\
        
        For example, if we said ``if both p and q are true, then p is true",
        then that will always be true. If we say $p \land \neg p$, then
        this will always be false - $p$ cannot be both true and not-true
        at the same time.


    \subsection{Truth tables}

        
        How do we know if a compound statement is true or false?
        How can we tell if two compound statements are logically equivalent?
        We can use \textbf{truth tables} to diagram out all possible states
        in our propositional statements.

        For a truth table, the left-hand side contains the propositional
        variables being used by the statement, and the right-hand side
        contains the compound statement itself (and perhaps some ``in-between"
        statements to help us solve the final form.)
        Using the basic truth tables for $p \land q$, $p \lor q$ and $\neg p$,
        we can figure out the truth tables for more complex propositional statements.

        \newpage
        
        
        Let's say we want to know what the result of $p \land \neg q$ is,
        for all possible values of $p$ and $q$. We could diagram it like this:
        
        \begin{center}
            \begin{tabular}{c c | c | c}
                & & & Compound
                \\
                \multicolumn{2}{c|}{Variables} & & statement
                \\
                $p$ & $q$ & $\neg q$ & $p \land \neg q$
                \\ \hline
                T & T & F & F
                \\
                T & F & T & T
                \\
                F & T & F & F
                \\
                F & F & T & F
            \end{tabular}
        \end{center}

        But first, let's look at our truth tables for AND, OR, and NOT.

\begin{figure}[h]
\centering
\begin{subfigure}{.3\textwidth}
\centering

\textbf{And} ~\\ ~\\

\begin{tabular}{ c c | c }
    $p$ & $q$ & $p \land q$
    \\ \hline
    T & T & T \\
    T & F & F \\
    F & T & F \\
    F & F & F
\end{tabular}

\end{subfigure}%
\begin{subfigure}{.3\textwidth}
\centering

\textbf{Or} ~\\ ~\\

\begin{tabular}{ c c | c }
    $p$ & $q$ & $p \lor q$
    \\ \hline
    T & T & T \\
    T & F & T \\
    F & T & T \\
    F & F & F
\end{tabular}
    
\end{subfigure}%
\begin{subfigure}{.3\textwidth}
\centering

\textbf{Not} ~\\ ~\\

\begin{tabular}{ c | c }
    $p$ & $\neg p$
    \\ \hline
    T & F \\
    F & T
\end{tabular}
    
\end{subfigure}
\end{figure}


    \hrulefill
    \subsection{Practice}
        
        

    
    \newpage
    \section{Predicates}

    \section{Implications}

\end{document}

