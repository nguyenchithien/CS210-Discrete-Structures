\documentclass[a4paper,12pt]{book}
\usepackage[utf8]{inputenc}
\title{}
\author{Rachel Morris}
\date{\today}

\usepackage{rachwidgets}
\usepackage{fancyhdr}
\usepackage{lastpage}
\usepackage{dirtree}
\usepackage{boxedminipage}

\newcommand{\laChapter}{1.5\ }

\pagestyle{fancy}
\fancyhf{}
\lhead{CS 210}
\chead{Fall 2017}
\rhead{Ch \laChapter Exercise}
\rfoot{\thepage\ of \pageref{LastPage}}
\lfoot{\scriptsize By Rachel Morris, last updated \today}

\renewcommand{\headrulewidth}{2pt}
\renewcommand{\footrulewidth}{1pt}

\begin{document}

    %\toggletrue{answerkey}
    \togglefalse{answerkey}
    
    %------------------------------------------------------------------%
    %- Exercise Begin -------------------------------------------------%
    %------------------------------------------------------------------%

    %------------------------------------------------------------------%
    \section*{1. Predicates}

        \begin{intro}{Implications} ~\\
            A statement like ``if $p$ is true, then $q$ is true" is known
            as an implication. It can be written symbolically as $p \to q$,
            and read as ``p implies q" or ``if p, then q".

            For the implication $p \to q$, $p$ is the \textbf{hypothesis}
            and $q$ is the \textbf{conclusion}

            \paragraph{Example:} Write the following statement symbolically,
                defining the propositional variable:
                \begin{center} ``If it is Kate's birthday, then Kate will get a cake." \end{center}

                I will define two propositional variables: ~\\
                $b$ is ``It is Kate's birthday" \tab $c$ is ``Kate will get a cake."

                ~\\
                And then I can write it symbolically: $b \to c$.
        \end{intro}

        % - QUESTION --------------------------------------------------%
        \begin{question}{1}{5\%}
            For the following ``if, then" statements, create two
            \textbf{propositional variables}, assign them values (from
            the original statement), and then write it symbolically.

            \begin{enumerate}
                \item[a.] IF you don't play, THEN you can't win!
                \item[b.] IF your friends don't dance, THEN they aren't friends of mine!
                \item[c.] IF Timmy's age is over 8 AND Timmy's age is less than 13, THEN Timmy gets Tween-priced movie tickets.
                \item[d.] IF I get enough sleep, OR I drink coffee, THEN I can go to work.
            \end{enumerate}
        \end{question}

        \newpage
        \begin{intro}{Truth for implications}
            With an implication, ``if \textbf{hypothesis}, then \textbf{conclusion}",
            the truth table will look a bit different than what you've done so far.
            Make sure to take notice because it is easy to make an error when it
            comes to implication truth tables!

            For an implication to be logically FALSE, it must be the case that
            the \textbf{hypothesis is true, but the conclusion is false}. In
            all other cases, the implication results to \textbf{true}.

            \begin{center}
                \begin{tabular}{| c | c | c | c |}
                    \hline{}
                    $p$ & $q$ & & $p \to q$
                    \\ \hline
                    T & T & & T
                    \\ \hline

                    T & F & & F
                    \\ \hline

                    F & T & & T
                    \\ \hline

                    F & F & & F
                    \\ \hline
                \end{tabular}
            \end{center}

            Seems weird? Think of it this way: The only FALSE result is if
            the \textbf{hypothesis} is true, but the \textbf{conclusion}
            ends up being false. In science, it would mean our hypothesis
            has been disproven. If, however, the \textbf{hypothesis}
            is false, we can't really say anything about the conclusion
            (again thinking about science experiments)
        \end{intro}

        % - QUESTION --------------------------------------------------%
        \begin{question}{1}{5\%}
            Complete the truth tables (on the worksheet) for the following
            compound expressions.

            \begin{enumerate}
                \item[a.] $(p \land q) \to q$
            \end{enumerate}
        \end{question}

    %------------------------------------------------------------------%
    %- WORKSHEET ------------------------------------------------------%
    %------------------------------------------------------------------%
    \newpage
    \begin{center} \section*{Chapter \laChapter In-class Exercise Worksheet} \end{center}

    \iftoggle{answerkey}{
      \begin{answer} \begin{center} ANSWER KEY \end{center} \end{answer}
    }{}

    % \iftoggle{answerkey}{ \begin{answer} asdfasdf \end{answer} }{ { ~\\ \raisebox{0pt}[2cm][0pt]{  } } }
    % \iftoggle{answerkey}{ \begin{answer} TRUE \end{answer} }{}

\begin{answersheetquestion}{1}{Basic If, Then Statements}{5}

    \begin{enumerate}
        \item[a.] IF you don't play, THEN you can't win!
        \item[b.] IF your friends don't dance, THEN they aren't friends of mine!
        \item[c.] IF Timmy's age is over 8 AND Timmy's age is less than 13, THEN Timmy gets Tween-priced movie tickets.
        \item[d.] IF I get enough sleep, OR I drink coffee, THEN I can go to work.
    \end{enumerate}
\end{answersheetquestion}

\hrulefill{}







    %- Team Info ------------------------------------------------------%

    \newpage
    
    \paragraph{Team:}
    Please write down all people in your team. ~\\

    % table %
    \begin{tabular}{ p{6cm} p{6cm} }
        1. & 2. \\
        3. & 4.
    \end{tabular}
    % table %
    ~\\

    \hrulefill
    \subsection*{Grading}
            
    \begin{center}
        
        \begin{tabular}{ | l | l | l | l | }
            \hline
            \textbf{ Question } & \textbf{ Weight } & \textbf{ 0-4 } & \textbf{ Adjusted score }
            \\ \hline
            
            1 & 5\% & &    \\ \hline
            
            2 & 6\% & &    \\ \hline
            
            3 & 12\% & &    \\ \hline
            
            4 & 15\% & &    \\ \hline
            
            5 & 25\% & &    \\ \hline
            
            6 & 25\% & &    \\ \hline
            
            7 & 12\% & &    \\ \hline
            
            
            
        \end{tabular}
    \end{center}

\end{document}
