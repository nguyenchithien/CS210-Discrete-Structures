\documentclass[a4paper,12pt]{book}
\usepackage[utf8]{inputenc}

\usepackage{rachwidgets}


\newcommand{\laClass}       {CS 210}
\newcommand{\laSemester}    {Spring 2018}
\newcommand{\laChapter}     {1.3}
\newcommand{\laType}        {Exercise}
\newcommand{\laPoints}      {5}
\newcommand{\laTitle}       {Propositional Logic}
\newcommand{\laDate}        {}
\setcounter{chapter}{1}
\setcounter{section}{3}
\addtocounter{section}{-1}
\newcounter{question}

\toggletrue{answerkey}
\togglefalse{answerkey}


\title{}
\author{Rachel Singh}
\date{\today}

\pagestyle{fancy}
\fancyhf{}

\lhead{\laClass, \laSemester, \laDate}

\chead{}

\rhead{\laChapter\ \laType\ \iftoggle{answerkey}{ KEY }{}}

\rfoot{\thepage\ of \pageref{LastPage}}

\lfoot{\scriptsize By Rachel Singh, last updated \today}

\renewcommand{\headrulewidth}{2pt}
\renewcommand{\footrulewidth}{1pt}

\begin{document}




\notonkey{

\footnotesize
~\\ 
\textbf{\laChapter\ \laType: } In-class exercises are meant to introduce you to a new topic
and provide some practice with the new topic. Work in a team of up to 4 people to complete this exercise.
You can work simultaneously on the problems, or work separate and then check your answers with each other.
You can take the exercise home, score will be based on the in-class quiz the following class period.
\textbf{Work out problems on your own paper} - this document just has examples and questions.

\hrulefill
\normalsize 

}{
\begin{center}
    \Large
    \textbf{Answer Key}
\end{center}
}


\notonkey{
% ASSIGNMENT ------------------------------------ %

    \section{\laTitle}

    \subsection{Propositional Logic}

    \begin{introNOHEAD}{}
        A \textbf{proposition} is a statement which has truth value: it is either true (T) or false (F).
        \footnote{From https://en.wikibooks.org/wiki/Discrete\_Mathematics/Logic}

        The statement doesn't have to necessarily be TRUE, it could also
        be FALSE, but it has to be unambiguously so. ~\\

        \begin{tabular}{ | l | l | l | }
            \hline
            \textbf{Statement} & \textbf{Proposition?} & \textbf{Result?}
            \\ \hline
            2 + 3 = 5 & Yes & True
            \\ \hline
            2 + 2 = 5 & Yes & False
            \\ \hline
            This class has 30 students & Yes & False\footnote{Probably :)}
            \\ \hline
            Is the movie good? & No &
            \\ \hline
        \end{tabular}
    \end{introNOHEAD}

    \stepcounter{question}
    \begin{questionNOGRADE}{\thequestion}
        For the following, mark whether the statement is a \textbf{proposition} and,
        if it is, mark whether it is \textbf{true} or \textbf{false}.

        \begin{center}
            \begin{tabular}{l l | c | c}
                & \textbf{statement} & \textbf{is proposition?} & \textbf{is true/false?}
                \\ \hline
                a. & 10 + 20 = 30
                & & \\
                
                b. & 2 times an integer is always even.
                & & \\

                c. & Is $4 < 5$?
                & & \\

                d. & My cat is the best cat.
                & & \\

                e. & $x \cdot y$ is odd.
                & & \\

                f. & $5 + 1$ is odd.
                & & \\
            \end{tabular}
        \end{center}

    \end{questionNOGRADE}

    \newpage

    \subsection{Compound Propositions}

    \begin{introNOHEAD}{}
        We can also create a \textbf{compound proposition} using the logical operators
        for AND $\land$, OR $\lor$, and NOT $\neg$. When we're writing out
        a compound proposition, we will usually assign each proposition a
        \textbf{propositional variable}.

        ~\\ If we have the propositions
        $p$ is the proposition ``Paul is taking discrete math",
        $q$ is the proposition, ``Paul has a calculator",
        then we can build the compound propositions like:
        \begin{itemize}
            \item[] $p \land q$: Paul is taking discrete math and Paul has a calculator
            \item[] $p \lor q$: Paul is taking discrete math or Paul has a calculator
            \item[] $\neg p$: Paul is NOT taking discrete math
            \item[] $\neg q$: Paul does NOT have a calculator
        \end{itemize}

        The result of a compound proposition depends on the value of
        each proposition it is made up of:

        \begin{enumerate}
            \item A compound proposition $a \land b \land c$ is \textbf{only true}
                if all propositions are true; it will be \textbf{false} if
                one or more of the propositions is false.
            \item A compound proposition $a \lor b \lor c$ is \textbf{true}
                if one or more of the propositions is true; it is \textbf{only false}
                if all propositions are false.
            \item A compound proposition $\neg a$ is \textbf{true} only
                if the proposition is false; it is only \textbf{false} if
                the proposition is true.
        \end{enumerate}

    \end{introNOHEAD}

    \newpage

    \stepcounter{question}
    \begin{questionNOGRADE}{\thequestion}
        Given the following compound propositions and proposition values,
        write out whether the full compound proposition is \textbf{true} or \textbf{false}.
    \end{questionNOGRADE}

        \paragraph{a AND b:} ~\\

        \begin{tabular}{ | l  c | c | p{6cm} | }
            \hline
            & \textbf{Compound} & \textbf{Values} & \textbf{Result}
            \\ \hline

            a. &        $a \land b$ &       $a = true, b = true$ &       TRUE   \\ \hline
            b. &        $a \land b$ &       $a = true, b = false$ &       FALSE   \\ \hline
            c. &        $a \land b$ &       $a = false, b = true$ &       FALSE   \\ \hline
            d. &        $a \land b$ &       $a = false, b = false$ &       FALSE   \\ \hline
        \end{tabular}

        \paragraph{a OR b:} ~\\

        \begin{tabular}{ | l  c | c | p{6cm} | }
            \hline
            & \textbf{Compound} & \textbf{Values} & \textbf{Result}
            \\ \hline

            e. &        $a \lor b$ &       $a = true, b = true$ &     \\ \hline
            f. &        $a \lor b$ &       $a = true, b = false$ &    \\ \hline
            g. &        $a \lor b$ &       $a = false, b = true$ &    \\ \hline
            h. &        $a \lor b$ &       $a = false, b = false$ &   \\ \hline
        \end{tabular}

        \paragraph{Combinations:} ~\\

        \begin{tabular}{ | l  c | c | p{6cm} | }
            \hline
            & \textbf{Compound} & \textbf{Values} & \textbf{Result}
            \\ \hline

            i. &        $a \land \neg b$ &       $a = true, b = false$ &   \\ \hline
            j. &        $a \lor \neg b$ &        $a = false, b = true$ &   \\ \hline

            k. &        $\neg a \land b$ &       $a = false, b = true$ &   \\ \hline
            l. &        $\neg a \lor b$ &        $a = false, b = false$ &  \\ \hline
        \end{tabular}

    \newpage

    \stepcounter{question}
    \begin{questionNOGRADE}{\thequestion}
        For the following, ``translate" the following English statements
        into compound propositions.

        ~\\ $p:$ ``The printer is offline" \tab
        $q:$ ``The printer is out of paper" \\ 
        $r:$ ``The document has finished printing"


        \begin{enumerate}
            \item[a.] The printer is not out of paper.
            \item[b.] The printer is online.
            \item[c.] The printer is offline and it is out of paper
            \item[d.] The printer is online and it is not out of paper.
            \item[e.] Either the printer is online, or it is out of paper.
            \item[f.] The printer is online, but it is out of paper.
            \item[g.] The printer is offline or it is out of paper, but not both.
            \item[h.] The printer is online and the printer has paper, and the document has not finished printing.
        \end{enumerate}
    \end{questionNOGRADE}

    \newpage

    \subsection{Truth Tables}

    \begin{introNOHEAD}{}
        When we're working with compound propositional statements,
        the result of the compound depends on the true/false values
        of each proposition it is built up of.
        ~\\
        We can diagram out all possible states of a compound proposition
        by using a \textbf{truth table}. In a truth table, we list
        all propositional variables first on the left, as well as
        all possible combinations of their states, and then
        the compound statement's result on the right.

        ~\\

        \begin{tabular}{ l l l }

            \begin{tabular}{ | c | c | c | c | }
                \hline
                $p$ & $q$ & & $p \land q$ \\ \hline
                T & T & & T \\ \hline
                T & F & & F \\ \hline
                F & T & & F \\ \hline
                F & F & & F \\ \hline

            \end{tabular}
            &

            \begin{tabular}{ | c | c | c | c | }
                \hline
                $p$ & $q$ & & $p \lor q$ \\ \hline
                T & T & & T \\ \hline
                T & F & & T \\ \hline
                F & T & & T \\ \hline
                F & F & & F \\ \hline

            \end{tabular}
            &

            \begin{tabular}{ | c | c | c | }
                \hline
                $p$ & & $\neg p$ \\ \hline
                T & & F \\ \hline
                F & & T \\ \hline

            \end{tabular}

        \end{tabular}
    \end{introNOHEAD}


    \stepcounter{question}
    \begin{questionNOGRADE}{\thequestion}
        Complete the following truth tables. ~\\
        
        ~\\
        a.
        \begin{tabular}{ | p{1cm} | p{1cm} | c | p{2cm} | p{3cm} | }
            \hline
            $p$ & $q$ & & $\neg q$ & $p \land \neg q$ \\ \hline
            T & T & & F & \\ \hline
            T & F & & T & \\ \hline
            F & T & &  & \\ \hline
            F & F & &  & \\ \hline
        \end{tabular}

        ~\\~\\
        b.
        \begin{tabular}{ | p{1cm} | p{1cm} | c | p{2cm} | p{2cm} | p{3cm} | }
            \hline
            $p$ & $q$ & & $\neg p$ & $\neg q$ & $\neg p \lor \neg q$ \\ \hline
            T & T & & & & \\ \hline
            T & F & & & & \\ \hline
            F & T & & & & \\ \hline
            F & F & & & & \\ \hline
        \end{tabular}
    \end{questionNOGRADE}

    \newpage
    \begin{introNOHEAD}{}
        When you're building out truth tables, there is a specific order
        you should write out the ``T" and ``F" states.
        Begin with all ``T" values first, and work your way down to
        all ``F" values first. As you go, change the right-most state
        from ``T" to ``F", working your way from right-to-left.

        ~\\
        So with two variables: \\ ``TT", ``TF", ``FT", ``FF".
    \end{introNOHEAD}
    
    \stepcounter{question}
    \begin{questionNOGRADE}{\thequestion}
        Using the rules above, write out all the states for a
        truth table with three propositional variables. ~\\
        
        \begin{center}
            \begin{tabular}{ | p{2cm} | p{2cm} | p{2cm} | }
                \hline
                $p$ & $q$ & $r$   \\ \hline
                T & T & T         \\ \hline
                T & T & F         \\ \hline
                T &  &          \\ \hline
                T & F & F         \\ \hline
                F &  &          \\ \hline
                F &  &          \\ \hline
                F &  &          \\ \hline
                F & F & F         \\ \hline
            \end{tabular}
        \end{center}
    \end{questionNOGRADE}

    ~\\
        
    \newpage

    \begin{introNOHEAD}{}
        Whenever the final columns of the truth tables for two propositions $p$ and $q$ are the same,
        we say that $p$ and $q$ are \textbf{logically equivalent}, and we write: $p \equiv q$.
        \footnote{From https://en.wikibooks.org/wiki/Discrete\_Mathematics/Logic}
    \end{introNOHEAD}




    \stepcounter{question}
    \begin{questionNOGRADE}{\thequestion}
        Use a truth table to show that the compound propositions,

        \begin{center}
        $ \neg p \land \neg q $
        \tab and \tab
        $ \neg ( p \lor q ) $
        \end{center}
        are logically equivalent. The final two columns are
        the compound propositions above
        ~\\
        \begin{center}
            \begin{tabular}{ | c | c | c | c | c | c  c | c | c | c | c | c | }
                \hline
                $p$ &
                $q$ & &

                $\neg p$ &
                $\neg q$ &

                $(p \lor q)$ & & &

                $ \neg p \land \neg q $ & &
                $ \neg ( p \lor q ) $
                \\ \hline

                T & T & & F & F & & & & & & \\ \hline
                T & F & & F & T & & & & & & \\ \hline
                F & T & & T & F & & & & & & \\ \hline
                F & F & & T & T & & & & & & \\ \hline
            \end{tabular} 
        \end{center}       
    \end{questionNOGRADE}




}{
% KEY ------------------------------------ %

    \begin{enumerate}
        \item[1a.]  Is a proposition, is true.
        \item[1b.]  Is a proposition, is true.
        \item[1c.]  Not a proposition - it is a question.
        \item[1d.]  Not a proposition - it is an opinion.
        \item[1e.]  Not a proposition - because of variables, it is not unambiguously true or false.
        \item[1f.]  Is a proposition, is false.

        \item[2a.]  true
        \item[2b.]  false
        \item[2c.]  false
        \item[2d.]  false

        \item[2e.]  true
        \item[2f.]  true
        \item[2g.]  true
        \item[2h.]  false

        \item[2i.]  true
        \item[2j.]  false
        \item[2k.]  true
        \item[2l.]  true

        \item[3a.]  $\neg q$
        \item[3b.]  $\neg p$
        \item[3c.]  $p \land q$
        \item[3d.]  $\neg p \land \neg q$
        \item[3e.]  $p \lor q$
        \item[3f.]  $\neg p \land q$
        \item[3g.]  $(p \lor q) \land (p \land q)$
        \item[3h.]  $\neg p \land \neg q \land \neg r$

        \item[4a.]
            \begin{tabular}{ | p{1cm} | p{1cm} | c | p{2cm} | p{3cm} | }
                \hline
                $p$ & $q$ & & $\neg q$ & $p \land \neg q$ \\ \hline
                T & T & & F & F 
                \\ \hline
                T & F & & T & T 
                \\ \hline
                F & T & & F & F 
                \\ \hline
                F & F & & T & F 
                \\ \hline
            \end{tabular}

        \item[4b.]
            \begin{tabular}{ | p{1cm} | p{1cm} | c | p{2cm} | p{2cm} | p{3cm} | }
                \hline
                $p$ & $q$ & & $\neg p$ & $\neg q$ & $\neg p \lor \neg q$ \\ \hline
                T & T &
                &  F 
                &  F 
                &  F 
                \\ \hline
                T & F &
                &  F 
                &  T 
                &  T 
                \\ \hline
                F & T &
                &  T 
                &  F 
                &  T 
                \\ \hline
                F & F &
                &  T 
                &  T 
                &  T 
                \\ \hline
            \end{tabular} ~\\

    \item[5.]
        \begin{tabular}{ | p{2cm} | p{2cm} | p{2cm} | }
            \hline
            $p$ & $q$ & $r$
            \\ \hline
            T & T & T
            \\ \hline
            T & T & F
            \\ \hline
            T &  F  &  T 
            \\ \hline
            T & F & F
            \\ \hline
            F &  T  &  T 
            \\ \hline
            F &  T  &  F 
            \\ \hline
            F &  F  &  T 
            \\ \hline
            F & F & F
            \\ \hline
        \end{tabular}

    \item[6.]
        \begin{tabular}{ | c | c | c | c |  c | c | c | c | c | c | }
            \hline
                $p$ &
                $q$ & &

                $\neg p$ &
                $\neg q$ &

                $(p \lor q)$ & &

                $ \neg p \land \neg q $ & &
                $ \neg ( p \lor q ) $
                \\ \hline
            T & T & & F & F
            &  T  &
            &  T  &
            &  T 
            \\ \hline
            T & F & & F & T
            &  T  &
            &  F  &
            &  F 
            \\ \hline
            F & T & & T & F
            &  T  &
            &  F  &
            &  F 
            \\ \hline
            F & F & & T & T
            &  F  &
            &  F  &
            &  F 
            \\ \hline
        \end{tabular}
    \end{enumerate}
}



\end{document}

