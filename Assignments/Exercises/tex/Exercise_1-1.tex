\documentclass[a4paper,12pt,oneside]{book}
\usepackage[utf8]{inputenc}

\usepackage{rachwidgets}


\newcommand{\laClass}       {CS 210}
\newcommand{\laSemester}    {Spring 2018}
\newcommand{\laChapter}     {1.1}
\newcommand{\laType}        {Exercise}
\newcommand{\laPoints}      {5}
\newcommand{\laTitle}       {First Examples}
\newcommand{\laDate}        {Week 1}
\setcounter{chapter}{5}
\setcounter{section}{1}
\addtocounter{section}{-1}
\newcounter{question}

\toggletrue{answerkey}
\togglefalse{answerkey}

\input{BASE-HEADER}

\input{INSTRUCTIONS-EXERCISE}

\notonkey{
    \section{\laTitle}

    \subsection{Coin Toss}

    \begin{intro}{Coin toss}
        When we're flipping a coin, there are two possible outcomes:
        \textit{heads} or \textit{tails}. If we flip more than one
        coin, then we end up with more possible outcomes. For example,
        when flipping two coins, we have four possible outcomes:

        \begin{center}
            \begin{tabular}{ | l | c | c | }
                \hline
                1. & HEADS & HEADS \\ \hline
                2. & HEADS & TAILS \\ \hline
                3. & TAILS & HEADS \\ \hline
                4. & TAILS & TAILS \\ \hline
            \end{tabular}
        \end{center}
    \end{intro}

    % - QUESTION --------------------------------------------------%
    \stepcounter{question}
    \begin{questionNOGRADE}{\thequestion}
        Draw a table of all possible outcomes if someone flips three coins.
    \end{questionNOGRADE}

    % - QUESTION --------------------------------------------------%
    \stepcounter{question}
    \begin{questionNOGRADE}{\thequestion}
        Write out how many outcomes there are for each of the following.

        \begin{tabular}{p{6cm} p{6cm}}
            a. Flipping one coin &
            b. Flipping two coins \\
            c. Flipping three coins &
            d. Rolling one 6-sided die \\
            e. Rolling two 6-sided dice &
            f. Rolling three 6-sided dice
        \end{tabular}
    \end{questionNOGRADE}

    % - QUESTION --------------------------------------------------%
    \stepcounter{question}
    \begin{questionNOGRADE}{\thequestion}
        Using the variables \texttt{outcomesPerEvent},
        \texttt{eventExecutions}, and \texttt{totalOutcomes},
        write an equation (line of code) that uses \texttt{outcomesPerEvent}
        and \texttt{eventExecutions} to find the value of \texttt{totalOutcomes}.
        Use $\wedge$
        to represent exponent.

        \texttt{totalOutcomes = }
    \end{questionNOGRADE}

    \newpage

    \subsection{Josephus game}

    \begin{intro}{The Josephus game}
        The Josephus game is a theoretical problem, and for now we
        will just solve it systematically by stepping through the
        instructions given.

        \paragraph{Setup:} People are sitting in a circle, each with
        an assigned number (their location in the circle).

        \paragraph{Step:} People are eliminated at every $n$th step,
        with counting beginning at the $n$th person.

        \paragraph{Result:} After stepping through, figure out the
        position of the \textit{last} and \textit{second-to-last}
        person left (not eliminated).

        \paragraph{Example:} Let's say that we begin with a circle of
        \textit{10} people, numbered from 1 to 10. We will eliminate
        people at an interval of \textit{2} - so, every-other-person,
        beginning with person \#2. Who will be the last person standing?

        \includegraphics[width=12cm]{images/josephus.png}
    \end{intro}

    \newpage

    % - QUESTION --------------------------------------------------%
    \stepcounter{question}
    \begin{questionNOGRADE}{\thequestion}
        Given a Josephus circle of 15 people,
        if we are eliminating every 3rd person (starting with person 3),
        who is the last to be ``killed" – and the 2nd to last to be ``killed"?

        \begin{center}
            \includegraphics[width=6cm]{images/josephus-15.png}
        \end{center}

        \begin{hint}{Hint:}
            Don't count ``dead" people, make sure to skip over them!
        \end{hint}
        
    \end{questionNOGRADE}

    % - QUESTION --------------------------------------------------%
    \stepcounter{question}
    \begin{questionNOGRADE}{\thequestion}
        Given a Josephus circle of 10 people,
        if we are eliminating every 4th person (starting with person 4),
        who is the last to be ``killed" – and the 2nd to last to be ``killed"?

        \begin{center}
            \includegraphics[width=7cm]{images/josephus-10.png}
        \end{center}
    \end{questionNOGRADE}
    
    \subsection{Game trees}

    \begin{intro}{Game trees}
        In section 1.1, we will also be looking at events that have multiple outcomes –
        such as flipping one coin, two coins, or three coins, or who of two people win
        one, two, or three tennis matches. With small amounts of ``variables", we can
        list out all the possible outcomes, and we can build a game tree based on this.
        
        \paragraph{Example:} ~\\
        If you flip one coin, the result with be HEADS (H) or TAILS (T).
        If you flip two coins, what are all the outcomes?

        \begin{center}
            \includegraphics[width=5cm]{images/gametree.png}
            
            \begin{tabular}{l l}
                1. HH &  2. HT \\
                3. TH &  4. TT
            \end{tabular}
        \end{center}

    \end{intro}

    % - QUESTION --------------------------------------------------%
    \stepcounter{question}
    \begin{questionNOGRADE}{\thequestion}
        Suppose you toss three coins – a nickel, a dime, and a quarter,
        and you record the results in that order. For example,
        ``HTH" means nickel = heads, dime = tails, quarter = heads.

        \begin{enumerate}
            \item[a.] In a systematic way, list all the different results you could record.
            \item[b.] Draw a game tree for recording the results.
            \item[c.] On the game tree, label each possible result either 0, 1, 2, or 3,
            to indicate how many \textit{heads} there are.
            \item[d.] Do you think a person who tosses three coins is more likely to get all
            three heads, or to get exactly two heads?                
        \end{enumerate}
    \end{questionNOGRADE}
    
    
}{
    \begin{enumerate}
        \item[1.]   
                \begin{tabular}{ | l | c | c | c | }
                    \hline
                    1. & HEADS & HEADS & HEADS \\ \hline
                    2. & HEADS & HEADS & tails \\ \hline
                    3. & HEADS & tails & HEADS \\ \hline
                    4. & HEADS & tails & tails \\ \hline
                    5. & tails & HEADS & HEADS \\ \hline
                    6. & tails & HEADS & tails \\ \hline
                    7. & tails & tails & HEADS \\ \hline
                    8. & tails & tails & tails \\ \hline
                \end{tabular}

        \item[2a.]  $2^{1} = 2$
        \item[2b.]  $2^{2} = 4$
        \item[2c.]  $2^{3} = 8$
        \item[2d.]  $6^{1} = 6$
        \item[2e.]  $6^{2} = 36$
        \item[2f.]  $6^{3} = 216$

        \item[3.]   \texttt{totalOutcomes = outcomesPerEvent $\wedge$ eventExecutions}

        \item[4.]   2nd-to-last: Person 14; last: Person 5

        \item[5.]   2nd-to-last: Person 6; last: Person 5

        \item[6a.]   1. HHH, 2. HHT, 3. HTH, 4. HTT, \\ 5. THH, 6. THT, 7. TTH, 8. TTT
        \item[6b.]  \includegraphics[width=6cm]{images/ans-gametree.png}
        \item[6c.]  ...
        \item[6d.]  More likely to get exactly 2 heads.
    \end{enumerate}
}

    

\input{BASE-FOOT}
