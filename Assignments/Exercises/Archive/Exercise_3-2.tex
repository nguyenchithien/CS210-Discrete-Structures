\documentclass[a4paper,12pt]{book}
\usepackage[utf8]{inputenc}
\title{}
\author{Rachel Morris}
\date{\today}

\usepackage{rachwidgets}
\usepackage{fancyhdr}
\usepackage{lastpage}
\usepackage{dirtree}
\usepackage{boxedminipage}

\setcounter{chapter}{3}
\setcounter{section}{1}
\newcommand{\laChapter}{3.2 More Operations on Sets\ }
\newcounter{question}

\newcommand{\laClass}{CS 210\ }
\newcommand{\laSemester}{Fall 2017\ }

\pagestyle{fancy}
\fancyhf{}
\lhead{\laClass \laSemester}
\chead{}
\rhead{Ch \laChapter}
\rfoot{\thepage\ of \pageref{LastPage}}
\lfoot{\scriptsize Compiled by Rachel Morris, last updated \today}

\renewcommand{\headrulewidth}{2pt}
\renewcommand{\footrulewidth}{1pt}

\begin{document}

    %\toggletrue{answerkey}
    \togglefalse{answerkey}

    %------------------------------------------------------------------%
    %- Exercise Begin -------------------------------------------------%
    %------------------------------------------------------------------%

    \section{More Operations on Sets}

    %------------------------------------------------------------------%
    \subsection{Cartesian Products}

    \notonkey{
    \begin{intro}{\ }
        We can compute the Cartesian Product of two sets, such as $A$ and $B$.
        The result will be a set of \textbf{ordered pairs}, such as $(a, b)$, combining
        the elements of $A$ and $B$ together.

        \paragraph{Example:} For $A = \{1, 2\}$ and $B = \{4, 5, 6\}$, find $A \times B$.

        \begin{center}
            \begin{tabular}{c | c | c | c}
                & $B_{1} = 4$ & $B_{2} = 5$ & $B_{3} = 6$
                \\ \hline
                $A_{1} = 1$ & (1, 4) & (1, 5) & (1, 6)
                \\
                $A_{2} = 2$ & (2, 4) & (2, 5) & (2, 6)
            \end{tabular}
        \end{center}

        So the result is that
        $$ A \times B = \{ (1, 4), (1, 5), (1, 6), (2, 4), (2, 5), (2, 6) \} $$
    \end{intro}
    }{}


        % - QUESTION --------------------------------------------------%
        \stepcounter{question}
        \begin{questionNOGRADE}{\thequestion}

            Given the following sets, calculate each Cartesian Product.
            Write it out in a table and as a set, like above.

            $$ A = \{1, 2\} \tab B = \{3, 4\} $$

            \begin{enumerate}
                \item[a.] $A \times B = $
                    \solution{ $\{ (1, 3), (1, 4), (2, 3), (2, 4) \}$ }{}
                    \begin{center}
                        \begin{tabular}{c | c | c}
                            & $B_{1} = 3$ & $B_{2} = 4$ \\ \hline
                            $A_{1} = 1$ & \solution{$(1,3)$}{} & \solution{$(1,4)$}{} \\
                            $A_{2} = 2$ & \solution{$(2,3)$}{} & \solution{$(2,4)$}{}
                        \end{tabular}
                    \end{center}


                \item[b.] $B \times A = $
                    \solution{ $\{ (3,1), (3,2), (4,1), (4,2) \}$ }{}
                    \begin{center}
                        \begin{tabular}{c | c | c}
                            & $A_{1} = 1$ & $A_{2} = 2$ \\ \hline
                            $B_{1} = 3$ & \solution{$(3,1)$}{} & \solution{$(3,2)$}{} \\
                            $B_{2} = 4$ & \solution{$(4,1)$}{} & \solution{$(4,2)$}{}
                        \end{tabular}
                    \end{center}

            \end{enumerate}

        \end{questionNOGRADE}

        \notonkey{ \newpage }{ \hrulefill }


        % - QUESTION --------------------------------------------------%
        \stepcounter{question}
        \begin{questionNOGRADE}{\thequestion}

            Given the following sets, calculate each Cartesian Product.
            Write it out in a table and as a set.

            $$ A = \{x, y, z\} \tab B = \{1, 3\} $$

            \begin{enumerate}
                \item[a.] $A \times B = $
                    \solution{ $\{ (x,1), (x,3), (y,1), (y,3), (z,1), (z,3) \}$ }{}
                    \begin{center}
                        \begin{tabular}{c | c | c}
                            & $B_{1} = 1$ & $B_{2} = 3$ \\ \hline
                            $A_{1} = x$ & \solution{ $(x,1)$ }{} & \solution{ $(x,3)$ }{}  \\
                            $A_{2} = y$ & \solution{ $(y,1)$ }{} & \solution{ $(y,3)$ }{} \\
                            $A_{3} = z$ & \solution{ $(z,1)$ }{} & \solution{ $(z,3)$ }{}
                        \end{tabular}
                    \end{center}


                \item[b.] $B \times A = $
                    \solution{ $\{ (1,x), (1,y), (1,z), (3,x), (3,y), (3,z) \}$ }{}
                    \begin{center}
                        \begin{tabular}{c | c | c | c}
                            & $A_{1} = x$ & $A_{2} = y$ & $A_{3} = z$ \\ \hline
                            $B_{1} = 1$ &  \solution{ $(1,x)$ }{} &  \solution{ $(1,y)$ }{} & \solution{ $(1,z)$ }{} \\
                            $B_{2} = 3$ &  \solution{ $(3,x)$ }{} &  \solution{ $(3,y)$ }{} & \solution{ $(3,z)$ }{}
                        \end{tabular}
                    \end{center}

            \end{enumerate}

        \end{questionNOGRADE}

        \hrulefill

        % - QUESTION --------------------------------------------------%
        \stepcounter{question}
        \begin{questionNOGRADE}{\thequestion}

            Calculate the Cartesian Products and write out the result
            as a set of coordinate pairs.

            $$ A = \{2, 4\} \tab B = \{1, 3\} \tab C = \{3, 4, 5, 6\} $$

            \begin{enumerate}
                \item[a.]   $A \times B$    \solution{
                    $\{2, 4\} \times \{1, 3\} = \{ (2, 1), (2, 3), (4, 1), (4, 3) \}$
                }{ ~\\ \raisebox{0pt}[1cm][0pt]{  } }

                \item[b.]   $A \times C$    \solution{
                    $\{2, 4\} \times \{3, 4, 5, 6\} =
                    \{  (2, 3), (2, 4), (2, 5), (2, 6),
                        (4, 3), (4, 4), (4, 5), (4, 6) \}$
                }{ ~\\ \raisebox{0pt}[1cm][0pt]{  } }

                \item[c.]   $B \times C$    \solution{
                    $\{1, 3\} \times \{3, 4, 5, 6\} = \{
                        (1, 3), (1, 4), (1, 5), (1, 6),
                        (3, 3), (3, 4), (3, 5), (3, 6) \}$
                }{ ~\\ \raisebox{0pt}[1cm][0pt]{  } }

                \item[d.]   $A^{2}$ (Hint: $A \times A$)    \solution{
                    $\{2, 4\} \times \{2, 4\} = \{
                        (2, 2), (2, 4),
                        (4, 2), (4, 4) \}$
                }{ ~\\ \raisebox{0pt}[1cm][0pt]{  } }
            \end{enumerate}

        \end{questionNOGRADE}

        \notonkey{ \newpage }{ \hrulefill }

        % - QUESTION --------------------------------------------------%
        \stepcounter{question}
        \begin{questionNOGRADE}{\thequestion}

            With the given sets, find the intersections, unions, and differences.

            $$ A = \{1\} \tab B = \{3, 5, 7\} \tab C = \{3, 5, 9, 11\} $$
            $$ A \times B = \{ (1, 3), (1, 5), (1, 7) \} \tab{}
                A \times C = \{ (1, 3), (1, 5), (1, 9), (1, 11) \} $$

            \begin{enumerate}
                \item[a.]   $(A \times B) - (A \times C)$
                    \solution{
                    $\{(1,7)\}$
                    }{ \vspace{1cm} }

                \item[b.]   $(A \times C) - (A \times B)$
                    \solution{
                    $\{(1,9), (1,11)\}$
                    }{ \vspace{1cm} }

                \item[c.]   $A \times (B \cup C)$
                    \solution{ \\
                    $B \cup C = \{3, 5, 7, 9, 11\}$ \\
                    $A \times (B \cup C) = \{ (1,3), (1,5), (1,7), (1,9), (1,11) \}$
                    }{ \vspace{1cm} }

                \item[d.]   $(A \times (B \cup C)) \cap (A \times B)$
                    \solution{
                    $\{ (1,3), (1,5), (1,7) \}$
                    }{ \vspace{1cm} }

                \item[e.]   $(A \times B) \cup (A \times C)$
                    \solution{
                    $\{ (1,3), (1,5), (1,7), (1,9), (1,11) \}$
                    }{ \vspace{1cm} }
            \end{enumerate}

        \end{questionNOGRADE}

        \notonkey{ \newpage }{ \hrulefill }

    %------------------------------------------------------------------%
    \subsection{Partitions}

    \notonkey{
    \begin{intro}{\ }
        The Partition of a set, usually denoted by $S$, is a set of subsets
        that, when combined together, form the original set.

        \paragraph{Definition:}
            For a set $A$, a partition of $A$ is a set $S = \{ S_{1}, S_{2}, S_{3}, ... \}$
            of subsets of $A$, such that:
            \begin{enumerate}
                \item   For all $i$, $S_{i} \neq \emptyset$; that is, each part is non-empty.
                \item   For all $i$ and $j$, if $S_{i} \neq S_{j}$, then $S_{i} \cap S_{j} = \emptyset$;{}
                    that is, different \textit{parts} have nothing in common.
                \item   $S_{1} \cup S_{2} \cup S_{3} \cup ... = A$; that is,
                    every element in $A$ is in some part.
            \end{enumerate}

            \subparagraph{Clarifications:}
                The elements in $S$, such as $S_{i}$, are just sets of elements that
                contain elements from $A$. None of the elements of $A$ can be duplicated
                across multiple elements from $S$. And, all elements of $A$ must be
                present in $S$.


            \paragraph{Example:}
                Let's say we have a set $A = \{1, 2, 3, 4\}$, we could form multiple partitions, such as:

                \begin{itemize}
                    \item   Partition 1:    $\{ \{1\}, \{2\}, \{3\}, \{4\} \}$
                    \item   Partition 2:    $\{ \{1, 2\}, \{3, 4\} \}$
                    \item   Partition 3:    $\{ \{1, 2, 3\}, \{4\} \}$
                    \item   Partition 4:    $\{ \{1, 2, 3, 4\} \}$
                \end{itemize}

                Essentially, it can be any combination of subsets of whatever size,
                so long as all elements of $A$ are represented in the partition.
    \end{intro}
    }{}


        % - QUESTION --------------------------------------------------%
        \stepcounter{question}
        \begin{questionNOGRADE}{\thequestion}

            For the given set, write out all possible partitions of
            $ A = \{1, 2\} $.
            There should be 2. Note that the \textit{order} of the
            elements of the set does not matter.

            \solution{
                $\{ \{1\}, \{2\} \}$ and $\{ \{1, 2\} \}$
            }{}

        \end{questionNOGRADE}

        \notonkey{ \newpage }{ \hrulefill }

        % - QUESTION --------------------------------------------------%
        \stepcounter{question}
        \begin{questionNOGRADE}{\thequestion}

            For the given set, write out all possible partitions.
            There should be 5.

            $$ B = \{1, 2, 3\} $$

            \solution{
                \begin{enumerate}
                    \item   $\{ \{1\}, \{2\}, \{3\} \}$
                    \item   $\{ \{1, 2\}, \{3\} \}$
                    \item   $\{ \{1\}, \{2, 3\} \}$
                    \item   $\{ \{1, 3\}, \{2\} \}$
                    \item   $\{ \{1, 2, 3\} \}$
                \end{enumerate}
            }{
                \begin{enumerate}
                    \item
                    \item
                    \item
                    \item
                    \item
                \end{enumerate}
            }

        \end{questionNOGRADE}

        \hrulefill

        % - QUESTION --------------------------------------------------%
        \stepcounter{question}
        \begin{questionNOGRADE}{\thequestion}

            Which of the following are valid partitions of the set
            $A = \{1, 2, 4, 8, 16, 32, 64, 128 \}$? For those that
            are not, explain why not.

            \begin{enumerate}
                \item[a.]   $\{ 1, 2, \{4, 8, 16\}, \{32, 64, 128\} \}$
                    \solution{ Not valid - 1 and 2 aren't sets }{ ~\\ \raisebox{0pt}[1cm][0pt]{  } }

                \item[b.]   $\{ \{1, 16\}, \{32, 64, 2\}, \{8, 4, 16\}, \{ 128 \} \}$
                    \solution{ Not valid - 16 shows up twice. } { ~\\ \raisebox{0pt}[1cm][0pt]{  } }

                \item[c.]   $\{ \{1, 128\}, \{8, 4, 16\}, \{64, 2\} \}$
                    \solution{ Not valid - 32 is missing. }{ ~\\ \raisebox{0pt}[1cm][0pt]{  } }

                \item[d.]   $\{ \{8, 4, 2\}, \{16, 1, 128\}, \{32, 64\} \}$
                    \solution{ Valid partition }{ ~\\ \raisebox{0pt}[1cm][0pt]{  } }
            \end{enumerate}

        \end{questionNOGRADE}

        \notonkey{ \newpage }{ \hrulefill }

        % - QUESTION --------------------------------------------------%
        \stepcounter{question}
        \begin{questionNOGRADE}{\thequestion}

            For the set $A = \{1, 2, 3, 4, 5, 6\}$, build partitions that
            meet the following criteria:

            \begin{enumerate}
                \item[a.]   Find a partition where each part has the same size.
                    \solution{ Multiple solutions }{ ~\\ \raisebox{0pt}[1cm][0pt]{  } }

                \item[b.]   Find a partition where no two parts have the same size.
                    \solution{ Multiple solutions }{ ~\\ \raisebox{0pt}[1cm][0pt]{  } }

                \item[c.]   Find a partition that has as many parts as possible.
                    \solution{ Multiple solutions }{ ~\\ \raisebox{0pt}[1cm][0pt]{  } }

                \item[d.]   Find the partition that has as few parts as possible.
                    \solution{ Multiple solutions }{ ~\\ \raisebox{0pt}[1cm][0pt]{  } }
            \end{enumerate}

        \end{questionNOGRADE}

        \notonkey{ \newpage }{ \hrulefill }

    %------------------------------------------------------------------%
    \subsection{Power Sets}

        \notonkey{
        \begin{intro}{\ }
            The Power Set of $A$ is defined as
            $\wp(A) = \{ S : S \subseteq A \}$. In other words, the Power Set
            is a set of all possible subsets you could build from $A$, including
            the empty set.

            \paragraph{Example 1:} Find the Power Set of $\{A\}$.

            $$\wp(\{A\}) = \{ \emptyset, \{A\} \}$$

            \paragraph{Example 2:} Find the Power Set of $\{A, B\}$.

            $$\wp(\{A, B\}) = \{ \emptyset, \{A\}, \{B\}, \{A, B\} \}$$

            \paragraph{Example 3:} Find the Power Set of $\{A, B, C\}$.

            $$\wp(\{A, B, C\}) =
                \{ \emptyset,
                    \{A\}, \{B\}, \{C\},
                    \{A, B\}, \{B, C\}, \{A, C\},
                    \{A, B, C\}
                \}$$

            \paragraph{Example 4:} Find the Power Set of $\{A, B, C, D\}$.

            ~\\
            $\wp(\{A, B, C, D\}) = \{ \\
                    \tab \emptyset, \\
                    \tab \{ A \}, \{ B \}, \{ C \}, \{ D \}, \\
                    \tab \{ A, B \}, \{ A, C \}, \{ A, D \},
                     \{ B, C \}, \{ B, D \},
                     \{ C, D \}, \\
                    \tab \{ A, B, C \}, \{ A, B, D \}, \{ A, C, D \}, \{ B, C, D \}, \\
                    \tab \{ A, B, C, D \} \\
                \}$

            (Phew!)
        \end{intro}
        }{}

        % - QUESTION --------------------------------------------------%
        \stepcounter{question}
        \begin{questionNOGRADE}{\thequestion}

            Find the Power Set for each.

            \begin{enumerate}
                \item[a.]   $\wp( \{1, 2\} ) = $
                    \solution{
                        $\{ \emptyset, \{1\}, \{2\}, \{1, 2\}$
                    }{ }
                \item[b.]   $\wp( \{3, 4\} ) = $
                    \solution{
                        $\{ \emptyset, \{3\}, \{4\}, \{3, 4\}$
                    }{ }
                \item[c.]   $\wp( \{1, 2, 3\} ) = $
                    \solution{
                        $\{ \emptyset, \{1\}, \{2\}, \{3\}, \{1, 2\}, \{2, 3\}, \{1, 3\}, \{1, 2, 3\}$
                    }{ }
            \end{enumerate}

        \end{questionNOGRADE}







\end{document}








