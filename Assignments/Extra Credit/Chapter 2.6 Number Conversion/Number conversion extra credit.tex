\documentclass[a4paper,12pt]{book}
\usepackage[utf8]{inputenc}
\title{}
\author{Rachel Morris}
\date{\today}

\usepackage{rachwidgets}
\usepackage{fancyhdr}
\usepackage{lastpage}
\usepackage{dirtree}
\usepackage{boxedminipage}

\setcounter{chapter}{2}
\setcounter{section}{6}
\newcommand{\laChapter}{Extra credit - number conversion\ }
\newcounter{question}

\newcommand{\laClass}{CS 210\ }
\newcommand{\laSemester}{Fall 2017\ }

\pagestyle{fancy}
\fancyhf{}
\lhead{\laClass \laSemester}
\chead{}
\rhead{Ch \laChapter}
\rfoot{\thepage\ of \pageref{LastPage}}
\lfoot{\scriptsize Compiled by Rachel Morris, last updated \today}

\renewcommand{\headrulewidth}{2pt}
\renewcommand{\footrulewidth}{1pt}

\begin{document}

    %\toggletrue{answerkey}
    \togglefalse{answerkey}

    %------------------------------------------------------------------%
    %- Exercise Begin -------------------------------------------------%
    %------------------------------------------------------------------%

    \section{Numerical Representation extra credit}

    \subsection{Programming a converter}

        Let's take the algorithm given by the textbook and write
        a program to do our conversions for us.
    
            \begin{intro}{Algorithm for converting a decimal number to base $b$:}
                \begin{enumerate}
                    \item   Input a natural number $n$
                    \item   While $n > 0$, do the following:
                        \begin{enumerate}
                            \item   Divide $n$ by $b$ and get a quotient $q$ and remainder $r$.
                            \item   Write $r$ as the next (right-to-left) digit.
                            \item   Replace the value of $n$ with $q$, and repeat.
                        \end{enumerate}
                \end{enumerate}
            \end{intro}

        Open up a Python IDE (e.g., IDLE, Wing) and start with the following code,
        which includes a function definition and the main program loop:

\begin{lstlisting}[style=pycode]
# Function definition
def ConvertFromDecimal( n, b ):
    print( "" )
    print( "n = " + str( n ) + ", b = " + str( b ) )

    number = ""

    return number

# Program
while( True ):
    n = input( "Enter a base-10 number to convert: " )
    b = input( "Enter a base to convert it to: " )

    result = ConvertFromDecimal( n, b )

    print( "Result: " + result )
\end{lstlisting}

        We are going to update the \texttt{ConvertFromDecimal}
        function to follow the algorithm above.

    \newpage

    We need to begin implementing the algorithm from step 2.
    For the step ``While $n > 0$, do the following:", write the Python
    code:

\begin{verbatim}
while ( n > 0 ):
\end{verbatim}

    Note that in Python, the inside of a while loop is specified by
    indenting all inner code forward one level; Python doesn't use
    curly braces like C++, Java, or C\# does.
    
\begin{lstlisting}[style=pycode]
def ConvertFromDecimal( n, b ):
    # Now we're inside the function...
    print( "" )
    print( "n = " + str( n ) + ", b = " + str( b ) )

    number = ""

    print( "" )
    while ( n > 0 ):
        # Now we're inside the while loop...
\end{lstlisting}

    Next, within the while loop, we need to calculate the quotient $q$
    and the remainder $r$, which we can use with division and modulus.
    This is step 2-a.

\begin{verbatim}
q = n / b
r = n % b
\end{verbatim}

    How does a normal division give us the correct value? Because
    we are treating $n$ and $b$ as integers (not floats or decimals),
    so it is \textbf{integer division}. In programming, this means
    it truncates any remainder.

    We can print out the results like this:

\begin{verbatim}
print( str( n ) + "/" + str( b ) + " = "
        + str( q ) + " + " + str( r ) + "/" + str( b ) )
\end{verbatim}

    Now we add $r$ onto our number string, following step 2-b:

\begin{verbatim}
number = str( r ) + number
\end{verbatim}

    And, finally, we replace $n$ with $q$ - step 2-c:

\begin{verbatim}
n = q
\end{verbatim}

    At the return of the function, the number is returned.

    \newpage

    Full code:

\begin{lstlisting}[style=pycode]
# Function definition
def ConvertFromDecimal( n, b ):
    print( "" )
    print( "n = " + str( n ) + ", b = " + str( b ) )

    number = ""

    print( "" )
    while ( n > 0 ):
        q = n / b
        r = n % b

        print( str( n ) + "/" + str( b ) + " = " + str( q ) + " + " + str( r ) + "/" + str( b ) )

        number = str( r ) + number
        n = q

    return number

# Program
while( True ):
    n = input( "Enter a base-10 number to convert: " )
    b = input( "Enter a base to convert it to: " )

    result = ConvertFromDecimal( n, b )

    print( "" )
    print( "Result: " + result )
    print( "\n" )
\end{lstlisting}

    \newpage

    Example output:

\begin{lstlisting}[style=output]
Enter a base-10 number to convert: 23
Enter a base to convert it to: 2

n = 23, b = 2

23/2 = 11 + 1/2
11/2 = 5 + 1/2
5/2 = 2 + 1/2
2/2 = 1 + 0/2
1/2 = 0 + 1/2

Result: 10111


Enter a base-10 number to convert: 65
Enter a base to convert it to: 16

n = 65, b = 16

65/16 = 4 + 1/16
4/16 = 0 + 4/16

Result: 41
\end{lstlisting}
    
\end{document}








