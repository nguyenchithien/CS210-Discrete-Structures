\documentclass[a4paper,12pt]{book}
\usepackage[utf8]{inputenc}
\title{}
\author{Rachel Morris}
\date{\today}

\usepackage{rachwidgets}
\usepackage{fancyhdr}
\usepackage{lastpage}
\usepackage{boxedminipage}

\newcommand{\laChapter}{1.3\ }

\pagestyle{fancy}
\fancyhf{}
\lhead{CS 210}
\chead{Fall 2017}
\rhead{Ch \laChapter Exercise}
\rfoot{\thepage\ of \pageref{LastPage}}
\lfoot{\scriptsize By Rachel Morris, last updated \today}

\renewcommand{\headrulewidth}{2pt}
\renewcommand{\footrulewidth}{1pt}

\begin{document}

    %\toggletrue{answerkey}
    \togglefalse{answerkey}

    %------------------------------------------------------------------%
    %- INSTRUCTIONS ---------------------------------------------------%
    %------------------------------------------------------------------%

    \chapter*{Chapter \laChapter In-class Exercise} \stepcounter{chapter}

    \iftoggle{answerkey}{
      \begin{answer} ANSWER KEY \end{answer}
    }{}

    \paragraph{Info:}
    In-class exercises are meant to introduce you to the new topics
    of this chapter of the book. Each part will have an introductory
    description of the content and example(s), followed by practice
    problems for you to work on. ~\\

    These assignments are \textbf{team assignments} - your team will
    turn in \textit{one} copy of the exericse. It is up to your team
    how to appropach the assignments; you can work separately and then
    check your work together, or you can collaborate on the assignment
    together. ~\\

    Work must be clean; points may be deducted if the instructor cannot
    read the work.

    \hrulefill{}

    \newpage{}

    %------------------------------------------------------------------%
    %- Exercise Begin -------------------------------------------------%
    %------------------------------------------------------------------%
    \begin{center} \section*{Chapter \laChapter In-class Exercise Instructions} \end{center}

    \begin{center}
        \textbf{This is the instruction page,
        make sure to fill out your answers in the worksheet later in the document}
    \end{center}

    %------------------------------------------------------------------%
    \section*{1. Propositional Logic}

        \begin{introNOHEAD}
            A \textbf{proposition} is a statement which has truth value: it is either true (T) or false (F).
            \footnote{From https://en.wikibooks.org/wiki/Discrete\_Mathematics/Logic}

            The statement doesn't have to necessarily be TRUE, it could also
            be FALSE, but it has to be unambiguously so. ~\\

            \begin{tabular}{ | l | l | l | }
                \hline
                \textbf{Statement} & \textbf{Proposition?} & \textbf{Result?}
                \\ \hline
                2 + 3 = 5 & Yes & True
                \\ \hline
                2 + 2 = 5 & Yes & False
                \\ \hline
                This class has 30 students & Yes & False\footnote{Probably :)}
                \\ \hline
                Is the movie good? & No &
                \\ \hline
            \end{tabular}
        \end{introNOHEAD}


        % - QUESTION --------------------------------------------------%
        \begin{question}{1}{8\%}
            For the following, mark whether the statement is a \textbf{proposition} and,
            if it is, mark whether it is \textbf{true} or \textbf{false}.
        \end{question}

        \begin{enumerate}
            \item[a.] 10 + 20 = 30.
            \item[b.] $2 \cdot a$ is always even.
            \item[c.] Is $4 < 5$?
            \item[d.] Pineapple belongs on pizza.\footnote{I'm not going to count off points whether you say true or false :)}
        \end{enumerate}

    %------------------------------------------------------------------%
    \newpage
    \section*{2. Compound Propositions}
        \begin{introNOHEAD}
            We can also create a \textbf{compound proposition} using the logical operators
            for AND $\land$, OR $\lor$, and NOT $\neg$. When we're writing out
            a compound proposition, we will usually assign each proposition a
            \textbf{propositional variable}.

            ~\\ If we have the propositions
            $p$ is the proposition ``Paul is taking discrete math",
            $q$ is the proposition, ``Paul has a calculator",
            then we can build the compound propositions like:
            \begin{itemize}
                \item[] $p \land q$: Paul is taking discrete math and Paul has a calculator
                \item[] $p \lor q$: Paul is taking discrete math or Paul has a calculator
                \item[] $\neg p$: Paul is NOT taking discrete math
                \item[] $\neg q$: Paul does NOT have a calculator
            \end{itemize}

            The result of a compound proposition depends on the value of
            each proposition it is made up of:

            \begin{enumerate}
                \item A compound proposition $a \land b \land c$ is \textbf{only true}
                    if all propositions are true; it will be \textbf{false} if
                    one or more of the propositions is false.
                \item A compound proposition $a \lor b \lor c$ is \textbf{true}
                    if one or more of the propositions is true; it is \textbf{only false}
                    if all propositions are false.
                \item A compound proposition $\neg a$ is \textbf{true} only
                    if the proposition is false; it is only \textbf{false} if
                    the proposition is true.
            \end{enumerate}

        \end{introNOHEAD}

        % - QUESTION --------------------------------------------------%
        \newpage
        \begin{question}{2}{24\%}
            Given the following compound propositions and proposition values,
            write out whether the full compound proposition is \textbf{true} or \textbf{false}.
        \end{question}

        \paragraph{a AND b:} ~\\

        \begin{tabular}{ | l  c | c | p{6cm} | }
            \hline
            & \textbf{Compound} & \textbf{Values} & \textbf{Result}
            \\ \hline

            a. &        $a \land b$ &       $a = true, b = true$ &      \iftoggle{answerkey}{ \begin{answer} TRUE \end{answer} }{}  \\ \hline
            b. &        $a \land b$ &       $a = true, b = false$ &      \iftoggle{answerkey}{ \begin{answer} FALSE \end{answer} }{}  \\ \hline
            c. &        $a \land b$ &       $a = false, b = true$ &      \iftoggle{answerkey}{ \begin{answer} FALSE \end{answer} }{}  \\ \hline
            d. &        $a \land b$ &       $a = false, b = false$ &      \iftoggle{answerkey}{ \begin{answer} FALSE \end{answer} }{}  \\ \hline
        \end{tabular}

        \paragraph{a OR b:} ~\\

        \begin{tabular}{ | l  c | c | p{6cm} | }
            \hline
            & \textbf{Compound} & \textbf{Values} & \textbf{Result}
            \\ \hline

            e. &        $a \lor b$ &       $a = true, b = true$ &     \\ \hline
            f. &        $a \lor b$ &       $a = true, b = false$ &    \\ \hline
            g. &        $a \lor b$ &       $a = false, b = true$ &    \\ \hline
            h. &        $a \lor b$ &       $a = false, b = false$ &   \\ \hline
        \end{tabular}

        \paragraph{Combinations:} ~\\

        \begin{tabular}{ | l  c | c | p{6cm} | }
            \hline
            & \textbf{Compound} & \textbf{Values} & \textbf{Result}
            \\ \hline

            i. &        $a \land \neg b$ &       $a = true, b = false$ &   \\ \hline
            j. &        $a \lor \neg b$ &        $a = false, b = true$ &   \\ \hline

            k. &        $\neg a \land b$ &       $a = false, b = true$ &   \\ \hline
            l. &        $\neg a \lor b$ &        $a = false, b = false$ &  \\ \hline
        \end{tabular}


        % - QUESTION --------------------------------------------------%
        \newpage
        \begin{question}{3}{40\%}
            For the following, ``translate" the following English statements
            into compound propositions.

            \begin{itemize}
                \item[] $p:$ ``The printer is offline"
                \item[] $q:$ ``The printer is out of paper"
                \item[] $r:$ ``The document has finished printing"
            \end{itemize}
        \end{question}

        \begin{enumerate}
            \item[a.] The printer is not out of paper.
            \item[b.] The printer is online.
            \item[c.] The printer is offline and it is out of paper
            \item[d.] The printer is online and it is not out of paper.
            \item[e.] Either the printer is online, or it is out of paper.
            \item[f.] The printer is online, but it is out of paper.
            \item[g.] The printer is offline or it is out of paper, but not both.
            \item[h.] The printer is online and the printer has paper, and the document has not finished printing.
        \end{enumerate}


    %------------------------------------------------------------------%
    \newpage
    \section*{3. Truth tables}
        \begin{introNOHEAD}
            When we're working with compound propositional statements,
            the result of the compound depends on the true/false values
            of each proposition it is built up of.
            ~\\
            We can diagram out all possible states of a compound proposition
            by using a \textbf{truth table}. In a truth table, we list
            all propositional variables first on the left, as well as
            all possible combinations of their states, and then
            the compound statement's result on the right.

            ~\\

            \begin{tabular}{ l l l }

                \begin{tabular}{ | c | c | c | c | }
                    \hline
                    $p$ & $q$ & & $p \land q$ \\ \hline
                    T & T & & T \\ \hline
                    T & F & & F \\ \hline
                    F & T & & F \\ \hline
                    F & F & & F \\ \hline

                \end{tabular}
                &

                \begin{tabular}{ | c | c | c | c | }
                    \hline
                    $p$ & $q$ & & $p \land q$ \\ \hline
                    T & T & & T \\ \hline
                    T & F & & F \\ \hline
                    F & T & & F \\ \hline
                    F & F & & F \\ \hline

                \end{tabular}
                &

                \begin{tabular}{ | c | c | c | }
                    \hline
                    $p$ & & $\neg p$ \\ \hline
                    T & & F \\ \hline
                    F & & T \\ \hline

                \end{tabular}

            \end{tabular}
        \end{introNOHEAD}

        ~\\
        % - QUESTION --------------------------------------------------%
        \begin{question}{4}{12\%}
            Complete the following truth tables. ~\\
        \end{question}

        ~\\
        a.
        \begin{tabular}{ | p{1cm} | p{1cm} | c | p{2cm} | p{3cm} | }
            \hline
            $p$ & $q$ & & $\neg q$ & $p \land \neg q$ \\ \hline
            T & T & & F & \\ \hline
            T & F & & T & \\ \hline
            F & T & &  & \\ \hline
            F & F & &  & \\ \hline
        \end{tabular}

        ~\\~\\
        b.
        \begin{tabular}{ | p{1cm} | p{1cm} | c | p{2cm} | p{2cm} | p{3cm} | }
            \hline
            $p$ & $q$ & & $\neg p$ & $\neg q$ & $\neg p \lor \neg q$ \\ \hline
            T & T & & & & \\ \hline
            T & F & & & & \\ \hline
            F & T & & & & \\ \hline
            F & F & & & & \\ \hline
        \end{tabular}

        \newpage
        \begin{introNOHEAD}
            When you're building out truth tables, there is a specific order
            you should write out the ``T" and ``F" states.
            Begin with all ``T" values first, and work your way down to
            all ``F" values first. As you go, change the right-most state
            from ``T" to ``F", working your way from right-to-left.

            ~\\
            So with two variables: \\ ``TT", ``TF", ``FT", ``FF".
        \end{introNOHEAD}


        % - QUESTION --------------------------------------------------%
        \begin{question}{5}{6\%}
            Using the rules above, write out all the states for a
            truth table with three propositional variables. ~\\
        \end{question}

        \begin{tabular}{ | p{2cm} | p{2cm} | p{2cm} | }
            \hline
            $p$ & $q$ & $r$   \\ \hline
            T & T & T         \\ \hline
            T & T & F         \\ \hline
            T &  &          \\ \hline
            T & F & F         \\ \hline
            F &  &          \\ \hline
            F &  &          \\ \hline
            F &  &          \\ \hline
            F & F & F         \\ \hline
        \end{tabular}

        \newpage

        \begin{introNOHEAD}
            Whenever the final columns of the truth tables for two propositions $p$ and $q$ are the same,
            we say that $p$ and $q$ are \textbf{logically equivalent}, and we write: $p \equiv q$.
            \footnote{From https://en.wikibooks.org/wiki/Discrete\_Mathematics/Logic}
        \end{introNOHEAD}

        % - QUESTION --------------------------------------------------%
        \begin{question}{6}{10\%}
                Use a truth table to show that the compound propositions,

                \begin{center}
                $ \neg p \land \neg q $
                \tab and \tab
                $ \neg ( p \lor q ) $
                \end{center}
                are logically equivalent. The final two columns are
                the compound propositions above.
        \end{question}
        ~\\
        \begin{tabular}{ | c | c | c | c | c | c  c | c | c | c | c | c | }
            \hline
            $p$ &
            $q$ & &

            $\neg p$ &
            $\neg q$ &

            $(p \lor q)$ & & &

            $ \neg p \land \neg q $ & &
            $ \neg ( p \lor q ) $
            \\ \hline

            T & T & & F & F & & & & & & \\ \hline
            T & F & & F & T & & & & & & \\ \hline
            F & T & & T & F & & & & & & \\ \hline
            F & F & & T & T & & & & & & \\ \hline
        \end{tabular}

    %------------------------------------------------------------------%
    %- WORKSHEET ------------------------------------------------------%
    %------------------------------------------------------------------%
    \newpage
    \begin{center} \section*{Chapter \laChapter In-class Exercise Worksheet} \end{center}

    \iftoggle{answerkey}{
      \begin{answer} \begin{center} ANSWER KEY \end{center} \end{answer}
    }{}

    \paragraph{Team:}
    Please write down all people in your team. ~\\

    % table %
    \begin{tabular}{ p{6cm} p{6cm} }
        1. & 2. \\
        3. & 4.
    \end{tabular}
    % table %

    \paragraph{Section:} ~\\
        $\Box$ MW 4:30 - 5:45 pm \tab
        $\Box$ M 6:00 - 8:50 pm \tab
        $\Box$ TR 2:00 - 3:15 pm

    \paragraph{Team rules:}

    \begin{itemize}
        \item \textbf{Only one worksheet will be turned in per team.}
            Each member of the team will receive the same score.
        \item You can collaborate on the exercise together, or you can
            work separately and then compare your answers with your team
            as you fill out the turn-in worksheet.
    \end{itemize}

    \paragraph{Work rules:}

    \begin{itemize}
        \item Fill out your answers on this answer sheet.
        \item Write cleanly and linearly - if I can't make sense of
            your solution then you won't get credit.
        \item Write out each step (within reason) - if I can't see the
            logic you followed to get to your answer, you may get points taken off.
        \item Don't scribble out cancellations - I can't read that.
            For example, if a numerator/denominator cancel out, or a +/-
            cancels out, don't scribble out the numbers - just use a single slash!
    \end{itemize}

    \paragraph{Grading:}
        The total amount of points for an in-class exercise is 5 points each.
        Each question will have a weight assigned to it, and will be given
        a a point value between 0 and 4:

    \begin{center}
        \begin{tabular}{ | l | l | }
            \hline
            0 & Nothing written \\ \hline
            1 & Something written, but incorrect \\ \hline
            2 & Partially correct, but multiple errors \\ \hline
            3 & Mostly correct, with one or two errors \\ \hline
            4 & Perfect; correct answer and notation \\ \hline

        \end{tabular}
    \end{center}

    %------------------------------------------------------------------%
    %- Exercise Begin -------------------------------------------------%
    %------------------------------------------------------------------%

    % \iftoggle{answerkey}{ \begin{answer} asdfasdf \end{answer} }{ { ~\\ \raisebox{0pt}[2cm][0pt]{  } } }
    % \iftoggle{answerkey}{ \begin{answer} TRUE \end{answer} }{}

    %------------------------------------------------------------------%
    \newpage{}
    \section*{Answer sheet}

    ~\\
    % - QUESTION --------------------------------------------------%
    \begin{answersheetquestion}{1}{Proposition or not?}{8}
        For the following, mark whether the statement is a \textbf{proposition} and,
        if it is, mark whether it is \textbf{true} or \textbf{false}.
    \end{answersheetquestion}

    \begin{enumerate}
        \item[a.] 10 + 20 = 30.
            \iftoggle{answerkey}{ \begin{answer} Proposition, true \end{answer} }{}
        \item[b.] $2 \cdot a$ is always even.
            \iftoggle{answerkey}{ \begin{answer} Proposition, true \end{answer} }{}
        \item[c.] Is $4 < 5$?
            \iftoggle{answerkey}{ \begin{answer} Not a proposition \end{answer} }{}
        \item[d.] Pineapple belongs on pizza.
            \iftoggle{answerkey}{ \begin{answer} Proposition, true \end{answer} }{}
    \end{enumerate}

    \hrulefill

    % - QUESTION --------------------------------------------------%
    \begin{answersheetquestion}{2}{Proposition or not?}{24}
        Given the following compound propositions and proposition values,
        write out whether the full compound proposition is \textbf{true} or \textbf{false}.
    \end{answersheetquestion}

    ~\\
    \begin{tabular}{ l l    l l     l l     l l }

        a. & \fitb[2cm] &
        b. & \fitb[2cm] &
        c. & \fitb[2cm] &
        d. & \fitb[2cm]

        \\
        & \iftoggle{answerkey}{ \begin{answer} TRUE \end{answer} }{}
        &
        & \iftoggle{answerkey}{ \begin{answer} FALSE \end{answer} }{}
        &
        & \iftoggle{answerkey}{ \begin{answer} FALSE \end{answer} }{}
        &
        & \iftoggle{answerkey}{ \begin{answer} FALSE \end{answer} }{}
        \\

        e. & \fitb[2cm] &
        f. & \fitb[2cm] &
        g. & \fitb[2cm] &
        h. & \fitb[2cm]

        \\
        & \iftoggle{answerkey}{ \begin{answer} TRUE \end{answer} }{}
        &
        & \iftoggle{answerkey}{ \begin{answer} TRUE \end{answer} }{}
        &
        & \iftoggle{answerkey}{ \begin{answer} TRUE \end{answer} }{}
        &
        & \iftoggle{answerkey}{ \begin{answer} FALSE \end{answer} }{}
        \\

        i. & \fitb[2cm] &
        j. & \fitb[2cm] &
        k. & \fitb[2cm] &
        l. & \fitb[2cm]

        \\
        & \iftoggle{answerkey}{ \begin{answer} TRUE \end{answer} }{}
        &
        & \iftoggle{answerkey}{ \begin{answer} FALSE \end{answer} }{}
        &
        & \iftoggle{answerkey}{ \begin{answer} TRUE \end{answer} }{} 
        &
        & \iftoggle{answerkey}{ \begin{answer} TRUE \end{answer} }{}
        \\
    \end{tabular} ~\\

    \hrulefill

    % - QUESTION --------------------------------------------------%
    \begin{answersheetquestion}{3}{Proposition or not?}{40}
        For the following, ``translate" the following English statements
        into compound propositions. \\
        $p:$ ``The printer is offline" ,
        $q:$ ``The printer is out of paper" , \\
        $r:$ ``The document has finished printing" ~\\
    \end{answersheetquestion}

    \begin{tabular}{ p{6cm}  p{6cm} }
        a. \iftoggle{answerkey}{ \begin{answer} $\neg q$ \end{answer} }{} &
        b. \iftoggle{answerkey}{ \begin{answer} $\neg p$ \end{answer} }{} \\
        c. \iftoggle{answerkey}{ \begin{answer} $p \land q$ \end{answer} }{} &
        d. \iftoggle{answerkey}{ \begin{answer} $\neg p \land \neg q$ \end{answer} }{} \\
        e. \iftoggle{answerkey}{ \begin{answer} $p \lor q$ \end{answer} }{} &
        f. \iftoggle{answerkey}{ \begin{answer} $\neg p \land q$ \end{answer} }{} \\
        g. \iftoggle{answerkey}{ \begin{answer} $(p \lor q) \land (p \land q)$ \end{answer} }{} &
        h. \iftoggle{answerkey}{ \begin{answer} $\neg p \land \neg q \land \neg r$ \end{answer} }{} \\
    \end{tabular}

    % - QUESTION --------------------------------------------------%
    \newpage
    \begin{answersheetquestion}{4}{Truth tables}{12}
    \end{answersheetquestion}

    ~\\
    a.
    \begin{tabular}{ | p{1cm} | p{1cm} | c | p{2cm} | p{3cm} | }
        \hline
        $p$ & $q$ & & $\neg q$ & $p \land \neg q$ \\ \hline
        T & T & & F     & \iftoggle{answerkey}{ \begin{answer} F \end{answer} }{}
        \\ \hline
        T & F & & T     & \iftoggle{answerkey}{ \begin{answer} T \end{answer} }{}
        \\ \hline
        F & T & & \iftoggle{answerkey}{ \begin{answer} F \end{answer} }{}      & \iftoggle{answerkey}{ \begin{answer} F \end{answer} }{}
        \\ \hline
        F & F & & \iftoggle{answerkey}{ \begin{answer} T \end{answer} }{}      & \iftoggle{answerkey}{ \begin{answer} F \end{answer} }{}
        \\ \hline
    \end{tabular}

    ~\\~\\
    b.
    \begin{tabular}{ | p{1cm} | p{1cm} | c | p{2cm} | p{2cm} | p{3cm} | }
        \hline
        $p$ & $q$ & & $\neg p$ & $\neg q$ & $\neg p \lor \neg q$ \\ \hline
        T & T &
        & \iftoggle{answerkey}{ \begin{answer} F \end{answer} }{}
        & \iftoggle{answerkey}{ \begin{answer} F \end{answer} }{}
        & \iftoggle{answerkey}{ \begin{answer} F \end{answer} }{}
        \\ \hline
        T & F &
        & \iftoggle{answerkey}{ \begin{answer} F \end{answer} }{}
        & \iftoggle{answerkey}{ \begin{answer} T \end{answer} }{}
        & \iftoggle{answerkey}{ \begin{answer} T \end{answer} }{}
        \\ \hline
        F & T &
        & \iftoggle{answerkey}{ \begin{answer} T \end{answer} }{}
        & \iftoggle{answerkey}{ \begin{answer} F \end{answer} }{}
        & \iftoggle{answerkey}{ \begin{answer} T \end{answer} }{}
        \\ \hline
        F & F &
        & \iftoggle{answerkey}{ \begin{answer} T \end{answer} }{}
        & \iftoggle{answerkey}{ \begin{answer} T \end{answer} }{}
        & \iftoggle{answerkey}{ \begin{answer} T \end{answer} }{}
        \\ \hline
    \end{tabular} ~\\

    \hrulefill

    % - QUESTION --------------------------------------------------%
    \begin{answersheetquestion}{5}{States}{6}
        Write out all the states for a
        truth table with three propositional variables. ~\\
    \end{answersheetquestion}

    \begin{tabular}{ | p{2cm} | p{2cm} | p{2cm} | }
        \hline
        $p$ & $q$ & $r$
        \\ \hline
        T & T & T
        \\ \hline
        T & T & F
        \\ \hline
        T & \iftoggle{answerkey}{ \begin{answer} F \end{answer} }{} & \iftoggle{answerkey}{ \begin{answer} T \end{answer} }{}
        \\ \hline
        T & F & F
        \\ \hline
        F & \iftoggle{answerkey}{ \begin{answer} T \end{answer} }{} & \iftoggle{answerkey}{ \begin{answer} T \end{answer} }{}
        \\ \hline
        F & \iftoggle{answerkey}{ \begin{answer} T \end{answer} }{} & \iftoggle{answerkey}{ \begin{answer} F \end{answer} }{}
        \\ \hline
        F & \iftoggle{answerkey}{ \begin{answer} F \end{answer} }{} &     \iftoggle{answerkey}{ \begin{answer} T \end{answer} }{}
        \\ \hline
        F & F & F
        \\ \hline
    \end{tabular}

    \newpage
    % - QUESTION --------------------------------------------------%
    \begin{answersheetquestion}{5}{Logically equivalent}{10}
                Use a truth table to show that the compound propositions,

                \begin{center}
                $ \neg p \land \neg q $
                \tab and \tab
                $ \neg ( p \lor q ) $
                \end{center}
                are logically equivalent. The final two columns are
                the compound propositions above.
    \end{answersheetquestion}
        ~\\ ~\\

    \begin{tabular}{ | c | c | c | c |  c | c | c | c | c | c | }
        \hline
            $p$ &
            $q$ & &

            $\neg p$ &
            $\neg q$ &

            $(p \lor q)$ & &

            $ \neg p \land \neg q $ & &
            $ \neg ( p \lor q ) $
            \\ \hline

        T & T & & F & F
        & \iftoggle{answerkey}{ \begin{answer} T \end{answer} }{} &
        & \iftoggle{answerkey}{ \begin{answer} T \end{answer} }{} &
        & \iftoggle{answerkey}{ \begin{answer} T \end{answer} }{}

        \\ \hline
        T & F & & F & T
        & \iftoggle{answerkey}{ \begin{answer} T \end{answer} }{} &
        & \iftoggle{answerkey}{ \begin{answer} F \end{answer} }{} &
        & \iftoggle{answerkey}{ \begin{answer} F \end{answer} }{}

        \\ \hline
        F & T & & T & F
        & \iftoggle{answerkey}{ \begin{answer} T \end{answer} }{} &
        & \iftoggle{answerkey}{ \begin{answer} F \end{answer} }{} &
        & \iftoggle{answerkey}{ \begin{answer} F \end{answer} }{}

        \\ \hline
        F & F & & T & T
        & \iftoggle{answerkey}{ \begin{answer} F \end{answer} }{} &
        & \iftoggle{answerkey}{ \begin{answer} F \end{answer} }{} &
        & \iftoggle{answerkey}{ \begin{answer} F \end{answer} }{}

        \\ \hline

    \end{tabular}
    ~\\

    \hrulefill

    %------------------------------------------------------------------%
    %- Grading --------------------------------------------------------%
    %------------------------------------------------------------------%
    \subsection*{Grading}

    \begin{center}

        \begin{tabular}{ | l | l | l | l | }
            \hline
            \textbf{ Question } & \textbf{ Weight } & \textbf{ 0-4 } & \textbf{ Adjusted score }
            \\ \hline{}

            1 & 8\% & &    \\ \hline
            2 & 24\% & &    \\ \hline
            3 & 40\% & &    \\ \hline
            4 & 12\% & &    \\ \hline
            5 & 6\% & &    \\ \hline
            6 & 10\% & &    \\ \hline
            & & & \\ \hline
            & & & \\ \hline


        \end{tabular}
    \end{center}

\end{document}
