\documentclass[a4paper,12pt]{book}
\usepackage[utf8]{inputenc}
\title{}
\author{Rachel Morris}
\date{\today}

\usepackage{rachwidgets}
\usepackage{fancyhdr}
\usepackage{lastpage}
\usepackage{boxedminipage}

\newcommand{\laChapter}{X.Z\ }

\pagestyle{fancy}
\fancyhf{}
\lhead{CS 210}
\chead{Fall 2017}
\rhead{Ch \laChapter Exercise}
\rfoot{\thepage\ of \pageref{LastPage}}
\lfoot{\scriptsize By Rachel Morris, last updated \today}

\renewcommand{\headrulewidth}{2pt}
\renewcommand{\footrulewidth}{1pt}

\begin{document}

    %\toggletrue{answerkey}
    \togglefalse{answerkey}

    %------------------------------------------------------------------%
    %- INSTRUCTIONS ---------------------------------------------------%
    %------------------------------------------------------------------%
    
    \chapter*{Chapter \laChapter In-class Exercise} \stepcounter{chapter}

    \iftoggle{answerkey}{
      \begin{answer} ANSWER KEY \end{answer}
    }{}
    
    \paragraph{Info:}
    In-class exercises are meant to introduce you to the new topics
    of this chapter of the book. Each part will have an introductory
    description of the content and example(s), followed by practice
    problems for you to work on. ~\\

    These assignments are \textbf{team assignments} - your team will
    turn in \textit{one} copy of the exericse. It is up to your team
    how to appropach the assignments; you can work separately and then
    check your work together, or you can collaborate on the assignment
    together. ~\\

    Work must be clean; points may be deducted if the instructor cannot
    read the work.

    \begin{center}
        \textbf{This is the instruction page, \\
        make sure to fill out your answers to turn in on the worksheet}
    \end{center}

    \hrulefill{}
    
    \newpage{}

    %------------------------------------------------------------------%
    %- Exercise Begin -------------------------------------------------%
    %------------------------------------------------------------------%

    %------------------------------------------------------------------%
    \section*{1. Cointoss}

        \begin{intro}{Cointoss}
            about
        \end{intro}

        \begin{question}{5}{10\%}
            asdfasdf
        \end{question}



    %------------------------------------------------------------------%
    %- WORKSHEET ------------------------------------------------------%
    %------------------------------------------------------------------%
    \newpage
    \begin{center} \section*{Chapter \laChapter In-class Exercise Worksheet} \end{center}

    \iftoggle{answerkey}{
      \begin{answer} \begin{center} ANSWER KEY \end{center} \end{answer}
    }{}

    \paragraph{Team:}
    Please write down all people in your team. ~\\

    % table %
    \begin{tabular}{ p{6cm} p{6cm} }
        1. & 2. \\
        3. & 4.
    \end{tabular}
    % table %
    
    \paragraph{Section:} ~\\
        $\Box$ MW 4:30 - 5:45 pm \tab
        $\Box$ M 6:00 - 8:50 pm \tab
        $\Box$ TR 2:00 - 3:15 pm

    \paragraph{Team rules:}

    \begin{itemize}
        \item \textbf{Only one worksheet will be turned in per team.}
            Each member of the team will receive the same score.
        \item You can collaborate on the exercise together, or you can
            work separately and then compare your answers with your team
            as you fill out the turn-in worksheet.
    \end{itemize}

    \paragraph{Work rules:}

    \begin{itemize}
        \item Fill out your answers on this answer sheet.
        \item Write cleanly and linearly - if I can't make sense of
            your solution then you won't get credit.
        \item Write out each step (within reason) - if I can't see the
            logic you followed to get to your answer, you may get points taken off.
        \item Don't scribble out cancellations - I can't read that.
            For example, if a numerator/denominator cancel out, or a +/-
            cancels out, don't scribble out the numbers - just use a single slash!
    \end{itemize}

    \paragraph{Grading:}
        The total amount of points for an in-class exercise is 5 points each.
        Each question will have a weight assigned to it, and will be given
        a a point value between 0 and 4:

    \begin{center}
        \begin{tabular}{ | l | l | }
            \hline{}
            0 & Nothing written \\ \hline
            1 & Something written, but incorrect \\ \hline
            2 & Partially correct, but multiple errors \\ \hline
            3 & Mostly correct, with one or two errors \\ \hline
            4 & Perfect; correct answer and notation \\ \hline
            
        \end{tabular}
    \end{center}
    
    %------------------------------------------------------------------%
    %- Exercise Begin -------------------------------------------------%
    %------------------------------------------------------------------%

    %------------------------------------------------------------------%
    \newpage{}
    \section*{Answer sheet}

    ~\\
    \begin{answersheetquestion}{1}{Do a thing}{10}
        \iftoggle{answerkey}{ \begin{answer}
            Here's the answer
        \end{answer} }{ ~\\ \raisebox{0pt}[4cm][0pt]{  } }
    \end{answersheetquestion}
        
    %------------------------------------------------------------------%
    %- Grading --------------------------------------------------------%
    %------------------------------------------------------------------%
    \hrulefill{}
    \subsection*{Grading}
    
    \begin{center}
        
        \begin{tabular}{ | l | l | l | l | }
            \hline
            \textbf{ Question } & \textbf{ Weight } & \textbf{ 0-4 } & \textbf{ Adjusted score }
            \\ \hline{}
            
            1 & 10\% & &    \\ \hline
            2 & 30\% & &    \\ \hline
            3 & 10\% & &    \\ \hline
            4 & 10\% & &    \\ \hline
            5 & 10\% & &    \\ \hline
            6 & 30\% & &    \\ \hline
            & & & \\ \hline
            & & & \\ \hline
            
            
        \end{tabular}
    \end{center}
    
\end{document}
