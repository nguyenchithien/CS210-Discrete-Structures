\documentclass[a4paper,12pt]{book}
\usepackage[utf8]{inputenc}
\title{}
\author{Rachel Morris}
\date{\today}

\usepackage{rachwidgets}
\usepackage{fancyhdr}
\usepackage{lastpage}
\usepackage{dirtree}
\usepackage{boxedminipage}

\setcounter{chapter}{3}
\setcounter{section}{0}
\newcommand{\laChapter}{3.1 Set Definitions and Operations\ }
\newcounter{question}

\newcommand{\laClass}{CS 210\ }
\newcommand{\laSemester}{Fall 2017\ }

\pagestyle{fancy}
\fancyhf{}
\lhead{\laClass \laSemester}
\chead{}
\rhead{Ch \laChapter}
\rfoot{\thepage\ of \pageref{LastPage}}
\lfoot{\scriptsize Compiled by Rachel Morris, last updated \today}

\renewcommand{\headrulewidth}{2pt}
\renewcommand{\footrulewidth}{1pt}

\begin{document}

    %\toggletrue{answerkey}
    \togglefalse{answerkey}

    %------------------------------------------------------------------%
    %- Exercise Begin -------------------------------------------------%
    %------------------------------------------------------------------%

    \section{Set Definitions and Operations}

    %------------------------------------------------------------------%
    \subsection{Common Sets}

        \begin{intro}{Common sets we will see in this chapter:}
            \begin{tabular}{l l}
                $\mathbb{N}$, the set of natural numbers &
                These numbers are ``counting
                \\ & numbers".
                This set contains 0 and
                \\ & positive integers.

                \\
                $\mathbb{Z}$, the set of integers &
                This set contains all integers:
                \\ & positive, negative, and zero.
                \\
                $\mathbb{Q}$, the set of rational numbers &
                This set contains all numbers that can
                \\ &  be characterized
                as ratios, such as $\frac{1}{2}$,
                \\ & $\frac{-17}{4}$,
                or even $\frac{3}{1}$.
                \\
                $\mathbb{R}$, the set of all real numbers &
                These can be thought of as decimal
                \\ & numbers with possibly
                unending
                \\ & strings of digits after
                the decimal point.
            \end{tabular}
        \end{intro}

        % - QUESTION --------------------------------------------------%
        \stepcounter{question}
        \begin{questionNOGRADE}{\thequestion}

            For the following numbers, which set(s) do they belong to?

            \begin{center}
                \begin{tabular}{| c | c | c | c | c |}
                    \hline
                    & $\mathbb{N}$ & $\mathbb{Z}$ & $\mathbb{Q}$ & $\mathbb{R}$
                    \\ \hline
                    10 & & & &
                    \\ \hline
                    -5 & & & &
                    \\ \hline
                    $12/6$ & & & &
                    \\ \hline
                    $\pi$ & & & &
                    \\ \hline
                    2.40 & & & &
                    \\ \hline
                \end{tabular}
            \end{center}

        \end{questionNOGRADE}


        % - QUESTION --------------------------------------------------%
        \stepcounter{question}
        \begin{questionNOGRADE}{\thequestion}
            Give examples for each of the following types of sets:

            \begin{enumerate}
                \item[a.]   List three numbers that are in the
                    set of all integers, $\mathbb{Z}$,
                    but are NOT in the set of natural numbers,
                    $\mathbb{N}$.
                    \solution{}{ ~\\ }

                \item[b.]   List three numbers that are in the
                    set of rational numbers, $\mathbb{Q}$,
                    but are NOT in the set of integers,
                    $\mathbb{Z}$.
                    \solution{}{ ~\\ }

                \item[c.]   List three numbers that are in the
                    set of all real numbers $\mathbb{R}$,
                    but are NOT in the set of rational numbers,
                    $\mathbb{Q}$.
                    \solution{}{ ~\\ }
            \end{enumerate}
        \end{questionNOGRADE}

        \newpage

        \begin{intro}{Writing out sets}
            When we are building a discrete (finite) set,
            we usually give the set a capital letter as its
            identifier. Then, the elements of the set are written
            within curly-braces, like this:

            $$ A = \{ 2, 4, 6, 8 \} $$

            The elements here are 2, 4, 6, and 8.
            The index of the element 2 is 1 - it is
            at position 1 of the set - so $A_1 = 2$.
        \end{intro}


        % - QUESTION --------------------------------------------------%
        \stepcounter{question}
        \begin{questionNOGRADE}{\thequestion}

            Create sets that meet the following criteria.
            Give the sets any letter identifier that you want.

            \begin{enumerate}
                \item[a.]   All elements of the set are odd integers.
                    \solution{}{ ~\\~\\ }

                \item[b.]   All elements of the set are fractions
                    such that, when divided, they result in
                    an infinite string of numbers to the right
                    of the decimal place (e.g., $3.333333\bar{3}$...)
                    \solution{}{ ~\\~\\ }

                \item[c.]   Create two sets of integers, where
                    the two sets have exactly two elements in common.
                    \solution{}{ ~\\~\\ }

                \item[d.]   Create two sets of natural numbers,
                    where the two sets have NO elements in common.
                    \solution{}{ ~\\~\\ }

                \item[e.]   Create a set that is empty.
                    \solution{}{ ~\\~\\ }
            \end{enumerate}

        \end{questionNOGRADE}

        \newpage

        \subsection{Subsets}

        \begin{intro}{Subsets and existence within sets:}
            ~\\
            \begin{tabular}{l l}
                $x$ exists in $A$ &
                    The notation $x \in A$ means ``$x$ is an element of $A$"
                    \\ &
                    which means that $x$ is one of the member elements
                    \\ & of $A$.
                \\ \\
                $A$ is a subset of $B$ &
                    $A$ is a subset of $B$ (written as $A \subseteq B$) if
                    \\ &
                    every element in $A$ is also an element in $B$.
                    \\ &
                    Formally, this means that for every $x$, if $x \in A$,
                    \\ & then $x \in B$.
                \\ \\
                $A$ is equal to $B$ &
                    $A$ is equal to $B$ (written $A = B$) means that
                    \\ &
                    $A$ and $B$ have exactly the same members. This is
                    \\ &
                    expressed formally by saying, $A \subseteq B$ and $B \subseteq A$.
                \\ \\
                An Empty set &
                    A set that contains no elements is called an empty
                    \\ & set,
                    and it is denoted by $\{ \}$ or $\emptyset$.
                \\ \\
                The Universal set &
                    For any given discussion, all the sets will be subsets
                    \\ & of a larger set called the universal set (or universe)
                    \\ &We commonly use the letter $U$ to denote this set.
            \end{tabular}
        \end{intro}


        \newpage
        % - QUESTION --------------------------------------------------%
        \stepcounter{question}
        \begin{questionNOGRADE}{\thequestion}

            Given theset sets:

            \begin{tabular}{l l l}
                $U = \{1, 2, 3, 4, 5, 6\}$ &
                $A = \{1, 1, 2, 2, 2, 4, 4\}$ &
                $B = \{2, 2\}$ \\
                $C = \{1, 2, 4, 5, 6\}$ &
                $D = \{6, 5, 4, 2, 1\}$ &
                $E = \{1, 4\}$
            \end{tabular}

            \begin{enumerate}
                \item[a.] Which of these statements are true? Mark with a $\checkmark$

                \begin{tabular}{l l l}
                    a. $A \subseteq B$      \solution{}{}
                \end{tabular}

                \item[b.] Fill in the blanks with either
                    $\subseteq$ (is a subset of), or
                    $\not\subseteq$ (is not a subset of), or
                    $=$ (is equal to) for the following:
            \end{enumerate}

        \end{questionNOGRADE}


        % - QUESTION --------------------------------------------------%
        \stepcounter{question}
        \begin{questionNOGRADE}{\thequestion}


        \end{questionNOGRADE}


        % - QUESTION --------------------------------------------------%
        \stepcounter{question}
        \begin{questionNOGRADE}{\thequestion}


        \end{questionNOGRADE}


\end{document}








