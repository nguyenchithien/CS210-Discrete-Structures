\documentclass[a4paper,12pt]{book}
\usepackage[utf8]{inputenc}
\title{}
\author{Rachel Morris}
\date{\today}

\usepackage{rachwidgets}
\usepackage{fancyhdr}
\usepackage{lastpage}
\usepackage{dirtree}
\usepackage{boxedminipage}

\newcommand{\laChapter}{2.1\ }
\newcommand{\laClass}{CS 210\ }
\newcommand{\laSemester}{Fall 2017\ }
\newcounter{question}

\pagestyle{fancy}
\fancyhf{}
\lhead{CS 210}
\chead{Fall 2017}
\rhead{Ch \laChapter Exercise}
\rfoot{\thepage\ of \pageref{LastPage}}
\lfoot{\scriptsize By Rachel Morris, last updated \today}

\renewcommand{\headrulewidth}{2pt}
\renewcommand{\footrulewidth}{1pt}

\begin{document}

    \toggletrue{answerkey}
    %\togglefalse{answerkey}
    
    %------------------------------------------------------------------%
    %- Exercise Begin -------------------------------------------------%
    %------------------------------------------------------------------%

    %------------------------------------------------------------------%
    \section*{1. Proving statements}

        \begin{intro}{Implications} ~\\
            This time we're exploring mathematical writing and getting introduced
            to proofs. This means that we are going to be working with
            \textbf{contrapositives} and \textbf{implications} some more
            in order to prove statements.
            To work with a statement, we turn it into an implication that
            we can work with mathematically.
            
            \paragraph{Example:}
            For every positive even integer $n$, $n+1$ is odd.
            
            Changing to an ``if, then" statement, we can form:
            
            If a positive integer $n$ is even, then $n+1$ is odd.
        \end{intro}

        % - QUESTION --------------------------------------------------%
        \stepcounter{question}
        \begin{questionNOGRADE}{\thequestion}
            Rewrite the following statements as ``if, then" statements:
            
            \begin{enumerate}
                \item[a.] When a positive integer $n$ is odd, then $n+1$ is even.
                
                    \iftoggle{answerkey}{ \begin{answer} if $n$ is odd, then $n+1$ is even. \end{answer} }{ { ~\\ \raisebox{0pt}[1cm][0pt]{  } } }
                
                \item[b.] All squares have four equal sides.
                    \begin{hint}{Hint}
                        Think of representing the square as a variable,
                        and the length of a side as a variable.
                    \end{hint}
                
                    \iftoggle{answerkey}{ \begin{answer} If $s$ is a square, then the length of every side is $l$. \end{answer} }{ { ~\\ \raisebox{0pt}[1cm][0pt]{  } } }
                
                \item[c.] asdf
            \end{enumerate}
        \end{questionNOGRADE}






    %- Team Info ------------------------------------------------------%

    \newpage
    
    \paragraph{Team:}
    Please write down all people in your team. ~\\

    % table %
    \begin{tabular}{ p{6cm} p{6cm} }
        1. & 2. \\
        3. & 4.
    \end{tabular}
    % table %
    ~\\

    \hrulefill
    \subsection*{Grading}
            
    \begin{center}
        
        \begin{tabular}{ | l | l | l | l | }
            \hline
            \textbf{ Question } & \textbf{ Weight } & \textbf{ 0-4 } & \textbf{ Adjusted score }
            \\ \hline
            
            1 & 5\% & &    \\ \hline
            
            2 & 6\% & &    \\ \hline
            
            3 & 12\% & &    \\ \hline
            
            4 & 15\% & &    \\ \hline
            
            5 & 25\% & &    \\ \hline

            & & & \\ \hline
            
            
            
        \end{tabular}
    \end{center}

\end{document}
