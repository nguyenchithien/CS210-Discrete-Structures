\documentclass[a4paper,12pt]{book}
\usepackage[utf8]{inputenc}
\title{}
\author{Rachel Morris}
\date{\today}

\usepackage{rachwidgets}
\usepackage{fancyhdr}
\usepackage{lastpage}
\usepackage{boxedminipage}

\newcommand{\laChapter}{1.2\ }

\pagestyle{fancy}
\fancyhf{}
\lhead{CS 210}
\chead{Fall 2017}
\rhead{Ch \laChapter Exercise}
\rfoot{\thepage\ of \pageref{LastPage}}
\lfoot{\scriptsize By Rachel Morris, last updated \today}

\renewcommand{\headrulewidth}{2pt}
\renewcommand{\footrulewidth}{1pt}

\begin{document}

    %\toggletrue{answerkey}
    \togglefalse{answerkey}

    %------------------------------------------------------------------%
    %- INSTRUCTIONS ---------------------------------------------------%
    %------------------------------------------------------------------%
    
    \chapter*{Chapter \laChapter In-class Exercise} \stepcounter{chapter}

    \iftoggle{answerkey}{
      \begin{answer} ANSWER KEY \end{answer}
    }{}
    
    \paragraph{Info:}
    In-class exercises are meant to introduce you to the new topics
    of this chapter of the book. Each part will have an introductory
    description of the content and example(s), followed by practice
    problems for you to work on. ~\\

    These assignments are \textbf{team assignments} - your team will
    turn in \textit{one} copy of the exericse. It is up to your team
    how to appropach the assignments; you can work separately and then
    check your work together, or you can collaborate on the assignment
    together. ~\\

    Work must be clean; points may be deducted if the instructor cannot
    read the work.

    \hrulefill{}
    
    \newpage{}

    %------------------------------------------------------------------%
    %- Exercise Begin -------------------------------------------------%
    %------------------------------------------------------------------%
    \begin{center} \section*{Chapter \laChapter In-class Exercise Instructions} \end{center}

    \begin{center}
        \textbf{This is the instruction page, 
        make sure to fill out your answers in the worksheet later in the document}
    \end{center}

    %------------------------------------------------------------------%
    \section*{1. Number sequences}

        \begin{introNOHEAD}
            For this section, we are analyzing sequences of numbers in order to build
            \textit{closed} and/or \textit{recursive} formulas to describe them.

            It can be a bit challenging at first to figure out the equation based
            on a list of numbers, so make sure to take note of some techniques for analyzing
            these sequences!

            Let's start off simple...
        \end{introNOHEAD}

        % - QUESTION --------------------------------------------------%
        \begin{question}{1}{4\%}
            For the given sequence of numbers: \tab 2, 4, 6, 8, 10
        \end{question}

        \begin{enumerate}
            \item[a.] What is the next number in the sequence?  If you can tell just by looking at it, how can you tell?

            \item[b.] If we assign numbers to each of these...

            \begin{center}
                \begin{tabular}{ | c | c | c | c | c | }
                    \hline{}
                    Item 1 & Item 2 & Item 3 & Item 4 & Item 5 \\ \hline
                    2 & 4 & 6 & 8 & 10 \\ \hline
                \end{tabular}
            \end{center}

            ...How can we come up with some formula to associate the item \# to the value?

            \item[c.] If we're describing \textbf{Item 2} in terms of \textbf{Item 1} as
                a math sequence, we can say that... \\
                Item 2 = Item 1 + ?

            \item[d.] If we're describing \textbf{Item 3} in terms of \textbf{Item 2} as a math
                equation, we can say that... \\
                Item 3 = ?

            \item[e.] If we want to generalize this and describe any item $n$ in terms of
                the previous item, $n-1$, we can say that... \\
                Item $n$ = ?
        \end{enumerate}

    %------------------------------------------------------------------%
    \section*{2. Sequences}
        \begin{introNOHEAD}{}
            \paragraph{Definition: Recursive formula} (aka recurrence relation) \\
            In mathematics, a recurrence relation is an equation that recursively defines
            a sequence [...] of values, once one or more initial terms are given:
            each further term of the sequence [...] is defined as a function
            of the preceding terms.
            \footnote{From https://en.wikipedia.org/wiki/Recurrence\_relation}

            \paragraph{Definition: Closed formula} ~\\
            A closed formula for a sequence is a formula where each term is described
            only in relation to its position in the list.
            \footnote{From Discrete Mathematics Mathematical Reasoning and Proof with Puzzles, Patterns, and Games by Douglas E Ensley}

            \paragraph{Definition: Sequence notation}
            Sequence notation is where we have some sequence, $a$,
            and $a_{n}$ denotes the element at position $n$. On a computer,
            the subscript may be written as \texttt{a[n]}.
        \end{introNOHEAD}

        

        % - QUESTION --------------------------------------------------%
        \begin{question}{2}{12\%}
            Write out the first 5 elements of the following equations: \\
            \tab a. The closed formula $a_{n} = n+1$ \\
            \tab b. The closed formula $a_{n} = 2n+1$ \\
            \tab c. The recursive formula $a_{1} = 1, a_{n} = a_{n-1} + 2$ \\
            \tab d. The resursive formula $a_{1} = 2, a_{n} = 2 a_{n-1} + 1$
        \end{question}

        \begin{hint}{Tips for finding equations}
            If it isn't immediately obvious what a sequence's function is, here are a few tips:
            
            \begin{itemize}
                \item Write out each element with its position, like
                $a_{1} = 2$, $a_{2} = 5$, $a_{3} = 10$, etc. This helps
                with trying to find a pattern between the \textbf{index} (position)
                and the \textbf{element} (value).
                
                \item Compare the difference between each element, like
                $5 - 2 = 3$, $10 - 5 = 5$, $17 - 10 = 7$. Can you find a pattern
                in the difference between the elements?
                
                \item Compare the difference between the \textit{differences}.
                Above, we can see that the difference \textit{increases}
                by 2 between each element.
            \end{itemize}
        \end{hint}

        \newpage{}
        
        % - QUESTION --------------------------------------------------%
        \begin{question}{3}{12\%}
            Figure out the \textbf{closed formula} for the following sequences. \\
            For these sequences, \textit{n} will not be multiplied
            by anything, but will have something added to it. 

            \begin{itemize}
                \item[a.]  3, 4, 5, 6, 7
                \item[b.]  6, 7, 8, 9, 10
            \end{itemize}
        \end{question}

        % - QUESTION --------------------------------------------------%
        \begin{question}{4}{12\%}
            Figure out the \textbf{closed formula} for the following sequences. \\
            For these sequences, \textit{n} will have something multiplied to it.

            \begin{itemize}
                \item[a.] 2, 4, 6, 8, 10
                \item[b.] 3, 6, 9, 12, 15
                \item[c.] 5, 10, 15, 20, 25
                \item[d.] 1, 4, 9, 16, 25
            \end{itemize}
        \end{question}
        
        % - QUESTION --------------------------------------------------%
        \begin{question}{5}{12\%}
            Figure out the \textbf{closed formula} for the following sequences. \\
            For these sequences, \textit{n} will have something multipled to it
            and added to (or subtracted from) the product.

            \begin{itemize}
                \item[a.] 1, 3, 5, 7, 9
                \item[b.] 4, 7, 10, 13, 16
                \item[c.] 7, 12, 17, 22, 27
                \item[d.] 2, 5, 10, 17, 26
            \end{itemize}
        \end{question}

        \newpage
        % - QUESTION --------------------------------------------------%
        \begin{question}{6}{12\%}
            Figure out the \textbf{recursive formula} for the following sequences. \\
            For these sequences, $a_{n-1}$ will not be multiplied by anything,
            but will have something added to it.
            Be sure to specify $a_{1}$ first. It will be the first number in the sequence.

            \begin{itemize}
                \item[a.] 1, 3, 5, 7, 9     \tab \textit{($a_{1} = 1$)}
                \item[b.] 1, 5, 9, 13, 17
                \item[c.] 2, 4, 6, 8, 10    \tab \textit{($a_{1} = 2$)}
                \item[d.] 2, 6, 10, 14, 18
            \end{itemize}
        \end{question}

        % - QUESTION --------------------------------------------------%
        \begin{question}{7}{12\%}
            Figure out the \textbf{recursive formula} for the following sequences. \\
            For these sequences, $a_{n-1}$ will have something multiplied to
            it, but nothing added to it.
            Be sure to specify $a_{1}$ first.

            \begin{itemize}
                \item[a.] 2, 4, 8, 16, 32
                \item[b.] 1, 3, 9, 27, 81
                \item[c.] 3, 6, 12, 24, 48
                \item[d.] 2, 4, 16, 256, 65536
            \end{itemize}
        \end{question}
        
        ~\\
        % - QUESTION --------------------------------------------------%
        \begin{question}{8}{12\%}
            Figure out the \textbf{recursive formula} for the following sequences. \\
            For these sequences, $a_{n-1}$ will have something multiplied
            to it and added to it.
            Be sure to specify $a_{1}$ first.

            \begin{itemize}
                \item[a.] 1, 3, 7, 15, 31
                \item[b.] 2, 5, 11, 23, 47
                \item[c.] 1, 5, 17, 53, 161
                \item[d.] 1, 4, 10, 22, 46
            \end{itemize}
        \end{question}

    \newpage
    %------------------------------------------------------------------%
    \section*{3. Summations}
        \begin{introNOHEAD}
            For a sequence of numbers (denoted $a_{k}$, where $k >= 1$,
            we can use the notation
            $$\sum_{k=1}^{n} a_{k}$$
            to denote the sum of the first $n$ terms of the sequence.
            This is called \textit{sigma notation}.

            \paragraph{Example:} Evaluate the sum $\sum_{k=1}^{3}(2k-1)$. ~\\
            First, we need to find the elements at $k=1$, $k=2$, and $k=3$:

            \begin{center}
                \begin{tabular}{| c | c | c |}
                    \hline
                    \textbf{ $k=1$ } & \textbf{ $k=2$ } & \textbf{ $k=3$ } \\
                    \hline
                    $a_{1} = (2 \cdot 1 - 1) = 1$ &
                    $a_{2} = (2 \cdot 2 - 1) = 3$ &
                    $a_{3} = (2 \cdot 3 - 1) = 5$
                    \\
                    \hline
                \end{tabular}
            \end{center}
            ~\\
            Then, we can add the values: \\
            $\sum_{k=1}^{3}(2k-1) $ \tab
            $= a_{1} + a_{2} + a_{3}$ \tab
            $= 1 + 3 + 5 $ \tab
            \fbox{ $= 9$ }
        \end{introNOHEAD}
        
        % - QUESTION --------------------------------------------------%
        \begin{question}{9}{12\%}
            Evaluate the following summations.

            \begin{itemize}
                \item[a.] $$\sum_{k=1}^{4}(3k) $$
                \item[b.] $$\sum_{k=1}^{5}(4) $$
            \end{itemize}
        \end{question}

        
    %------------------------------------------------------------------%
    %- WORKSHEET ------------------------------------------------------%
    %------------------------------------------------------------------%
    \newpage
    \begin{center} \section*{Chapter \laChapter In-class Exercise Worksheet} \end{center}

    \iftoggle{answerkey}{
      \begin{answer} \begin{center} ANSWER KEY \end{center} \end{answer}
    }{}

    \paragraph{Team:}
    Please write down all people in your team. ~\\

    % table %
    \begin{tabular}{ p{6cm} p{6cm} }
        1. & 2. \\
        3. & 4.
    \end{tabular}
    % table %
    
    \paragraph{Section:} ~\\
        $\Box$ MW 4:30 - 5:45 pm \tab
        $\Box$ M 6:00 - 8:50 pm \tab
        $\Box$ TR 2:00 - 3:15 pm

    \paragraph{Team rules:}

    \begin{itemize}
        \item \textbf{Only one worksheet will be turned in per team.}
            Each member of the team will receive the same score.
        \item You can collaborate on the exercise together, or you can
            work separately and then compare your answers with your team
            as you fill out the turn-in worksheet.
    \end{itemize}

    \paragraph{Work rules:}

    \begin{itemize}
        \item Fill out your answers on this answer sheet.
        \item Write cleanly and linearly - if I can't make sense of
            your solution then you won't get credit.
        \item Write out each step (within reason) - if I can't see the
            logic you followed to get to your answer, you may get points taken off.
        \item Don't scribble out cancellations - I can't read that.
            For example, if a numerator/denominator cancel out, or a +/-
            cancels out, don't scribble out the numbers - just use a single slash!
    \end{itemize}

    \paragraph{Grading:}
        The total amount of points for an in-class exercise is 5 points each.
        Each question will have a weight assigned to it, and will be given
        a a point value between 0 and 4:

    \begin{center}
        \begin{tabular}{ | l | l | }
            \hline
            0 & Nothing written \\ \hline
            1 & Something written, but incorrect \\ \hline
            2 & Partially correct, but multiple errors \\ \hline
            3 & Mostly correct, with one or two errors \\ \hline
            4 & Perfect; correct answer and notation \\ \hline
            
        \end{tabular}
    \end{center}
    
    %------------------------------------------------------------------%
    %- Exercise Begin -------------------------------------------------%
    %------------------------------------------------------------------%

    %------------------------------------------------------------------%
    \newpage{}
    \section*{Answer sheet}

    ~\\
    % - QUESTION --------------------------------------------------%
    \begin{answersheetquestion}{1}{Sequence of numbers}{4}

        \begin{enumerate}
            \item[a.] What is the next number in the sequence?  If you can tell just by looking at it, how can you tell?
                \iftoggle{answerkey}{ \begin{answer} 12 \end{answer} }{}

            \item[b.] What is a formula to associate the index and the value?

            \begin{center}
                \begin{tabular}{ | c | c | c | c | c | }
                    \hline{}
                    Item 1 & Item 2 & Item 3 & Item 4 & Item 5 \\ \hline
                    2 & 4 & 6 & 8 & 10 \\ \hline
                \end{tabular}
            \end{center}
                \iftoggle{answerkey}{ \begin{answer} Item $n$ is $2 \times n$. \end{answer} }{}

            \item[c.] Item 2 = Item 1 + \iftoggle{answerkey}{ \begin{answer} Item 2 = Item 1 + 2 \end{answer} }{}

            \item[d.] Item 3 = \iftoggle{answerkey}{ \begin{answer} Item 3 = Item 2 + 2 \end{answer} }{}

            \item[e.] Item $n$ = \iftoggle{answerkey}{ \begin{answer} Item $n$ = Item $n-1$ + 2 \end{answer} }{}
        \end{enumerate}
    \end{answersheetquestion}

    \hrulefill{}
    
    % - QUESTION --------------------------------------------------%
    \begin{answersheetquestion}{2}{First 5 elements}{12}

    \begin{figure}[h]
        \centering
        \begin{subfigure}{.5\textwidth}
            \centering
            
            \begin{itemize}
                \item[a.] $a_{n} = n+1$ \\
                    $a_{1} = $ \iftoggle{answerkey}{ \begin{answer} 2 \end{answer} }{} \\
                    $a_{2} = $ \iftoggle{answerkey}{ \begin{answer} 3 \end{answer} }{} \\
                    $a_{3} = $ \iftoggle{answerkey}{ \begin{answer} 4 \end{answer} }{} \\
                    $a_{4} = $ \iftoggle{answerkey}{ \begin{answer} 5 \end{answer} }{} \\
                    $a_{5} = $ \iftoggle{answerkey}{ \begin{answer} 6 \end{answer} }{} \\
   
                \item[c.] $a_{1} = 1, a_{n} = a_{n-1} + 2$ \\
                    $a_{1} = $ \iftoggle{answerkey}{ \begin{answer} 1 \end{answer} }{} \\
                    $a_{2} = $ \iftoggle{answerkey}{ \begin{answer} 3 \end{answer} }{} \\
                    $a_{3} = $ \iftoggle{answerkey}{ \begin{answer} 5 \end{answer} }{} \\
                    $a_{4} = $ \iftoggle{answerkey}{ \begin{answer} 7 \end{answer} }{} \\
                    $a_{5} = $ \iftoggle{answerkey}{ \begin{answer} 9 \end{answer} }{} \\
            \end{itemize}
            
        \end{subfigure}%
        \begin{subfigure}{.5\textwidth}
            \centering
            
            \begin{itemize}
                \item[b.] $a_{n} = 2n+1$ \\
                        $a_{1} = $ \iftoggle{answerkey}{ \begin{answer} 3 \end{answer} }{} \\
                        $a_{2} = $ \iftoggle{answerkey}{ \begin{answer} 5 \end{answer} }{} \\
                        $a_{3} = $ \iftoggle{answerkey}{ \begin{answer} 7 \end{answer} }{} \\
                        $a_{4} = $ \iftoggle{answerkey}{ \begin{answer} 9 \end{answer} }{} \\
                        $a_{5} = $ \iftoggle{answerkey}{ \begin{answer} 11 \end{answer} }{} \\
 
                \item[d.] $a_{1} = 2, a_{n} = 2 a_{n-1} + 1$ \\
                        $a_{1} = $ \iftoggle{answerkey}{ \begin{answer} 2 \end{answer} }{} \\
                        $a_{2} = $ \iftoggle{answerkey}{ \begin{answer} 5 \end{answer} }{} \\
                        $a_{3} = $ \iftoggle{answerkey}{ \begin{answer} 11 \end{answer} }{} \\
                        $a_{4} = $ \iftoggle{answerkey}{ \begin{answer} 23 \end{answer} }{} \\
                        $a_{5} = $ \iftoggle{answerkey}{ \begin{answer} 47 \end{answer} }{} \\
            \end{itemize}
        \end{subfigure}
        %\caption{A figure with two subfigures}
        %\label{fig:test}
        \end{figure}
    \end{answersheetquestion}

    ~\\
    
    \hrulefill
    
    \newpage
    ~\\
    % - QUESTION --------------------------------------------------%
    \begin{answersheetquestion}{3a}{Find closed formula}{6}
        Find closed formula for sequence: 3, 4, 5, 6, 7
        \iftoggle{answerkey}{ \begin{answer} $a_{n} = n+2$ \end{answer} }{ { ~\\ \raisebox{0pt}[2cm][0pt]{  } } }
    \end{answersheetquestion}

    ~\\
    
    \hrulefill
    
    ~\\
    % - QUESTION --------------------------------------------------%
    \begin{answersheetquestion}{3b}{Find closed formula}{6}
        Find closed formula for sequence: 6, 7, 8, 9, 10
        \iftoggle{answerkey}{ \begin{answer} $a_{n} = n+5$ \end{answer} }{ { ~\\ \raisebox{0pt}[2cm][0pt]{  } } }
    \end{answersheetquestion}

    ~\\
    
    \hrulefill
    
    ~\\
    % - QUESTION --------------------------------------------------%
    \begin{answersheetquestion}{4a}{Find closed formula}{3}
        Find closed formula for sequence: 2, 4, 6, 8, 10
        \iftoggle{answerkey}{ \begin{answer} $a_{n} = 2n$ \end{answer} }{ { ~\\ \raisebox{0pt}[2cm][0pt]{  } } }
    \end{answersheetquestion}

    ~\\
    
    \hrulefill
    
    ~\\
    % - QUESTION --------------------------------------------------%
    \begin{answersheetquestion}{4b}{Find closed formula}{3}
        Find closed formula for sequence: 3, 6, 9, 12, 15
        \iftoggle{answerkey}{ \begin{answer} $a_{n} = 3n$ \end{answer} }{ { ~\\ \raisebox{0pt}[2cm][0pt]{  } } }
    \end{answersheetquestion}

    ~\\
    
    \hrulefill
    
    ~\\
    % - QUESTION --------------------------------------------------%
    \begin{answersheetquestion}{4c}{Find closed formula}{3}
        Find closed formula for sequence: 5, 10, 15, 20, 25
        \iftoggle{answerkey}{ \begin{answer} $a_{n} = 5n$ \end{answer} }{ { ~\\ \raisebox{0pt}[2cm][0pt]{  } } }
    \end{answersheetquestion}

    ~\\
    
    \hrulefill
    
    ~\\
    % - QUESTION --------------------------------------------------%
    \begin{answersheetquestion}{4d}{Find closed formula}{3}
        Find closed formula for sequence: 1, 4, 9, 16, 25
        \iftoggle{answerkey}{ \begin{answer} $a_{n} = n^{2}$ \end{answer} }{ { ~\\ \raisebox{0pt}[2cm][0pt]{  } } }
    \end{answersheetquestion}

    ~\\
    
    \hrulefill
    
    ~\\
    % - QUESTION --------------------------------------------------%
    \begin{answersheetquestion}{5a}{Find closed formula}{3}
        Find closed formula for sequence: 1, 3, 5, 7, 9
        \iftoggle{answerkey}{ \begin{answer} $a_{n} = 2n-1$ \end{answer} }{ { ~\\ \raisebox{0pt}[2cm][0pt]{  } } }
    \end{answersheetquestion}
    
    ~\\
    
    \hrulefill
    
    ~\\
    % - QUESTION --------------------------------------------------%
    \begin{answersheetquestion}{5b}{Find closed formula}{3}
        Find closed formula for sequence: 4, 7, 10, 13, 16
        \iftoggle{answerkey}{ \begin{answer} $a_{n} = 3n+1$ \end{answer} }{ { ~\\ \raisebox{0pt}[2cm][0pt]{  } } }
    \end{answersheetquestion}
    
    ~\\
    
    \hrulefill
    
    ~\\
    % - QUESTION --------------------------------------------------%
    \begin{answersheetquestion}{5c}{Find closed formula}{3}
        Find closed formula for sequence: 7, 12, 17, 22, 27
        \iftoggle{answerkey}{ \begin{answer} $a_{n} = 5n+2$ \end{answer} }{ { ~\\ \raisebox{0pt}[2cm][0pt]{  } } }
    \end{answersheetquestion}
    
    ~\\
    
    \hrulefill
    
    ~\\
    % - QUESTION --------------------------------------------------%
    \begin{answersheetquestion}{5d}{Find closed formula}{3}
        Find closed formula for sequence: 2, 5, 10, 17, 26
        \iftoggle{answerkey}{ \begin{answer} $a_{n} = n^{2} + 1$ \end{answer} }{ { ~\\ \raisebox{0pt}[2cm][0pt]{  } } }
    \end{answersheetquestion}

    ~\\
    
    \hrulefill
    
    ~\\
    % - QUESTION --------------------------------------------------%
    \begin{answersheetquestion}{6a}{Find recursive formula}{3}
        Find recursive formula for sequence: 1, 3, 5, 7, 9
        \iftoggle{answerkey}{ \begin{answer} $a_{1} = 1$; $a_{n} = a_{n-1} + 2$ \end{answer} }{ { ~\\ \raisebox{0pt}[2cm][0pt]{  } } }
    \end{answersheetquestion}

    ~\\
    
    \hrulefill
    
    ~\\
    % - QUESTION --------------------------------------------------%
    \begin{answersheetquestion}{6b}{Find recursive formula}{3}
        Find recursive formula for sequence: 1, 5, 9, 13, 17
        \iftoggle{answerkey}{ \begin{answer} $a_{1} = 1$; $a_{n} = a_{n-1} + 4$ \end{answer} }{ { ~\\ \raisebox{0pt}[2cm][0pt]{  } } }
    \end{answersheetquestion}

    ~\\
    
    \hrulefill
    
    ~\\
    % - QUESTION --------------------------------------------------%
    \begin{answersheetquestion}{6c}{Find recursive formula}{3}
        Find recursive formula for sequence: 2, 4, 6, 8, 10
        \iftoggle{answerkey}{ \begin{answer} $a_{1} = 2$; $a_{n} = a_{n-1} + 2$ \end{answer} }{ { ~\\ \raisebox{0pt}[2cm][0pt]{  } } }
    \end{answersheetquestion}

    ~\\
    
    \hrulefill
    
    ~\\
    % - QUESTION --------------------------------------------------%
    \begin{answersheetquestion}{6d}{Find recursive formula}{3}
        Find recursive formula for sequence: 2, 6, 10, 14, 18
        \iftoggle{answerkey}{ \begin{answer} $a_{1} = 2$; $a_{n} = a_{n-1} + 4$ \end{answer} }{ { ~\\ \raisebox{0pt}[2cm][0pt]{  } } }
    \end{answersheetquestion}

    ~\\
    
    \hrulefill
    
    ~\\
    % - QUESTION --------------------------------------------------%
    \begin{answersheetquestion}{7a}{Find recursive formula}{3}
        Find recursive formula for sequence: 2, 4, 8, 16, 32
        \iftoggle{answerkey}{ \begin{answer} $a_{1} = 2$; $a_{n} = 2 \cdot a_{n-1}$ \end{answer} }{ { ~\\ \raisebox{0pt}[2cm][0pt]{  } } }
    \end{answersheetquestion}

    ~\\
    
    \hrulefill
    
    ~\\
    % - QUESTION --------------------------------------------------%
    \begin{answersheetquestion}{7b}{Find recursive formula}{3}
        Find recursive formula for sequence: 1, 3, 9, 27, 81
        \iftoggle{answerkey}{ \begin{answer} $a_{1} = 1$; $a_{n} = 3 \cdot a_{n-1}$ \end{answer} }{ { ~\\ \raisebox{0pt}[2cm][0pt]{  } } }
    \end{answersheetquestion}

    ~\\
    
    \hrulefill
    
    ~\\
    % - QUESTION --------------------------------------------------%
    \begin{answersheetquestion}{7c}{Find recursive formula}{3}
        Find recursive formula for sequence: 3, 6, 12, 24, 48
        \iftoggle{answerkey}{ \begin{answer} $a_{1} = 3$; $a_{n} = 2 \cdot a_{n-1}$ \end{answer} }{ { ~\\ \raisebox{0pt}[2cm][0pt]{  } } }
    \end{answersheetquestion}

    ~\\
    
    \hrulefill
    
    ~\\
    % - QUESTION --------------------------------------------------%
    \begin{answersheetquestion}{7d}{Find recursive formula}{3}
        Find recursive formula for sequence: 2, 4, 16, 256, 65536
        \iftoggle{answerkey}{ \begin{answer} $a_{1} = 2$; $a_{n} = (a_{n-1})^{2}$ \end{answer} }{ { ~\\ \raisebox{0pt}[2cm][0pt]{  } } }
    \end{answersheetquestion}

    ~\\
    
    \hrulefill
    
    ~\\
    % - QUESTION --------------------------------------------------%
    \begin{answersheetquestion}{8a}{Find recursive formula}{3}
        Find recursive formula for sequence: 1, 3, 7, 15, 31
        \iftoggle{answerkey}{ \begin{answer} $a_{1} = 1$; $a_{n} = 2 \cdot a_{n-1} + 1$ \end{answer} }{ { ~\\ \raisebox{0pt}[2cm][0pt]{  } } }
    \end{answersheetquestion}

    ~\\
    
    \hrulefill
    
    ~\\
    % - QUESTION --------------------------------------------------%
    \begin{answersheetquestion}{8b}{Find recursive formula}{3}
        Find recursive formula for sequence: 2, 5, 11, 23, 47
        \iftoggle{answerkey}{ \begin{answer} $a_{1} = 2$; $a_{n} = 2 \cdot a_{n-1} + 1$ \end{answer} }{ { ~\\ \raisebox{0pt}[2cm][0pt]{  } } }
    \end{answersheetquestion}

    ~\\
    
    \hrulefill
    
    ~\\
    % - QUESTION --------------------------------------------------%
    \begin{answersheetquestion}{8c}{Find recursive formula}{3}
        Find recursive formula for sequence: 1, 5, 17, 53, 161
        \iftoggle{answerkey}{ \begin{answer} $a_{1} = 1$; $a_{n} = 3 \cdot a_{n-1} + 2$ \end{answer} }{ { ~\\ \raisebox{0pt}[2cm][0pt]{  } } }
    \end{answersheetquestion}

    ~\\
    
    \hrulefill
    
    ~\\
    % - QUESTION --------------------------------------------------%
    \begin{answersheetquestion}{8d}{Find recursive formula}{3}
        Find recursive formula for sequence: 1, 4, 10, 22, 46
        \iftoggle{answerkey}{ \begin{answer} $a_{1} = 1$; $a_{n} = 2 \cdot a_{n-1} + 2$ \end{answer} }{ { ~\\ \raisebox{0pt}[2cm][0pt]{  } } }
    \end{answersheetquestion}

    ~\\

    \hrulefill
    
    ~\\
    % - QUESTION --------------------------------------------------%
    \begin{answersheetquestion}{9a}{Summations}{6}
        Evaluate the sum: $\sum_{k=1}^{4}(3k) $

        ~\\
        First, List out all the elements of the sequence $a_{k} = 3k$: \\
        $a_{1} = $ \iftoggle{answerkey}{ \begin{answer} 3 \end{answer} }{} \tab
        $a_{2} = $ \iftoggle{answerkey}{ \begin{answer} 6 \end{answer} }{} \tab
        $a_{3} = $ \iftoggle{answerkey}{ \begin{answer} 9 \end{answer} }{} \tab
        $a_{4} = $ \iftoggle{answerkey}{ \begin{answer} 12 \end{answer} }{}

        ~\\
        Then, add each of the elements to get the result of
        $\sum_{k=1}^{4}(3k) $ \iftoggle{answerkey}{ \begin{answer}
            $3 + 6 + 9 + 12 = 30$
        \end{answer} }{}
    \end{answersheetquestion}

    ~\\
    
    \hrulefill{}

    
    ~\\
    % - QUESTION --------------------------------------------------%
    \begin{answersheetquestion}{9b}{Summations}{6}
        Evaluate the sum: $\sum_{k=1}^{5}(4) $

        ~\\
        First, List out all the elements of the sequence $a_{k} = 4$: \\
        $a_{1} = $ \iftoggle{answerkey}{ \begin{answer} 4 \end{answer} }{} \tab
        $a_{2} = $ \iftoggle{answerkey}{ \begin{answer} 4 \end{answer} }{} \tab
        $a_{3} = $ \iftoggle{answerkey}{ \begin{answer} 4 \end{answer} }{} \tab
        $a_{4} = $ \iftoggle{answerkey}{ \begin{answer} 4 \end{answer} }{} \tab
        $a_{5} = $ \iftoggle{answerkey}{ \begin{answer} 4 \end{answer} }{}

        ~\\
        Then, add each of the elements to get the result of 
        $\sum_{k=1}^{5}(4) $ \iftoggle{answerkey}{ \begin{answer}
            $4 + 4 + 4 + 4 + 4 = 20$
        \end{answer} }{}
    \end{answersheetquestion}
    
    %------------------------------------------------------------------%
    %- Grading --------------------------------------------------------%
    %------------------------------------------------------------------%
    \newpage
    \subsection*{Grading}
    
    \begin{center}
        
        \begin{tabular}{ | l | l | l | l | }
            \hline
            \textbf{ Question } & \textbf{ Weight } & \textbf{ 0-4 } & \textbf{ Adjusted score }
            \\ \hline{}
            
            1 & 4\% & &    \\ \hline
            2 & 12\% & &    \\ \hline
            3a & 6\% & &    \\ \hline
            3b & 6\% & &    \\ \hline
            4a & 3\% & &    \\ \hline
            4b & 3\% & &    \\ \hline
            4c & 3\% & &    \\ \hline
            4d & 3\% & &    \\ \hline
            5a & 3\% & &    \\ \hline
            5b & 3\% & &    \\ \hline
            5c & 3\% & &    \\ \hline
            5d & 3\% & &    \\ \hline
            6a & 3\% & &    \\ \hline
            6b & 3\% & &    \\ \hline
            6c & 3\% & &    \\ \hline
            6d & 3\% & &    \\ \hline
            7a & 3\% & &    \\ \hline
            7b & 3\% & &    \\ \hline
            7c & 3\% & &    \\ \hline
            7d & 3\% & &    \\ \hline
            8a & 3\% & &    \\ \hline
            8b & 3\% & &    \\ \hline
            8c & 3\% & &    \\ \hline
            8d & 3\% & &    \\ \hline
            9a & 6\% & &    \\ \hline
            9b & 6\% & &    \\ \hline
            & & & \\ \hline
            & & & \\ \hline
            
            
        \end{tabular}
    \end{center}
    
\end{document}
