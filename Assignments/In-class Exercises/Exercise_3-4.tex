\documentclass[a4paper,12pt]{book}
\usepackage[utf8]{inputenc}
\title{}
\author{Rachel Morris}
\date{\today}

\usepackage{rachwidgets}
\usepackage{fancyhdr}
\usepackage{lastpage}
\usepackage{dirtree}
\usepackage{boxedminipage}

\setcounter{chapter}{3}
\setcounter{section}{3}
\newcommand{\laChapter}{3.4 Boolean Algebra\ }
\newcounter{question}

\newcommand{\laClass}{CS 210\ }
\newcommand{\laSemester}{Fall 2017\ }

\pagestyle{fancy}
\fancyhf{}
\lhead{\laClass \laSemester}
\chead{}
\rhead{Ch \laChapter}
\rfoot{\thepage\ of \pageref{LastPage}}
\lfoot{\scriptsize Compiled by Rachel Morris, last updated \today}

\renewcommand{\headrulewidth}{2pt}
\renewcommand{\footrulewidth}{1pt}

\begin{document}

    %\toggletrue{answerkey}
    \togglefalse{answerkey}

    %------------------------------------------------------------------%
    %- Exercise Begin -------------------------------------------------%
    %------------------------------------------------------------------%

    \section{Boolean Algebra}

	\subsection{Logic, Sets, and Boolean Algebra}
	
	\begin{intro}{\ }
		There are a lot of similarities between the operations we have
		been doing on sets and with the logic operators we used back in Chapter 1.
		Now, we're also going to introduce the idea of Boolean Algebra,
		which also has similarities to Sets and Logic.
		
		\begin{center}
			\begin{tabular}{l c c c}
				& & & \textbf{Boolean}
				\\
				& \textbf{Logic} & \textbf{Sets} & \textbf{Algebra}
				\\ \hline
				\textbf{Variables} &
					$p$, $q$, $r$ & $A$, $B$, $C$ & $a$, $b$, $c$ 
				\\ \\
				\textbf{``and'' operation} &
					$\land$ & $\cap$ & $\cdot$
				\\ \\
				\textbf{``or'' operation} &
					$\lor$ & $\cup$ & $+$
				\\ \\
				\textbf{``not'' operation} &
					$\neg$ & $'$ & $'$
				\\ \\
				\textbf{``-'' operation} &
					$a \land \neg b$ & $A - B$ & $a \cdot b'$
				\\ \\
				\textbf{Special} &
					Tautology & Universal set U & 1
					\\ \\
					& Contradiction & Empty set $\emptyset$ & 0
			\end{tabular}
		\end{center}
		
		\paragraph{Example:} Rephrase the following Logic operation using Set and Boolean Algebra notations: $ ( p \land q ) \lor r $
		
		Sets: $(P \cap Q) \cup R$
		
		Boolean algebra: $( p \cdot q ) + r$
	\end{intro}
		
		\newpage
		
        % - QUESTION --------------------------------------------------%
        \stepcounter{question}
        \begin{questionNOGRADE}{\thequestion}
			
			Rewrite the following using Boolean Algebra notation:
			
			\begin{enumerate}
				\item[a.] 	$p \land q$							\solution{ $p \cdot q$ }{}
				\item[b.] 	$p \lor q$								\solution{ $p + q$ }{}
				\item[c.] 	$\neg p$								\solution{ $p'$ }{}
				\item[d.]	$(p \land \neg q) \lor p$			\solution{ $(p \cdot q' ) + p$ }{}
				\item[e.]	$\neg(\neg p))$						\solution{ $(p')'$ }{}
				\item[f.]	$(p \land \neg q) \lor p \equiv p$	\solution{ $(p \cdot q') + p = p$ }{}
			\end{enumerate}
			
        \end{questionNOGRADE}
        
		\hrulefill
		
        % - QUESTION --------------------------------------------------%
        \stepcounter{question}
        \begin{questionNOGRADE}{\thequestion}
			
			Rewrite the following using Boolean Algebra notation:
			
			\begin{enumerate}
				\item[a.] 	$A \cap B$								\solution{ $a \cdot b$ }{}
				\item[b.] 	$A \cup B$								\solution{ $a + b$ }{}
				\item[c.] 	$A'$									\solution{ $a'$ }{}
				\item[d.]	$(A-B)$									\solution{ $a \cdot b'$ }{}
				\item[e.]	$A' \cup (A \cap B)$					\solution{ $a' + (a \cdot b)$ }{}
				\item[f.]	$(A - B)' = A' \cup (A \cap B)$		\solution{ $(a \cdot b')' = a'$ }{}
			\end{enumerate}
			
        \end{questionNOGRADE}
        
        

\end{document}








