\documentclass[a4paper,12pt]{book}
\usepackage[utf8]{inputenc}
\title{}
\author{Rachel Morris}
\date{\today}

\usepackage{rachwidgets}
\usepackage{fancyhdr}
\usepackage{lastpage}
\usepackage{boxedminipage}

\newcommand{\laChapter}{1.4\ }

\pagestyle{fancy}
\fancyhf{}
\lhead{CS 210}
\chead{Fall 2017}
\rhead{Ch \laChapter Exercise}
\rfoot{\thepage\ of \pageref{LastPage}}
\lfoot{\scriptsize By Rachel Morris, last updated \today}

\renewcommand{\headrulewidth}{2pt}
\renewcommand{\footrulewidth}{1pt}

\begin{document}

    %\toggletrue{answerkey}
    \togglefalse{answerkey}

    %------------------------------------------------------------------%
    %- INSTRUCTIONS ---------------------------------------------------%
    %------------------------------------------------------------------%

    \chapter*{Chapter \laChapter In-class Exercise} \stepcounter{chapter}
    \footnote{
        \iftoggle{answerkey}{
          \begin{answer} ANSWER KEY \end{answer}
        }{}
    }

    %------------------------------------------------------------------%
    %- Exercise Begin -------------------------------------------------%
    %------------------------------------------------------------------%
    \begin{center} \section*{Chapter \laChapter In-class Exercise Instructions} \end{center}

    \begin{center}
        \textbf{This is the instruction page,
        make sure to fill out your answers in the worksheet later in the document}
    \end{center}

    %------------------------------------------------------------------%
    \section*{1. Predicates}

        \begin{intro}{Predicates\\}
            In mathematical logic, a \textbf{predicate} is commonly understood to be a Boolean-valued function.
            \footnote{From https://en.wikipedia.org/wiki/Predicate\_(mathematical\_logic)}

            We write a predicate as a function, such as $P(x)$, for example:

            \begin{center}
                $P(x)$ is the predicate, ``x is less than 2".
            \end{center}

            Once some value is plugged in for $x$, the result is a proposition -
            something either unambiguously \textbf{true} or \textbf{false},
            but until we have some input for $x$, we don't know whether it
            is true or false.

            \begin{center}
                $P(0)$ = true \tab $P(2)$ = false \tab $P(10)$ = false
            \end{center}

            Additionally, predicates can also be combined with the logical
            operators AND $\land$ OR $\lor$ and NOT $\neg$.
        \end{intro}

        \newpage
        % - QUESTION --------------------------------------------------%
        \begin{question}{1}{15\%}
            For the following predicates given, plug in \textbf{2, 23, -5, and 15}
            as inputs and write out whether the result is true or false.
        \end{question}

        \begin{enumerate}
            \item[a.] $P(x)$ is the predicate ``$x > 15$"
            \item[b.] $Q(x)$ is the predicate ``$x \leq 15$"
            \item[c.] $R(x)$ is the predicate ``$(x > 5) \land (x < 20)$"
        \end{enumerate}

        \begin{intro}{Domain\\}
        When we're working with predicates, we will also define the domain.

        The \textbf{domain} is the set of all possible inputs for our predicate.
        In other words, $x$ must be chosen from the domain.
        \end{intro}
        
        % - QUESTION --------------------------------------------------%
        \begin{question}{2}{20\%}
            For the following predicates and domains given, specify
            whether the predicate is true for \textbf{all members of the domain},
            \textbf{some members of the domain}, or \textbf{no members of the domain.}
        \end{question}

        \begin{enumerate}
            \item[a.] $P(x)$ is the predicate ``$x > 15$", the domain is \{10, 12, 14, 16, 18\}.
            \item[b.] $Q(x)$ is the predicate ``$x \leq 15$", the domain is \{0, 1, 2, 3\}.
            \item[c.] $R(x)$ is the predicate ``$(x > 5) \land (x < 20)$", the domain is \{0, 1, 2\}.
            \item[d.] $S(x)$ is the predicate ``$(x > 1) \land (x < 5)$", domain is \{2, 3, 4\}.
        \end{enumerate}
         

    %------------------------------------------------------------------%
    \newpage
    \section*{2. Quantifiers}

        \begin{intro}{Quantifiers\\}
        Symbolically, we can specify that the input of our predicate, $x$,
        belongs in some domain set $D$ with the notation: $x \in D$.
        This is read as, ``$x$ exists in the domain $D$."

        Additionally, we can also specify whether a predicate is true
        \textbf{for all inputs $x$ from the domain $D$} using the ``for all"
        symbol $\forall$, or we can specify that the predicate is true
        \textbf{for \textit{some} inputs $x$ from the domain $D$} using
        the ``there exists" symbol $\exists$.{}

        \paragraph{Example:} Rewrite the predicate symbolically.
            $P(x)$ is ``$x > 15$", the domain D is \{16, 17, 18\}.
            Here we can see that all inputs from the domain will
            result in the predicate evaluating to true, so we can write:

            \begin{center}
                $ \forall x \in D, P(x) $ (``For all x in D, x is greater than 15.")
            \end{center}

        \begin{itemize}
            \item The symbol $\in$ (``in") indicates membership in a set.
            \item The symbol $\forall$ (``for all") means "for all", or "every".
            \item The symbol $\exists$ (``there exists") means "there is (at least one)", or "there exists (at least one)".
            \item The symbols $\forall$ and $\exists$ are called \textbf{quantifiers}. When used
                with predicates, the statement is called a \textbf{quantified predicate}.
        \end{itemize}
        \end{intro}

        
        % - QUESTION --------------------------------------------------%
        \begin{question}{3}{12\%}
            For the following predicates, rewrite the sentence symbolically,
            as in the example above.
            Use either $\forall$ or $\exists$, based on whether the
            predicate is true for the domain given.
        \end{question}

        \begin{hint}{Hint}
            If a predicate P(x) is false for all elements in the domain, you
            can phrase it as: ``$\forall x \in D, \neg P(x)$".
        \end{hint}

        \begin{enumerate}
            \item[a.] $P(x)$ is the predicate ``$x > 15$", the domain is \{10, 12, 14, 16, 18\}.
            \item[b.] $Q(x)$ is the predicate ``$x \leq 15$", the domain is \{0, 1, 2, 3\}.
            \item[c.] $R(x)$ is the predicate ``$(x > 5) \land (x < 20)$", the domain is \{0, 1, 2\}.
            \item[d.] $S(x)$ is the predicate ``$(x > 1) \land (x < 5)$", domain is \{2, 3, 4\}.
        \end{enumerate}

        \newpage
        % - QUESTION --------------------------------------------------%
        \begin{question}{4}{14\%}
            For the following predicates, rewrite the sentence symbolically
            using the domain D = \{ 3, 4, 5, 10, 20, 25 \}.
            Make sure to define your predicates (state that ``$P(x)$ is the predicate...",
            and afterwards specify whether the quantified predicate is true or false.
            (If it states ``there exists" but none exist, then the quantified predicate is false.)
        \end{question}

        \begin{enumerate}
            \item[a.] There is (at least one) $k$ in the set $D$ with the property
                that $k^{2}$ is also in the set $D$.
                \begin{hint}{Hint}
                    How can you specify the predicate, ``$k^{2}$ is in the set $D$" symbolically?
                \end{hint}

            \item[b.] There exists some $m$ that is a member of $D$, such that
                $m \geq 3$.
        \end{enumerate}


    %------------------------------------------------------------------%
    \section*{3. Negating quantifiers}

        \begin{intro}{Proposition 1\\}
        For any predicates $P$ and $Q$ over a domain $D$,
        \begin{itemize}
            \item The negation of $\forall x \in D, P(x)$ is $\exists x \in D, \neg P(x)$.
            \item The negation of $\exists x \in D, P(x)$ is $\forall x \in D, \neg P(x)$.
        \end{itemize}

        When negating a predicate that uses an equal sign, the negation would be ``not equals".

        \paragraph{Example:} $\forall x \in \mathbb{Z}, \exists y \in \mathbb{Z}, x \cdot y = 0$.
        \begin{enumerate}
            \item $\neg( \forall x \in \mathbb{Z}, \exists y \in \mathbb{Z}, x \cdot y = 0 )$
            \item $\equiv$ \tab $\exists x \in \mathbb{Z}, \forall y \in \mathbb{Z}, x \cdot y \neq 0$
        \end{enumerate}
        \end{intro}

    \newpage

        % - QUESTION --------------------------------------------------%
        \begin{question}{5}{15\%}
            Write the negation of each of these statements. Simplify as much as possible.
        \end{question}

        \begin{enumerate}
            \item[a.] $\forall x \in \mathbb{Z},$ \tab[0.2cm] $\exists y \in \mathbb{Z},$ \tab[0.2cm] $2x+y = 3$
            \item[b.] $\exists x \in \mathbb{N},$ \tab[0.2cm] $\forall y \in \mathbb{N},$ \tab[0.2cm] $x \cdot y < x$
            \item[c.] $\exists x \in \mathbb{Z},$ \tab[0.2cm] $\exists y \in \mathbb{Z},$ \tab[0.2cm] $(x + y = 13) \land (x \cdot y = 36)$
        \end{enumerate}

        \begin{hint}{Hint:Negating propositions}
            From DeMorgan's laws,
            \begin{center}
                $\neg(p \land q) \equiv \neg p \lor \neg q$.
            \end{center} 
        \end{hint}

        \begin{hint}{Sets of numbers}
            Some sets we will be using often in this class are...
            \begin{itemize}
                \item[] $\mathbb{Z}$ = ``The set of all integers".
                \item[] $\mathbb{N}$ = ``The set of all natural numbers".
                \item[] $\mathbb{Q}$ = ``The set of all rational numbers".
                \item[] $\mathbb{R}$ = ``The set of all real numbers".
            \end{itemize}
        \end{hint}

        \hrulefill
        % - QUESTION --------------------------------------------------%
        \begin{question}{6}{24\%}
            Which elements of the set D = \{2, 4, 6, 8, 10, 12\} make the
            \textbf{negation} of each of these predicates true?
        \end{question}

        \begin{enumerate}
            \item[a.] $Q(n)$ is the predicate, ``$n > 10$".
            \item[b.] $R(n)$ is the predicate, ``$n$ is even".
            \item[c.] $S(n)$ is the predicate, ``$n^{2} < 1$".
            \item[d.] $T(n)$ is the predicate, ``$n-2$ is an element of $D$".
        \end{enumerate}

    
    %------------------------------------------------------------------%
    %- WORKSHEET ------------------------------------------------------%
    %------------------------------------------------------------------%
    \newpage
    \begin{center} \section*{Chapter \laChapter In-class Exercise Worksheet} \end{center}

    \iftoggle{answerkey}{
      \begin{answer} \begin{center} ANSWER KEY \end{center} \end{answer}
    }{}

    \paragraph{Team:}
    Please write down all people in your team. ~\\

    % table %
    \begin{tabular}{ p{6cm} p{6cm} }
        1. & 2. \\
        3. & 4.
    \end{tabular}
    % table %

    \paragraph{Section:} ~\\
        $\Box$ MW 4:30 - 5:45 pm \tab
        $\Box$ M 6:00 - 8:50 pm \tab
        $\Box$ TR 2:00 - 3:15 pm

    \paragraph{Rules:}

    \begin{itemize}
        \item \textbf{Only one worksheet will be turned in per team.}
            Each member of the team will receive the same score.
        \item You can collaborate on the exercise together, or you can
            work separately and then compare your answers with your team
            as you fill out the turn-in worksheet.
        \item Fill out your answers on this worksheet.
        \item \textbf{Write cleanly and linearly} - if I can't make sense of
            your solution then you won't get credit.
        \item \textbf{Don't scribble out cancellations} - I can't read that.
            For example, if a numerator/denominator cancel out, or a +/-
            cancels out, don't scribble out the numbers - just use a single slash!
    \end{itemize}


    
    %------------------------------------------------------------------%
    %- Grading --------------------------------------------------------%
    %------------------------------------------------------------------%
    \newpage
    \subsection*{Grading}
    
        The total amount of points for an in-class exercise is 5 points each.
        Each question will have a weight assigned to it, and will be given
        a a point value between 0 and 4:

    \begin{center}
        \begin{tabular}{ | l | l | }
            \hline
            0 & Nothing written \\ \hline
            1 & Something written, but incorrect \\ \hline
            2 & Partially correct, but multiple errors \\ \hline
            3 & Mostly correct, with one or two errors \\ \hline
            4 & Perfect; correct answer and notation \\ \hline

        \end{tabular}
    \end{center}

    \begin{center}

        \begin{tabular}{ | l | l | l | l | }
            \hline
            \textbf{ Question } & \textbf{ Weight } & \textbf{ 0-4 } & \textbf{ Adjusted score }
            \\ \hline

            1 & 15\% & &    \\ \hline
            2 & 20\% & &    \\ \hline
            3 & 12\% & &    \\ \hline
            4 & 14\% & &    \\ \hline
            5 & 15\% & &    \\ \hline
            6 & 24\% & &    \\ \hline
            & & & \\ \hline
            & & & \\ \hline


        \end{tabular}
    \end{center}

    %------------------------------------------------------------------%
    %- Exercise Begin -------------------------------------------------%
    %------------------------------------------------------------------%

    % \iftoggle{answerkey}{ \begin{answer} asdfasdf \end{answer} }{ { ~\\ \raisebox{0pt}[2cm][0pt]{  } } }
    % \iftoggle{answerkey}{ \begin{answer} TRUE \end{answer} }{}

    %------------------------------------------------------------------%
    \newpage{}
    \section*{Answer sheet}

    ~\\
    % - QUESTION --------------------------------------------------%
    \begin{answersheetquestion}{1}{True or false?}{15}
    \end{answersheetquestion}

    \begin{tabular}{l l l l}
        $P(2):$         \iftoggle{answerkey}{ \begin{answer} False \end{answer} }{}
        & $P(23):$      \iftoggle{answerkey}{ \begin{answer} True \end{answer} }{}
        & $P(-5):$      \iftoggle{answerkey}{ \begin{answer} False \end{answer} }{}
        & $P(15):$      \iftoggle{answerkey}{ \begin{answer} False \end{answer} }{}
        \\ \hline
        $Q(2):$         \iftoggle{answerkey}{ \begin{answer} True \end{answer} }{}
        & $Q(23):$      \iftoggle{answerkey}{ \begin{answer} False \end{answer} }{}
        & $Q(-5):$      \iftoggle{answerkey}{ \begin{answer} True \end{answer} }{}
        & $Q(15):$      \iftoggle{answerkey}{ \begin{answer} True \end{answer} }{}
        \\ \hline
        $R(2):$         \iftoggle{answerkey}{ \begin{answer} False \end{answer} }{}
        & $R(23):$      \iftoggle{answerkey}{ \begin{answer} False \end{answer} }{}
        & $R(-5):$      \iftoggle{answerkey}{ \begin{answer} False \end{answer} }{}
        & $R(15):$      \iftoggle{answerkey}{ \begin{answer} True \end{answer} }{}
        \\ \hline
    \end{tabular} ~\\

    \hrulefill
    
    % - QUESTION --------------------------------------------------%
    \begin{answersheetquestion}{2}{All, some, or none?}{20}
    \end{answersheetquestion}

    \begin{enumerate}
        \item[a.] \Square \ True for all \tab \Square \ True for some \tab \Square \ True for none \\
            \iftoggle{answerkey}{ \begin{answer} True for some \end{answer} }{}
            
        \item[b.] \Square \ True for all \tab \Square \ True for some \tab \Square \ True for none \\
            \iftoggle{answerkey}{ \begin{answer} True for all \end{answer} }{}

        \item[c.] \Square \ True for all \tab \Square \ True for some \tab \Square \ True for none \\
            \iftoggle{answerkey}{ \begin{answer} True for none \end{answer} }{}
            
        \item[d.] \Square \ True for all \tab \Square \ True for some \tab \Square \ True for none \\
            \iftoggle{answerkey}{ \begin{answer} True for all \end{answer} }{}
    \end{enumerate}

    \hrulefill
    
    % - QUESTION --------------------------------------------------%
    \begin{answersheetquestion}{3}{Write symbolically}{12}
    \end{answersheetquestion}

    \begin{enumerate}
        \item[a.] $P(x)$ is the predicate ``$x > 15$", the domain is \{10, 12, 14, 16, 18\}.
            \iftoggle{answerkey}{ \begin{answer}     $\exists x in D, P(x)$     \end{answer} }{}
        
        \item[b.] $Q(x)$ is the predicate ``$x \leq 15$", the domain is \{0, 1, 2, 3\}.
            \iftoggle{answerkey}{ \begin{answer}     $\forall x in D, Q(x)$     \end{answer} }{}
        
        \item[c.] $R(x)$ is the predicate ``$(x > 5) \land (x < 20)$", the domain is \{0, 1, 2\}.
            \iftoggle{answerkey}{ \begin{answer}     $\forall x in D, \neg R(x)$     \end{answer} }{}
        
        \item[d.] $S(x)$ is the predicate ``$(x > 1) \land (x < 5)$", domain is \{2, 3, 4\}.
            \iftoggle{answerkey}{ \begin{answer}     $\forall x in D, S(x)$     \end{answer} }{}
    \end{enumerate}
        

    \newpage
    % - QUESTION --------------------------------------------------%
    \begin{answersheetquestion}{4}{Write symbolically}{14}
    \end{answersheetquestion}

    \begin{enumerate}
        \item[a.] \iftoggle{answerkey}{ \begin{answer}     $\exists k \in D, P(x)$, where $P(x)$ is the predicate ``$k^{2} \in D$".     \end{answer} }{}
        \item[b.] \iftoggle{answerkey}{ \begin{answer}     $\exists m \in D, Q(x)$, where $Q(m)$ is the predicate ``$m \geq 3$".     \end{answer} }{}
    \end{enumerate}

    \hrulefill
    
    % - QUESTION --------------------------------------------------%
    \begin{answersheetquestion}{5}{Negations}{15}
    \end{answersheetquestion}

        \begin{enumerate}
            \item[a.] Negate $\forall x \in \mathbb{Z},$ \tab[0.2cm] $\exists y \in \mathbb{Z},$ \tab[0.2cm] $2x+y = 3$ \\
                \iftoggle{answerkey}{ \begin{answer}     $\exists x \in \mathbb{Z}, \forall y \in \mathbb{Z}, 2x + y \neq 3$     \end{answer} }{}
            ~\\
            \item[b.] Negate $\exists x \in \mathbb{N},$ \tab[0.2cm] $\forall y \in \mathbb{N},$ \tab[0.2cm] $x \cdot y < x$ \\
                \iftoggle{answerkey}{ \begin{answer}     $\forall x \in \mathbb{N}, \forall y \in \mathbb{N}, x \cdot y \geq x$     \end{answer} }{}
            ~\\
            \item[c.] Negate $\exists x \in \mathbb{Z},$ \tab[0.2cm] $\exists y \in \mathbb{Z},$ \tab[0.2cm] $(x + y = 13) \land (x \cdot y = 36)$ \\
                \iftoggle{answerkey}{ \begin{answer}     $\forall x \in \mathbb{Z}, \forall y \in \mathbb{Z}, (x + y \neq 13) \lor (x \cdot y \neq 36)$     \end{answer} }{}
            ~\\
        \end{enumerate}
        
    \hrulefill
    
    % - QUESTION --------------------------------------------------%
    \begin{answersheetquestion}{6}{True or false?}{24}
        Given the domain D = \{2, 4, 6, 8, 10, 12\}
    \end{answersheetquestion}

        \begin{enumerate}
            \item[a.] $Q(n)$ is the predicate, ``$n > 10$"; negation is true for inputs: \\
                \iftoggle{answerkey}{ \begin{answer}
                $\neg Q(n)$ is $n \leq 10$, so \{2, 4, 6, 8, and 10\}.
                \end{answer} }{}
                
            \item[b.] $R(n)$ is the predicate, ``$n$ is even"; negation is true for inputs: \\
                \iftoggle{answerkey}{ \begin{answer}
                $\neg R(n)$ is $n$ is odd, so NONE.
                \end{answer} }{}
            
            \item[c.] $S(n)$ is the predicate, ``$n^{2} < 1$"; negation is true for inputs: \\
                \iftoggle{answerkey}{ \begin{answer}
                $\neg S(n)$ is $n^{2} \geq 1$, so ALL.
                \end{answer} }{}
            
            \item[d.] $T(n)$ is the predicate, ``$n-2$ is an element of $D$"; negation is true for inputs: \\
                \iftoggle{answerkey}{ \begin{answer}
                $\neg T(n)$ is $n-1$ is not an element of $D$, so \{2\}.
                \end{answer} }{}
        \end{enumerate}

\end{document}
