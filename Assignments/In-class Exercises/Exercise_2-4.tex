\documentclass[a4paper,12pt]{book}
\usepackage[utf8]{inputenc}
\title{}
\author{Rachel Morris}
\date{\today}

\usepackage{rachwidgets}
\usepackage{fancyhdr}
\usepackage{lastpage}
\usepackage{dirtree}
\usepackage{boxedminipage}

\setcounter{chapter}{2}
\setcounter{section}{3}
\newcommand{\laChapter}{2.4 More about induction\ }
\newcounter{question}

\newcommand{\laClass}{CS 210\ }
\newcommand{\laSemester}{Fall 2017\ }

\pagestyle{fancy}
\fancyhf{}
\lhead{\laClass \laSemester}
\chead{}
\rhead{Ch \laChapter}
\rfoot{\thepage\ of \pageref{LastPage}}
\lfoot{\scriptsize Compiled by Rachel Morris, last updated \today}

\renewcommand{\headrulewidth}{2pt}
\renewcommand{\footrulewidth}{1pt}

\begin{document}

    %\toggletrue{answerkey}
    \togglefalse{answerkey}

    %------------------------------------------------------------------%
    %- Exercise Begin -------------------------------------------------%
    %------------------------------------------------------------------%

    \section{More about induction}

    %------------------------------------------------------------------%
    \subsection{Sums as recursive sequences}

        \begin{intro}{\ }
            Last time, we were proving that a sum and a closed formula
            were equivalent. Now we will use induction to prove that
            a sum and a recursive sequence are equivalent.
            Again, we have a set of steps that you'll need to follow
            in order to solve these.

            \paragraph{Example 1 from the textbook}
            Consider the sum $\sum_{i=1}^{n}(2i-1)$, which is the same
            as $1 + 3 + 5 + ... + (2n-1)$. Use the notation $s_{n}$ to
            denote this sum. Find a recursive description of $s_{n}$.

            \subparagraph{Step 1: Find the first term, $s_{1}$:}
                We solve for $s_{1}$, or in other words, $\sum_{i=1}^{1}(2i-1)$.
                
                $\sum_{i=1}^{1}(2i-1) = (2 \cdot 1 - 1) = 1$ \tab
                $s_{1} = 1$

            \subparagraph{Step 2: Restate the result of $s_{n}$ as $s_{n-1}$ plus the final term:}
                Similar to last time, we were rewriting some $\sum_{i=1}^{n}$
                as the sum up until $n-1$, plus the final term. Here,
                we're using $s_{n}$ to represent this sum from $i=1$ to $n$,
                so we can rewrite it as:

                $$ s_{n} = s_{n-1} + (2n-1)$$

            So we have found an equation and the first term, and we can
            show that the recursive formula that is equivalent is:

            $$ s_{1} = 1 \tab s_{n} = s_{n-1} + 2n - 1 $$
        \end{intro}

        \newpage
        % - QUESTION --------------------------------------------------%
        \stepcounter{question}
        \begin{questionNOGRADE}{\thequestion}

            Consider the sum $\sum_{i=1}^{n} (3i^{2})$. Use the notation
            $s_{n}$ to denote this sum. Find a recursive description of $s_{n}$.

            \paragraph{Step 1: Find $s_{1}$}

            \solution{
                $s_{1} = \sum_{i=1}^{1} (2i^{2}) = 2(1)^{2} = 2$
            }{ { ~\\ \raisebox{0pt}[1cm][0pt]{  } } }

            \paragraph{Step 2: Rewrite $s_{n}$ in terms of $s_{n-1}$ plus final term}
            
            \solution{
                $s_{n} = s_{n-1} + 3n^{2}$
            }{ { ~\\ \raisebox{0pt}[2cm][0pt]{  } } }

            So, the recursive formula is:

            \begin{framed}
                $s_{1} = $ \solution{2}{}
                \tab[4cm]
                $s_{n} = $ \solution{$s_{n-1} + 3n^{2}$}{}
            \end{framed}
            
        \end{questionNOGRADE}

        \hrulefill

        % - QUESTION --------------------------------------------------%
        \stepcounter{question}
        \begin{questionNOGRADE}{\thequestion}

            Consider the sum $\sum_{i=1}^{n} (2^{i-1}+1)$. Use the notation
            $s_{n}$ to denote this sum. Find a recursive description of $s_{n}$.

            \paragraph{Step 1: Find $s_{1}$}

            \solution{
                $s_{1} = \sum_{i=1}^{1} (2^{i-1}+1) = 2^{0} + 1 = 2$
            }{ { ~\\ \raisebox{0pt}[1cm][0pt]{  } } }

            \paragraph{Step 2: Rewrite $s_{n}$ in terms of $s_{n-1}$ plus final term}
            
            \solution{
                $s_{n} = s_{n-1} + 2^{n-1} + 1$
            }{ { ~\\ \raisebox{0pt}[2cm][0pt]{  } } }

            So, the recursive formula is:

            \begin{framed}
                $s_{1} = $ \solution{2}{}
                \tab[4cm]
                $s_{n} = $ \solution{$s_{n-1} + 2^{n-1} + 1$}{}
            \end{framed}
            
        \end{questionNOGRADE}

        \subsection{More proofs by induction}

        \begin{intro}{\ }
            \small 
            Back in the early parts of Chapter 2, we were proving that
            items were divisible by some number, or even, or odd.
            If we want to do this for a formula, we can use induction
            to show that the statement is true for \textit{all}
            values of $n$.

            \paragraph{Example 6 from the book}
                Show that $n^{3} + 2n$ is divisible by 3 for all positive
                integers $n$.

            \subparagraph{Step 1: Rephrase as a function:}
                $D(n) = n^{3} + 2n$

            \subparagraph{Step 2: Check proposition for $D(1)$:}
                Make sure that $D(1)$ is divisible by 3. \tab
                $D(1) = 1^{3} + 2 \cdot 1 = 3$ \checkmark{}

            ~\\
            Next, we assume that we have shown that the proposition holds
            for all numbers from $D(1)$ to $D(m-1)$, as part of our inductive
            proof. (We could also check $D(2)$, $D(3)$, etc.but we will just say
            we've checked $m-1$ values...)

            \subparagraph{Step 3: Write out $D(m-1)$ and simplify:} ~\\            
                $D(m-1) = (m-1)^{3} + 2(m-1) = m^{3} - 3m^{2} + 3m - 1 + 2m - 2$ \\
                Note that we're not completely adding like terms here.

            \subparagraph{Step 4: Rewrite so that $D(m)$ is part of the equation:} ~\\
                We want to reorganize our terms of $D(m-1)$ to include $D(m)$, which is
                $m^{3} + 2m$.

                $D(m-1) = (m^{3} + 2m) - 3m^{2} + 3m - 3$.

            \subparagraph{Step 5: Rewrite with $D(m)$:}
                $D(m-1) = D(m) - 3m^{2} + 3m - 3$.

            \subparagraph{Step 6: Solve for $D(m)$:}
                $D(m) = D(m-1) + 3m^{2} - 3m + 3$

            \subparagraph{Step 7: Replace $D(m-1)$...}
                From an earlier step, we said that we have proven
                the proposition for $D(1)$ through $D(m-1)$. This means
                that we ``have shown" that $D(m-1)$ is divisible by 3, or
                is some integer times 3. We can then write $D(m-1)$ as
                $3k$ in our equation instead...

                $D(m) = 3k + 3m^{2} - 3m + 3$

            \subparagraph{Step 8: Factor out common terms to prove:}
                Finally, to show it is divisible by 3, we factor out
                the 3 in the equation:

                $D(m) = 3(k + m^{2} - m + 1)$
        \end{intro}

        \newpage
        
        % - QUESTION --------------------------------------------------%
        \stepcounter{question}
        \begin{questionNOGRADE}{\thequestion}
            Use induction to prove that for each integer $n \geq 1$, $2n$ is even.

            \subparagraph{Step 1: Rephrase as a function:} ~\\
            \solution{
                $D(m) = 2m$
            }{ { ~\\ \raisebox{0pt}[0.5cm][0pt]{  } } }
            
            \subparagraph{Step 2: Check proposition for $D(1)$:} ~\\
            \solution{
                $D(1) = 2(1) = 2$ \checkmark
            }{ { ~\\ \raisebox{0pt}[0.5cm][0pt]{  } } }

            \subparagraph{Step 3: Write out $D(m-1)$ and simplify:} ~\\
            \solution{
                $D(m-1) = 2(m-1) = 2m - 2$
            }{ { ~\\ \raisebox{0pt}[1cm][0pt]{  } } }

            \subparagraph{Step 4: Rewrite so that $D(m)$ is part of the equation:} ~\\
            \solution{
                $D(m-1) = 2m - 2$
            }{ { ~\\ \raisebox{0pt}[0.5cm][0pt]{  } } }

            \subparagraph{Step 5: Rewrite with $D(m)$:}~\\
            \solution{
                $D(m-1) = D(m) - 2$
            }{ { ~\\ \raisebox{0pt}[0.5cm][0pt]{  } } }

            \subparagraph{Step 6: Solve for $D(m)$:}~\\
            \solution{
                $D(m) = D(m-1) + 2$
            }{ { ~\\ \raisebox{0pt}[0.5cm][0pt]{  } } }

            \subparagraph{Step 7: Replace $D(m-1)$:}~\\
            \solution{
                $D(m) = 2k + 2$
            }{ { ~\\ \raisebox{0pt}[0.5cm][0pt]{  } } }

            \subparagraph{Step 8: Factor out common terms to prove:}~\\
            \solution{
                $D(m) = 2(k + 1)$
            }{ { ~\\ \raisebox{0pt}[0.5cm][0pt]{  } } }
            
        \end{questionNOGRADE}

        \newpage
        
        % - QUESTION --------------------------------------------------%
        \stepcounter{question}
        \begin{questionNOGRADE}{\thequestion}
            Use induction to prove that for each integer $n \geq 1$, $4n+1$ is odd.

            \subparagraph{Step 1: Rephrase as a function:} ~\\
            \solution{
                $D(m) = 4m+1$
            }{ { ~\\ \raisebox{0pt}[0.5cm][0pt]{  } } }
            
            \subparagraph{Step 2: Check proposition for $D(1)$:} ~\\
            \solution{
                $D(1) = 4(1) + 1 = 5$ \checkmark
            }{ { ~\\ \raisebox{0pt}[0.5cm][0pt]{  } } }

            \subparagraph{Step 3: Write out $D(m-1)$ and simplify:} ~\\
            \solution{
                $D(m-1) = 4(m-1) + 1 = 4m - 4 + 1$
            }{ { ~\\ \raisebox{0pt}[1cm][0pt]{  } } }

            \subparagraph{Step 4: Rewrite so that $D(m)$ is part of the equation:} ~\\
            \solution{
                $D(m-1) = (4m + 1) - 4$
            }{ { ~\\ \raisebox{0pt}[0.5cm][0pt]{  } } }

            \subparagraph{Step 5: Rewrite with $D(m)$:}~\\
            \solution{
                $D(m-1) = D(m) - 4$
            }{ { ~\\ \raisebox{0pt}[0.5cm][0pt]{  } } }

            \subparagraph{Step 6: Solve for $D(m)$:}~\\
            \solution{
                $D(m) = D(m-1) + 4$
            }{ { ~\\ \raisebox{0pt}[0.5cm][0pt]{  } } }

            \subparagraph{Step 7: Replace $D(m-1)$:}~\\
            \solution{
                $D(m) = (2k+1) + 4 = 2k + 4 + 1$
            }{ { ~\\ \raisebox{0pt}[0.5cm][0pt]{  } } }

            \subparagraph{Step 8: Factor out common terms to prove:}~\\
            \solution{
                $D(m) = 2(k + 2) + 1$
            }{ { ~\\ \raisebox{0pt}[0.5cm][0pt]{  } } }
        \end{questionNOGRADE}

        \newpage
        
        % - QUESTION --------------------------------------------------%
        \stepcounter{question}
        \begin{questionNOGRADE}{\thequestion}
            Use induction to prove that for each integer $n \geq 1$, $n^{2} - n$ is even.

            \subparagraph{Step 1: Rephrase as a function:} ~\\
            \solution{
                $D(m) = m^{2} - m$
            }{ { ~\\ \raisebox{0pt}[0.5cm][0pt]{  } } }
            
            \subparagraph{Step 2: Check proposition for $D(1)$:} ~\\
            \solution{
                $D(1) = 1^{2} - 1 = 0$ \checkmark
            }{ { ~\\ \raisebox{0pt}[0.5cm][0pt]{  } } }
            ~\\ (This will be 0, but according to $n = 2k$, $2(0)$ is ``even"?)

            \subparagraph{Step 3: Write out $D(m-1)$ and simplify:} ~\\
            \solution{
                $D(m-1) = (m-1)^{2} - (m-1) = m^{2} - 2m + 1 - m + 1$
            }{ { ~\\ \raisebox{0pt}[1cm][0pt]{  } } }

            \subparagraph{Step 4: Rewrite so that $D(m)$ is part of the equation:} ~\\
            \solution{
                $D(m-1) = (m^{2} - m) - 2m + 2$
            }{ { ~\\ \raisebox{0pt}[0.5cm][0pt]{  } } }

            \subparagraph{Step 5: Rewrite with $D(m)$:}~\\
            \solution{
                $D(m-1) = D(m) - 2m  + 2$
            }{ { ~\\ \raisebox{0pt}[0.5cm][0pt]{  } } }

            \subparagraph{Step 6: Solve for $D(m)$:}~\\
            \solution{
                $D(m) = D(m-1) + 2m - 2$
            }{ { ~\\ \raisebox{0pt}[0.5cm][0pt]{  } } }

            \subparagraph{Step 7: Replace $D(m-1)$:}~\\
            \solution{
                $D(m) = 2k + 2m - 2$
            }{ { ~\\ \raisebox{0pt}[0.5cm][0pt]{  } } }

            \subparagraph{Step 8: Factor out common terms to prove:}~\\
            \solution{
                $D(m) = 2(k + m - 1)$
            }{ { ~\\ \raisebox{0pt}[0.5cm][0pt]{  } } }
        \end{questionNOGRADE}
        
\end{document}
