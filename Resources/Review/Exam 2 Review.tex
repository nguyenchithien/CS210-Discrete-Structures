\documentclass[a4paper,12pt]{book}
\usepackage[utf8]{inputenc}
\title{}
\author{Rachel Morris}
\date{\today}

\usepackage{rachwidgets}
\usepackage{fancyhdr}
\usepackage{lastpage}
\usepackage{boxedminipage}

\pagestyle{fancy}
\fancyhf{}
\lhead{CS 210}
\chead{Fall 2017}
\rhead{Exam 2 Review}
\rfoot{\thepage\ of \pageref{LastPage}}
\lfoot{By Rachel Morris, last updated \today}

\renewcommand{\headrulewidth}{2pt}
\renewcommand{\footrulewidth}{1pt}

\begin{document}

% ----------------------------------------------------------------------
% - question -----------------------------------------------------------
% ----------------------------------------------------------------------

\section*{Practice problems}

\textbf{Question 1: Direct proofs}
\hfill
Chapter 2.1
\hfill
10\%

Prove the following statements using a \textbf{direct proof}.
Make sure to write the \textbf{final answer} in terms of
the appropriate definition.

\begin{enumerate}
	\item		If $ n $ is even, then $ n^{2} - n $ is even.
	\item		The product of two odd integers is always odd.
\end{enumerate}

\hrulefill

% ----------------------------------------------------------------------
% - question -----------------------------------------------------------
% ----------------------------------------------------------------------
\textbf{Question 2: Division theorem}
\hfill
Chapter 2.2
\hfill
10\%

Fill in the blanks in the style of the \textbf{division theorem}.

\begin{quote} For all integers a and b (with $ b > 0 $),
there is an integer $ q $ and an integer $ r $ such that: \\
1. $ a = b \cdot q + r $ and \\
2. $ 0 \leq r < b $.
\end{quote}

\begin{enumerate}	
	\item		$ 20 = 				\underline{\hspace{1cm}} \cdot 		6 + \underline{\hspace{1cm}} $
	\item	$ -13 = 			\underline{\hspace{1cm}} \cdot 		6 + \underline{\hspace{1cm}} $
	\item	$ (3k^{2} + 8) = 	\underline{\hspace{1cm}} \cdot 		3 + \underline{\hspace{1cm}} $
\end{enumerate}

\hrulefill

% ----------------------------------------------------------------------
% - question -----------------------------------------------------------
% ----------------------------------------------------------------------
\textbf{Question 3: Modulus}
\hfill
Chapter 2.2
\hfill
10\%

Solve the following modulus problems

\begin{enumerate}	
	\item			$ 73 			$ mod $ 	6 $
	\item		$ -15  	$ mod $ 		2  $
\end{enumerate}

\hrulefill

% ----------------------------------------------------------------------
% - question -----------------------------------------------------------
% ----------------------------------------------------------------------
\textbf{Question 4: Proof by induction}
\hfill
Chapter 2.3
\hfill
10\%

Show that the sequence defined by the recursive formula

\begin{center}
$ a_{k} = a_{k-1} + 2 $ , where $ a_{1} = 2 $, and for $ k \geq 2 $
\end{center}

is equivalently described by the closed formula \tab
$ a_{n} = 2n $

\newpage


% ----------------------------------------------------------------------
% - question -----------------------------------------------------------
% ----------------------------------------------------------------------
~\\
\textbf{Question 5: Proof by induction}
\hfill
Chapter 2.3
\hfill
10\%

Use induction to prove that $ \sum_{i=1}^{n} (2i-1) = n^{2} $
for each $ n \geq 1 $

\hrulefill
% ----------------------------------------------------------------------
% - question -----------------------------------------------------------
% ----------------------------------------------------------------------
~\\
\textbf{Question 6: Deriving a recursive formula from a sum}
\hfill
Chapter 2.4
\hfill
10\%

Consider the sum
$ \sum_{i=1}^{n} (3i + 2) $

Use $ s_{n} $ to denote this sum.
Find a recursive description of $ s_{n} $.

\hrulefill

% ----------------------------------------------------------------------
% - question -----------------------------------------------------------
% ----------------------------------------------------------------------
~\\
\textbf{Question 7: Proof by contradiction}
\hfill
Chapter 2.5
\hfill
10\%

Use \textbf{proof by contradiction} to prove the following statement:

If $ n^{2} - 1 $ is divisible by 5, then $ n $ is not divisible by 5.

\hrulefill

% ----------------------------------------------------------------------
% - question -----------------------------------------------------------
% ----------------------------------------------------------------------
~\\
\textbf{Question 8: Numerical representations}
\hfill
Chapter 2.6
\hfill
10\%

Write the following numbers as the sum of multiples of powers of the base.
\textit{\textbf{Do not simplify!}}

\begin{enumerate}
	\item	\texttt{(246)}$_{10} = $		% $ 2 \cdot 10^{2} + 4 \cdot 10^{1} + 6 \cdot 10^{0} $
	\item	\texttt{(0100 1001)}$_{2} = $	% $ 1 \cdot 2^{0} + 0 \cdot 2^{1} + 0 \cdot 2^{2} + 1 \cdot 2^{4} + 0 \cdot 2^{5} + 0 \cdot 2^{6} + 1 \cdot 2^{7} $
	\item	\texttt{(F00D)}$_{16} = $		% $ 15 \cdot 16^{3} + 0 \cdot 16^{2} + 0 \cdot 16^{2} + 13 \cdot 16^{0} $
\end{enumerate}

\hrulefill

% ----------------------------------------------------------------------
% - question -----------------------------------------------------------
% ----------------------------------------------------------------------
~\\
\textbf{Question 9: Binary $ \leftrightarrow $ Hexadecimal}
\hfill
Chapter 2.6
\hfill
10\%

\begin{center}
	\begin{tabular}{ | l | l | l | l | l | l | l | l | l |  }
	
	\hline 		\textbf{HEX} & 0 & 1 & 2 & 3 & 4 & 5 & 6 & 7  	\\ 
	\hline 		\textbf{BINARY} & 0000 & 0001 & 0010 & 0011 & 0100 & 0101 & 0110 & 0111  	\\ \hline
				\textbf{HEX} & 8 & 9 & A & B & C & D & E & F 	\\ 
	\hline 		\textbf{BINARY} & 1000 & 1001 & 1010 & 1011 & 1100 & 1101 & 1110 & 1111 	\\ \hline
		
	\end{tabular}
\end{center}

Using the table, convert directly between base-2 and base-16.

\begin{enumerate}
	\item	Convert \texttt{( 1111 0000 0000 1101 )}$_{2} $ to base-16
	\item	Convert \texttt{( CAF3 )}$_{16} $ to base-2
\end{enumerate}

\newpage

% ----------------------------------------------------------------------
% - question -----------------------------------------------------------
% ----------------------------------------------------------------------
~\\
\textbf{Question 10: Binary $ \leftrightarrow $ Decimal}
\hfill
Chapter 2.6
\hfill
10\%

Using an algorithm, convert between base-2 and base-10.

\begin{enumerate}
	\item	Convert \texttt{( 75 )}$_{10} $ to base-2
	\item	Convert \texttt{( 0101 1010 )}$_{2} $ to base-10
\end{enumerate}

% ----------------------------------------------------------------------
% - solution -----------------------------------------------------------
% ----------------------------------------------------------------------

\solution{
\section*{Answer key}
\subsection*{Solution: Question 1}

1. If $ n $ is even, then $ n^{2} - n $ is even.

\begin{enumerate}
    \item   $n = 2k$
    \item   $n^{2} - n \Rightarrow (2k)^{2} - 2k$
    \item   $4k^{2} - 2k$
    \item   $2(2k^{2} - k)$
\end{enumerate}

~\\
2. The product of two odd integers is always odd.

\begin{enumerate}
    \item   Integer 1: $n = 2k+1$ \tab Integer 2: $m = 2j+1$.
    \item   $n \cdot m \Rightarrow (2k+1)(2j+1)$
    \item   $4kj + 2k + 2j + 1$
    \item   $2(2kj + k + j) + 1$
\end{enumerate}

}{}

\solution{
\subsection*{Solution: Question 2}

\begin{enumerate}
    \item   $20             = 3 \cdot 6 + 2$
    \item   $-13            = -3 \cdot 6 + 5$
    \item   $(3k^{2} + 8)   = (k^{2} + 2) \cdot 3 + 2$
\end{enumerate}

}{}

\solution{
\subsection*{Solution: Question 3}

\begin{enumerate}
    \item   73 mod 6 = 1
    \item   -15 mod 2 = 1
\end{enumerate}

}{}

\solution{
\subsection*{Solution: Question 4}

\begin{enumerate}
    \item   Check first term: \\
        Recursive: $a_{1} = 2$ \tab Closed: $a_{1} = 2(1)$
    \item   Find $a_{m-1}$ with closed formula: \\
        $a_{m-1} = 2(m-1) = 2m-2$
    \item   Plug $a_{m-1}$ into the recursive formula: \\
        $a_{m} = a_{m-1} + 2$ \\
        $a_{m} = 2m-2 + 2$ \\
        $a_{m} = 2m$
    \item   This matches the closed formula given, so we have proved it.
\end{enumerate}

}{}

% ----------------------------------------------------------------------
% - solution -----------------------------------------------------------
% ----------------------------------------------------------------------

\newpage

\solution{
\subsection*{Solution: Question 5}

\begin{enumerate}
    \item   Check first 3 terms:
        \begin{enumerate}
            \item   $n = 1$: \\
                Left-hand side: $\sum_{i=1}^{1} (2i-1)
                    = (2(1) - 1) = 1$ \\
                Right-hand side: $1^{2} = 1$ \tab \checkmark{}

            \item   $n = 2$: \\
                Left-hand side: $\sum_{i=1}^{2} (2i-1)
                    = (2(1) - 1) + (2(2) - 1) \\ = 1 + 3 = 4$ \\
                Right-hand side: $2^{2} = 4$ \tab \checkmark{}

            \item   $n = 3$: \\
                Left-hand side: $\sum_{i=1}^{3} (2i-1)
                    = (2(1) - 1) + (2(2) - 1) + (2(3) - 1) \\ = 1 + 3 + 5 = 9$ \\
                Right-hand side: $3^{2} = 9$ \tab \checkmark{}
        \end{enumerate}

    \item Rewrite the sum: \\
        $ \sum_{i=1}^{m} (2i-1) = \sum_{i=1}^{m-1} (2i-1) + (2m-1) $

    \item Find $\sum_{i=1}^{m-1} (2i-1)$ using the proposition: \\
        $ \sum_{i=1}^{n} (2i-1) = n^{2} $ \\
        $ \sum_{i=1}^{m-1} (2i-1) = (m-1)^{2} $ \\
        $ \sum_{i=1}^{m-1} (2i-1) = m^{2} - 2m + 1 $

    \item Plug $\sum_{i=1}^{m-1} (2i-1)$ into the sum from (2): \\
        $ \sum_{i=1}^{m} (2i-1) = \sum_{i=1}^{m-1} (2i-1) + (2m-1) $ \\
        $ \sum_{i=1}^{m} (2i-1) = m^{2} - 2m + 1 + 2m - 1 $
        $ \sum_{i=1}^{m} (2i-1) = m^{2} $

    \item This matches the original proposition, so we have proved it.
\end{enumerate}

}{}

\newpage


\solution{
\subsection*{Solution: Question 6}

\begin{enumerate}
    \item   Rewrite the sum: \\
        $\sum_{i=1}^{m}(3i+2) = \sum_{i=1}^{m-1}(3i+2) + 3m+2$ \\
        $ s_{m} = s_{m-1} + 3m + 2$
    \item   Find $s_{1}$: \\
        $s_{1} = \sum_{i=1}^{1}(3i+2) = 3(1)+2 = 5$
    \item   Write out together: \\
        $ s_{m} = s_{m-1} + 3m + 2 \tab s_{1} = 5$
\end{enumerate}
}{}


\solution{
\subsection*{Solution: Question 7}

\begin{enumerate}
    \item   Hypothesis: $n^{2} - 1$ is divisible by 5. \\
            Conclusion: $n$ is not divisible by 5.
    \item   Negation: \\
        $n^{2} - 1$ is divisible by 5 and $n$ IS divisible by 5.
    \item   Symbolically: \\
        $n^{2} - 1 = 5k$ \tab $n = 5j$
    \item   Equation: \\
        $(5j)^{2} - 1 = 5k$
    \item   Simplify until contradiction: \\
        $25j^{2} - 5k = 1 \tab \to 5(5j^{2} - k) = 1 \tab \to 5j^{2} - k = \frac{1}{5}$
\end{enumerate}
}{}


\solution{
\subsection*{Solution: Question 8}

\begin{enumerate}
    \item   \texttt{(246)}$_{10}$ = $2 \cdot 10^{2} + 4 \cdot 10^{1} + 6 \cdot 10^{0}$
    \item   \texttt{(0100 1001)}$_{2}$ = $1 \cdot 2^{6} + 1 \cdot 2^{3} + 1 \cdot 2^{0}$
    \item   \texttt{(F00D)}$_{16}$ = $15 \cdot 16^{3} + 13 \cdot 16^{0}$
\end{enumerate}
}{}

\newpage

\solution{
\subsection*{Solution: Question 9}

\begin{enumerate}
	\item	Convert \texttt{( 1111 0000 0000 1101 )}$_{2} $ to base-16 \\
        = F 0 0 D
	\item	Convert \texttt{( CAF3 )}$_{16} $ to base-2 \\
        = 1100 1010 1111 0011
\end{enumerate}
}{}

\solution{
\subsection*{Solution: Question 10}

\begin{enumerate}
	\item	Convert \texttt{( 75 )}$_{10} $ to base-2
        \begin{enumerate}
            \item   75 / 2 = 37 r 1
            \item   37 / 2 = 18 r 1
            \item   18 / 2 = 9 r 0
            \item   9 / 2 = 4 r 1
            \item   4 / 2 = 2 r 0
            \item   2 / 2 = 1 r 0
            \item   1 / 2 = 0 r 1
        \end{enumerate}

        = 0100 1011

    
	\item	Convert \texttt{( 0101 1010 )}$_{2} $ to base-10   
        \begin{enumerate}
            \item = $ 1 \cdot 2^{6} + 1 \cdot 2^{4} + 1 \cdot 2^{3} + 1 \cdot 2^{1} = 90$
        \end{enumerate}
\end{enumerate}
}{}

\end{document}
