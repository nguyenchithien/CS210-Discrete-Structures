\documentclass[a4paper,12pt,oneside]{book}
\usepackage[utf8]{inputenc}

\usepackage{rachwidgets}


\newcommand{\laClass}       {CS 210}
\newcommand{\laSemester}    {Spring 2018}
\newcommand{\laChapter}     {2.5}
\newcommand{\laType}        {Exercise}
\newcommand{\laPoints}      {5}
\newcommand{\laTitle}       {Contradiction}
\newcommand{\laDate}        {}
\setcounter{chapter}{2}
\setcounter{section}{5}
\addtocounter{section}{-1}
\newcounter{question}

\toggletrue{answerkey}

\input{BASE-HEADER}

\input{INSTRUCTIONS-EXERCISE}

% ASSIGNMENT ------------------------------------ %
    \section{Proof by contradiction}

    %------------------------------------------------------------------%
    \subsection{Review practice}

        % - QUESTION --------------------------------------------------%
        \stepcounter{question}
        \begin{questionNOGRADE}{\thequestion}
            For the statement, ``if $n \% 3 = 1$, then $n \% 9 \neq 5$",
            where $\%$ stands for ``modulus"...

            \begin{enumerate}
                \item[a.]   What is the hypothesis $p$?
                    \solution{$n \% 3 = 1$}{ ~\\~\\ }
                    
                \item[b.]   What is the conclusion $q$?
                    \solution{$n \% 9 \neq 5$}{ ~\\~\\ }
                
                \item[c.]   Using $\neg(p \to q) \equiv p \land \neg q$,
                    write out the negation of this implication in English.
                    \solution{
                    $n \% 3 = 1$ and $n \% 9 = 5$
                    }{ ~\\~\\ }
            \end{enumerate}

            This negation can be used to build the counter-example,
            which we will use for the proof by contradiction.
        \end{questionNOGRADE}

        ~\\~\\
        \begin{figure}[h]
            \centering
            \includegraphics[height=4cm]{images/cat.png}
            \caption*{There is extra space on the page}
        \end{figure}
    
    \newpage
    %------------------------------------------------------------------%
    \subsection{Proof by contradiction}

        \begin{framed}{Prove by contradiction: If $n^{2}$ is even, then $n$ is even}

            \subparagraph{Step 1: Identify the hypothesis and conclusion:}
            Here, our \textbf{hypothesis} is ``$n^{2}$ is even",
            and our \textbf{conclusion} is ``$n$ is even".

            \subparagraph{Step 2: Identify the negation:}
            The negation of an implication is \textbf{not} an implication.
            A counter-example of this would be if we have $p \land \neg q$,
            or ``$n^{2}$ is even and $n$ is odd".

            \subparagraph{Step 3: Build a counter-example:}
            Our counter-example is the scenario where the hypothesis
            is true and the conclusion is false... or in other words,
            the negation. 
            \textbf{Counter-example:} ``$n^{2}$ is even and $n$ is odd"

            \subparagraph{Step 4: Write the hypothesis \& conclusion symbolically:}
            (For our counter-example implication) \\
            $n^{2} = 2k$ (some even integer) \tab{}
            $n = 2j + 1$ (some odd integer)

            \subparagraph{Step 5: Write equation:}
            Using the statement, we are going to turn this into an equation.
            $n$ is odd, and $n^{2}$ is even, so if we square the odd $n$
            to get the even $n^{2}$, we would have...
            to get the even $n^{2}$, we would have...
            $(2j + 1)^{2} = 2k$

            \subparagraph{Step 6: Simplify until we have a contradiction:} ~\\
            $(2j + 1)^{2} = 2k$ \\
            \tab[0.5cm] $\Rightarrow$ \tab[0.5cm]
            $4j^{2} + 4j + 1 = 2k$ \\
            \tab[0.5cm] $\Rightarrow$ \tab[0.5cm]
            $1 = 2k - 4j^{2} - 4j$ \\
            \tab[0.5cm] $\Rightarrow$ \tab[0.5cm]
            $\frac{1}{2} = k - 2j^{2} - 2j$

            ~\\
            Since $k$ and $j$ are both integers, through the closure property
            of integers (+, -, and $\times$ results in an integer),
            we can show that $k - 2j^{2} - 2j$ results in something that
            is \textit{not an integer} -- this is a contradiction.
            It shows that our counter-example
            is \textbf{false}, and no counter-example can exist.
        \end{framed}

        \newpage
        % - QUESTION --------------------------------------------------%
        \stepcounter{question}
        \begin{questionNOGRADE}{\thequestion}

            Prove by contradiction: If $n^{2}$ is odd, then $n$ is odd.

            \subparagraph{Step 1: Identify the hypothesis and conclusion:}
                ~\\ \tab Hypothesis $p$:    \solution{$n^{2}$ is odd}{}
                ~\\ \tab Conclusion $q$:    \solution{$n$ is odd}{}
            
            \subparagraph{Step 2: Identify the negation (counter-example):} ~\\~\\
            	\tab[0.5cm]
                $p$:
                \solution{
                    $n^{2}$ is odd
                }{ \fitb[4cm] } AND \tab[0.5cm]
                $\neg q$:
                \solution{
                    $n$ is even
                }{ \fitb[4cm] }
            
                
            \subparagraph{Step 3: Write the hypothesis \& conclusion-negation symbolically:}
            ~\\
            (Make sure you use different variables for $n^{2}$ and $n$.)
                
                ~\\~\\ \tab ($p$) \tab $n^{2} = $ \solution{$2k+1$}{}
                ~\\~\\ \tab ($\neg q$) \tab $n = $ \solution{$2j$}{}
                ~\\~\\ 

            \subparagraph{Step 5: Write equation:}
            
            Set the equation for $n$-squared equal to the equation for $n^2$.
            
                \solution{
                $2k+1 = (2j)^{2}$
                }{ { ~\\ \raisebox{0pt}[1cm][0pt]{  } } }

            \subparagraph{Step 6: Simplify until we have a contradiction:} ~\\
                \solution{
                $2k + 1 = 4j^{2}$ \\ 
                \tab[0.5cm] $\Rightarrow$ \tab[0.5cm]
                $1 = 4j^{2} - 2k$ \\
                \tab[0.5cm] $\Rightarrow$ \tab[0.5cm]
                $\frac{1}{2} = 2j^{2} - k$
                }{ { ~\\ \raisebox{0pt}[4cm][0pt]{  } } }

            Result:
                \solution{ The result is a fraction, not an integer, therefore no counter-example exists. }{}
            
        \end{questionNOGRADE}

        \newpage

        % - QUESTION --------------------------------------------------%
        \stepcounter{question}
        \begin{questionNOGRADE}{\thequestion}

            Use proof by contradiction to explain why it is impossible
            for a number $n$ to be of the form $5k+3$ and of $5j+1$ for
            integers $k$ and $j$.

            \begin{hint}{Hint}
                $n = 5k+3$ is one statement, and $n = 5j+1$ is the other statement,
                so $5k+3 = 5j+1$ is your starting point.
            \end{hint}

            This isn't in an implication form, so we begin at Step 5...
                
            \subparagraph{Step 5: Write equation:} $5k+3 = 5j+1$

            \subparagraph{Step 6: Simplify until we have a contradiction:} ~\\
                \solution{
                $3 - 1 = 5j - 5k$ \\
                \tab[0.5cm] $\Rightarrow$ \tab[0.5cm]
                $2 = 5(j - k)$ \\
                \tab[0.5cm] $\Rightarrow$ \tab[0.5cm]
                $\frac{2}{5} = j - k$
                ~\\~\\
                }{ { ~\\ \raisebox{0pt}[6cm][0pt]{  } } }

            Result:
                \solution{ Since $j$ is an integer and $k$ is an integer,
                $j - k$ must also be an integer. Here, we get a fraction,
                so the counter-example is invalid.
                }{}
            
        \end{questionNOGRADE}

\input{BASE-FOOT}
