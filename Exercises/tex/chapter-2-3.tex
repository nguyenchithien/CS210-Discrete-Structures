\documentclass[../Template-Assignment.tex]{subfiles}
 
\newcommand{\laChapter}{2.3 Mathematical induction\ }
\setcounter{question}{0}

\begin{document}
    \newpage
    \section{Mathematical induction}

    %------------------------------------------------------------------%
    \subsection{Intro practice}

        % - QUESTION --------------------------------------------------%
        \stepcounter{question}
        \begin{questionNOGRADE}{\thequestion}

            Plug in the following values to the given predicate, and state
            whether it results in a true proposition, or a false one.
            $P(n)$ is ``$n^{2} + 1$ is prime".

            \begin{itemize}
                \item[a.] $P(1) = $ \solution{$1+1 = 2$, true}{}
                \item[b.] $P(3) = $ \solution{$9+1 = 10$, false}{}
                \item[c.] $P(9) = $ \solution{$81+1 = 81$, false}{}
            \end{itemize}
            
        \end{questionNOGRADE}

        \hrulefill

        % - QUESTION --------------------------------------------------%
        \stepcounter{question}
        \begin{questionNOGRADE}{\thequestion}

            For the recursive formula $a_{k} = a_{k-1} + 4$, $a_{1} = 1$,
            find the values for the following.

            \begin{itemize}
                \item[a.] $a_{1} = $    \solution{1}{}
                \item[b.] $a_{2} = $    \solution{$1 + 4 = 5$}{}
                \item[c.] $a_{3} = $    \solution{$5 + 4 = 9$}{}
                \item[d.] $a_{m-1} = $  (Anywhere you see $k$, plug in $m-1$.)
                \solution{$a_{m-2} + 4$}{}
            \end{itemize}
            
        \end{questionNOGRADE}

        \hrulefill

        % - QUESTION --------------------------------------------------%
        \stepcounter{question}
        \begin{questionNOGRADE}{\thequestion}

            For the closed formula $a_{n} = 4n - 3$,
            find the values for the following.

            \begin{itemize}
                \item[a.] $a_{1} = $    \solution{$4 - 3 = 1$}{}
                \item[b.] $a_{3} = $    \solution{$12 - 3 = 9$}{}
                \item[c.] $a_{5} = $    \solution{$20 - 3 = 27$}{}
                \item[d.] $a_{m-1} = $  (Anywhere you see $n$, plug in $m-1$.) \\
                \solution{$4(m-1) - 3 = 4m -4 -3 = 4m-7$}{}
            \end{itemize}
            
        \end{questionNOGRADE}

    \newpage

    \subsection{Recursive / Closed formula equivalence}

    \begin{intro}{Question 3a from the textbook}
        Show that the sequence defined by $a_{k} = a_{k-1} + 4; a_{1} = 1$ for $k \geq 2$
        is equivalently described by the closed formula $a_{n} = 4n - 3$.

        \paragraph{Step 1: Check $a_{1}$ for both formulas.} ~\\
            Recursive: $a_{1} = 1$ (provided); \tab{}
            Closed: $a_{1} = 4(1) - 3 = 1$  \tab \checkmark OK

        \paragraph{Step 2: Rewrite the recursive formula in terms of $m$:}
        $$ a_{m} = a_{m-1} + 4 $$

        \paragraph{Step 3: Find the equation for $a_{m-1}$ via the closed formula:}
        $$ a_{n} = 4n-3; \tab a_{m-1} = 4(m-1) - 3; \tab a_{m-1} = 4m-7 $$

        \paragraph{Step 4: Plug $a_{m-1}$ back into the recursive formula and simplify.}
        $$ a_{m} = a_{m-1} + 4; \tab \to \tab a_{m} = (4m-7) + 4 $$
        $$ a_{m} = 4m - 7 + 4; \tab a_{m} = 4m - 3 $$

        Looking back at the closed formula, $a_{n} = 4n - 3$, our result from
        Step 4 and this match, so we have proven that, for all values $k \geq 2$,
        the recursive formula and closed formula give the same sequence.
    \end{intro}

    Next you will solve these types of proofs, so follow the steps.
    We will go over several types of proofs this chapter, and they will
    be on the exam, so make sure you can tell the different \textit{types}
    of proofs apart!

    \newpage{}
    
        % - QUESTION --------------------------------------------------%
        \stepcounter{question}
        \begin{questionNOGRADE}{\thequestion}
            Show that the sequence defined by
            $a_{n} = a_{n-1} + 2; a_{1} = 5$ for $k \geq 2$ is equivalently
            described by the closed formula, $a_{n} = 2n+3$.

        \paragraph{Step 1: Check $a_{1}$ for both formulas.} ~\\
            \solution{
            Recursive: $a_{1} = 5$; \tab
            Closed: $a_{1} = 2(1) + 3 = 5$
            \tab \checkmark
            }{ { ~\\ \raisebox{0pt}[0.5cm][0pt]{  } } }
            
        \paragraph{Step 2: Rewrite the recursive formula in terms of $m$:} ~\\
            \solution{
            $a_{m} = 4 \cdot a_{m-1} + 2$
            }{ { ~\\ \raisebox{0pt}[0.5cm][0pt]{  } } }

        \paragraph{Step 3: Find the equation for $a_{m-1}$ via the closed formula:} ~\\
            \solution{
            $$ a_{m-1} = 2(m-1) + 3 \tab{}
            = 2m - 2 + 3 \tab{}
            = 2m + 1
            $$
            }{ { ~\\ \raisebox{0pt}[3cm][0pt]{  } } }

        \paragraph{Step 4: Plug $a_{m-1}$ back into the recursive formula and simplify.} ~\\
            \solution{
            $$ a_{m} = a_{m-1} + 2 $$

            $$ a_{m} = (2m+1) + 2 $$

            $$ a_{m} = 2m + 3 $$
            
            }{ { ~\\ \raisebox{0pt}[4cm][0pt]{  } } }
        \end{questionNOGRADE}

        
    \newpage{}

        \begin{hint}{Exponent rules}
            \textbf{Power rule:} $(a^{m})^{n} = a^{mn}$ \\
            \textbf{Negative exponent rule:} $a^{-n} = \frac{1}{a^{n}}$ \\
            \textbf{Product rule:} $a^{m} \cdot a^{n} = a^{m+n}$ \\
            \textbf{Quotient rule:} $\frac{a^{m}}{a^{n}} = a^{m-n}$
        \end{hint}
    
        % - QUESTION --------------------------------------------------%
        \stepcounter{question}
        \begin{questionNOGRADE}{\thequestion}
            Show that the sequence defined by
            $ a_{k} = 2 \cdot a_{k-1} + 1 ; a_{1} = 1 $
            for $k \geq 2$ is equivalently
            described by the closed formula,
            $ a_{n} = 2^{n} - 1 $

        \paragraph{Step 1: Check $a_{1}$ for both formulas.} ~\\
            \solution{
            Recursive:
            $ a_{1} = 1 $
            \tab
            Closed:
            $ a_{1} = 2^{1} - 1  = 1$
            \tab \checkmark
            }{ { ~\\ \raisebox{0pt}[0.5cm][0pt]{  } } }
            
        \paragraph{Step 2: Rewrite the recursive formula in terms of $m$:} ~\\
            \solution{
            $ a_{m} = 2 \cdot a_{m-1} + 1 $
            }{ { ~\\ \raisebox{0pt}[0.5cm][0pt]{  } } }

        \paragraph{Step 3: Find the equation for $a_{m-1}$ via the closed formula:} ~\\
            \solution{
            $$ a_{m-1} = 2^{m-1} - 1
            \tab{}
            = 2^{m} \cdot 2^{-1} - 1
            \tab{}
            = \frac{2^{m}}{2^{1}} - 1
            $$
            }{ { ~\\ \raisebox{0pt}[3cm][0pt]{  } } }

        \paragraph{Step 4: Plug $a_{m-1}$ back into the recursive formula and simplify.} ~\\
            \solution{
            $$ a_{m} = 2 \cdot a_{m-1} + 1 $$

            $$ a_{m} = 2^{1}(\frac{2^{m}}{2^{1}} - 1) + 1 $$

            $$ a_{m} = 2^{m} - 2 + 1 $$

            $$ a_{m} = 2^{m} - 1 $$
            
            }{ { ~\\ \raisebox{0pt}[4cm][0pt]{  } } }
        \end{questionNOGRADE}

    \newpage{}
    
        % - QUESTION --------------------------------------------------%
        \stepcounter{question}
        \begin{questionNOGRADE}{\thequestion}

            Show that the sequence defined by $b_{k} = 4 \cdot b_{k-1} + 3, b_{1} = 3$ for $k \geq 2$,
            is equivalently described by the closed formula $b_{n} = 2^{2n} - 1$.

        \paragraph{Step 1: Check $a_{1}$ for both formulas.} ~\\
            \solution{
            Recursive: $a_{1} = 3$; \tab
            Closed: $a_{1} = 2^{2\cdot1} - 1 = 4 - 1 = 3$ \tab \checkmark
            }{ { ~\\ \raisebox{0pt}[0.5cm][0pt]{  } } }
            
        \paragraph{Step 2: Rewrite the recursive formula in terms of $m$:} ~\\
            \solution{
            $a_{m} = 4 \cdot a_{m-1} + 3$
            }{ { ~\\ \raisebox{0pt}[0.5cm][0pt]{  } } }

        \paragraph{Step 3: Find the equation for $a_{m-1}$ via the closed formula:} ~\\
            \solution{
            $$ a_{m-1} = 2^{2(m-1)} - 1
            \tab = 2^{2m} \cdot 2^{-2} - 1
            \tab = \frac{ 2^{2m} }{2^{2}} - 1;$$
            }{ { ~\\ \raisebox{0pt}[3cm][0pt]{  } } }

        \paragraph{Step 4: Plug $a_{m-1}$ back into the recursive formula and simplify.} ~\\
            \solution{
            $$a_{m} = 4 a_{m-1} + 3$$

            $$a_{m} = 4 (\frac{2^{2m}}{2^{2}} - 1) + 3$$

            $$a_{m} = 2^{2}(\frac{2^{2m}}{2^{2}} - 1) + 3$$

            $$a_{m} = 2^{2m} - 4 + 3$$

            $$a_{m} = 2^{2m} - 1$$
            }{ { ~\\ \raisebox{0pt}[4cm][0pt]{  } } }

        \end{questionNOGRADE}

    \newpage


    \subsection{Summation / Closed formula equivalence}

    Another type of proof we will do is to show that, for some $n$ plugged
    into a sum and into a closed formula for that value.

    \begin{intro}{Question 8a from the textbook}

        Use induction to prove the proposition. As part of the proof,
        verify the statement for $n = 1$, $n = 2$, and $n = 3$.
        $ \sum_{i=1}^{n} (2i - 1) = n^{2} $        for each $n \geq 1$.

        \paragraph{Step 1: Show that the proposition is true for 1, 2, and 3. } ~\\

            \begin{tabular}{l | p{4cm} | p{4cm} }
                \textbf{ $i$ value } &
                \textbf{ $ \sum_{i=1}^{n} (2i - 1) $ } &
                \textbf{ $n^{2}$ }
                \\ \hline
                $i = 1$ &
                    $ \sum_{i=1}^{1} (2i - 1) $ &
                    $1^{2}$
                    \\
                    &
                    $= ( 2 \cdot 1 - 1 ) = 1$ &
                    $= 1$ \tab\checkmark
                \\ && \\
                $i = 2$ &
                    $ \sum_{i=1}^{2} (2i - 1) $&
                    $2^{2}$
                    \\
                    &
                    $ = (2 \cdot 1 - 1) + (2 \cdot 2 - 1)$ &
                    
                    \\
                    &
                    $= (1) + (3) = 4$ &
                    $ = 4$ \tab\checkmark
                \\ && \\
                $i = 3$ &
                    $ \sum_{i=1}^{3} (2i - 1) $&
                    $3^{2}$
                    \\
                    &
                    $ = 1 + 3 + (2 \cdot 3 - 1)$
                    &
                    \\
                    &
                    $ = 1 + 3 + 5 = 9 $ &
                    $ = 9 $  \tab\checkmark               
            \end{tabular}
            

        \paragraph{Step 2: Rewrite the summation as $ \sum_{i=1}^{m-1} (2i - 1) + (2m - 1) $:}

        ~\\~\\
        $ \sum_{i=1}^{m}(2i-1) = \sum_{i=1}^{m-1}(2i-1) + (2m - 1)$

        \paragraph{Step 3: Find an equation  for $\sum_{i=1}^{m-1}$ via the proposition: }

        ~\\~\\
        $ \sum_{i=1}^{n} (2i - 1) = n^{2} \tab ... \sum_{i=1}^{m-1}(2i-1) = (m-1)^{2} $
        
        \paragraph{Step 4: Plug $\sum_{i=1}^{m-1}$ into the summation from step (2):}

        ~\\~\\
        $ \sum_{i=1}^{m}(2i-1) = \sum_{i=1}^{n-1}(2i-1) + (2m - 1) \tab = (m-1)^{2} + (2m - 1)$ \\
        $ \sum_{i=1}^{m}(2i-1) =  m^{2} - 2m + 1 + 2m - 1 $ \\
        $ \sum_{i=1}^{m}(2i-1) =  m^{2} $ \tab \checkmark{}

        ~\\
        We get the same form as the original proposition, proving our statement.
    \end{intro}

    \newpage

        % - QUESTION --------------------------------------------------%
        \stepcounter{question}
        \begin{questionNOGRADE}{\thequestion}

            Use induction to prove
            $$ \sum_{i=1}^{n} (2i+4) = n^{2} + 5n$$
            for each $n \geq 1$.

        \paragraph{Step 1: Show that the proposition is true for 1, 2, and 3. } ~\\

            \begin{tabular}{l | p{4cm} | p{4cm} }
                \textbf{ $i$ value } &
                \textbf{ $ \sum_{i=1}^{n} (2i+4) $ } &
                \textbf{ $n^{2} + 5n$ }
                \\ \hline
                $i = 1$ &
                    \solution{ $2(1)+4 = 6$ }{} &
                    \solution{ $1^{2} + 5(1) = 6$ }{} 

                \\ && \\
                $i = 2$ &
                    \solution{ $6 + 2(2)+4 = 14$ }{} &
                    \solution{ $2^{2} + 5(2) = 14$ }{} 

                \\ && \\
                $i = 3$ &
                    \solution{ $14 + 2(3) + 4 = 24$ }{} &
                    \solution{ $3^{2} + 5(3) = 9+15 = 24$ }{} 
            \end{tabular}

        \paragraph{Step 2: Rewrite the summation:} ~\\
            As the sum from $i = 1$ to $m-1$, plus the final term $(2m+4)$.
            ~\\
            \solution{
            $ \sum_{i=1}^{m} (2i+4) = \sum_{i=1}^{m-1} (2i+4) + (2m+4)$
            }{ { ~\\ \raisebox{0pt}[1cm][0pt]{  } } }

        \paragraph{Step 3: Find an equation  for $\sum_{i=1}^{m-1}$ via the proposition: } 

            ~\\
            \solution{
            $ \sum_{i=1}^{m-1} (2i+4) = (m-1)^{2} + 5(m-1) $ ~\\
            $ \sum_{i=1}^{m-1} (2i+4) = m^{2} - 2m + 1 + 5m - 5 $ ~\\
            $ \sum_{i=1}^{m-1} (2i+4) = m^{2} + 3m - 4 $
            }{ { ~\\ \raisebox{0pt}[2cm][0pt]{  } } }


        \paragraph{Step 4: Plug $\sum_{i=1}^{m-1}$ into the summation from step 2 and simplify:}

            ~\\
            \solution{
            $ \sum_{i=1}^{m} (2i+4) = (m^{2} + 3m - 4) + (2m+4) $ ~\\
            $ \sum_{i=1}^{m} (2i+4) = m^{2} + 5m $ ~\\
            This matches the original proposition.
            }{ { ~\\ \raisebox{0pt}[4cm][0pt]{  } } }
        \end{questionNOGRADE}

        \newpage

        % - QUESTION --------------------------------------------------%
        \stepcounter{question}
        \begin{questionNOGRADE}{\thequestion}

            Use induction to prove that for every positive integer $n$,
            $$ \sum_{i=1}^{n} i = \frac{n(n+1)}{2}$$

        \paragraph{Step 1: Show that the proposition is true for 1, 2, and 3. } ~\\

            \begin{tabular}{l | p{4cm} | p{4cm} }
                \textbf{ $i$ value } &
                \textbf{ $ \sum_{i=1}^{n} i $ } &
                \textbf{ $ \frac{n(n+1)}{2} $ }
                \\ \hline
                $i = 1$ &
                    \solution{ $1$ }{} &
                    \solution{ $\frac{1(2)}{2} = 1$ }{} 

                \\ && \\
                $i = 2$ &
                    \solution{ $1 + 2 = 3$ }{} &
                    \solution{ $ \frac{2(2+1)}{2} = 3 $ }{} 

                \\ && \\
                $i = 3$ &
                    \solution{ $ 1 + 2 + 3 = 6$ }{} &
                    \solution{ $ \frac{3(3+1)}{2} = 6 $ }{} 
            \end{tabular}

        \paragraph{Step 2: Rewrite the summation:} ~\\
            As the sum from $i = 1$ to $m-1$, plus the final term $m$.
            ~\\
            \solution{
            $ \sum_{i=1}^{m} i = \sum_{i=1}^{m-1} (i) + m$
            }{ { ~\\ \raisebox{0pt}[1cm][0pt]{  } } }

        \paragraph{Step 3: Find an equation  for $\sum_{i=1}^{m-1}$ via the proposition: } 

            ~\\
            \solution{
            $ \sum_{i=1}^{m-1} i = \frac{(m-1)(m)}{2} $ ~\\
            $ \sum_{i=1}^{m-1} i = \frac{m^{2} - m}{2} $
            }{ { ~\\ \raisebox{0pt}[2cm][0pt]{  } } }


        \paragraph{Step 4: Plug $\sum_{i=1}^{m-1}$ into the summation from step 2 and simplify:}

            ~\\
            \solution{
            $ \sum_{i=1}^{m} i = \frac{m^{2} - m}{2} + m $ ~\\
            $ \sum_{i=1}^{m} i = \frac{m^{2} - m}{2} + \frac{2m}{2} $ ~\\
            $ \sum_{i=1}^{m} i = \frac{m^{2} + m}{2} $ ~\\
            $ \sum_{i=1}^{m} i = \frac{m(m + 1)}{2} $ ~\\
            This matches the original proposition.
            }{ { ~\\ \raisebox{0pt}[4cm][0pt]{  } } }
        \end{questionNOGRADE}

\end{document}

