\documentclass[a4paper,12pt]{book}
\usepackage[utf8]{inputenc}

\usepackage{rachwidgets}


\newcommand{\laClass}       {CS 210}
\newcommand{\laSemester}    {Spring 2018}
\newcommand{\laChapter}     {2.3}
\newcommand{\laType}        {Exercise}
\newcommand{\laPoints}      {5}
\newcommand{\laTitle}       {Mathematical Induction}
\newcommand{\laDate}        {}
\setcounter{chapter}{2}
\setcounter{section}{3}
\addtocounter{section}{-1}
\newcounter{question}

\toggletrue{answerkey}


\title{}
\author{Rachel Singh}
\date{\today}

\pagestyle{fancy}
\fancyhf{}

\lhead{\laClass, \laSemester, \laDate}

\chead{}

\rhead{\laChapter\ \laType\ \iftoggle{answerkey}{ KEY }{}}

\rfoot{\thepage\ of \pageref{LastPage}}

\lfoot{\scriptsize By Rachel Singh, last updated \today}

\renewcommand{\headrulewidth}{2pt}
\renewcommand{\footrulewidth}{1pt}

\begin{document}




\notonkey{

\footnotesize
~\\ 
\textbf{\laChapter\ \laType: } In-class exercises are meant to introduce you to a new topic
and provide some practice with the new topic. Work in a team of up to 4 people to complete this exercise.
You can work simultaneously on the problems, or work separate and then check your answers with each other.
You can take the exercise home, score will be based on the in-class quiz the following class period.
\textbf{Work out problems on your own paper} - this document just has examples and questions.

\hrulefill
\normalsize 

}{
\begin{center}
    \Large
    \textbf{Answer Key}
\end{center}
}


% ASSIGNMENT ------------------------------------ %

    \begin{enumerate}
        \item   
            \begin{itemize}
                \item[a.] $P(1) = $ {$1+1 = 2$, true}{}
                \item[b.] $P(3) = $ {$9+1 = 10$, false}{}
                \item[c.] $P(9) = $ {$81+1 = 81$, false}{}
            \end{itemize}

        \item
            \begin{itemize}
                \item[a.] $a_{1} = $    {1}{}
                \item[b.] $a_{2} = $    {$1 + 4 = 5$}{}
                \item[c.] $a_{3} = $    {$5 + 4 = 9$}{}
                \item[d.] $a_{m-1} = $  (Anywhere you see $k$, plug in $m-1$.)
                {$a_{m-2} + 4$}{}
            \end{itemize}

        \item
            \begin{itemize}
                \item[a.] $a_{1} = $    {$4 - 3 = 1$}{}
                \item[b.] $a_{3} = $    {$12 - 3 = 9$}{}
                \item[c.] $a_{5} = $    {$20 - 3 = 27$}{}
                \item[d.] $a_{m-1} = $  (Anywhere you see $n$, plug in $m-1$.) \\
                {$4(m-1) - 3 = 4m -4 -3 = 4m-7$}{}
            \end{itemize}

        \item
            \textbf{Step 1:}
            
            Recursive: $a_{1} = 5$; \tab
            Closed: $a_{1} = 2(1) + 3 = 5$
            \tab \checkmark
            
            \textbf{Step 2:}
            
            $a_{m} = 4 \cdot a_{m-1} + 2$

            \textbf{Step 3:}
            
            $ a_{m-1} = 2(m-1) + 3 \tab{}
            = 2m - 2 + 3 \tab{}
            = 2m + 1
            $

            \textbf{Step 4:}
            
            $ a_{m} = a_{m-1} + 2 $

            $ a_{m} = (2m+1) + 2 $

            $ a_{m} = 2m + 3 $

        \item
            \textbf{Step 1:}
            
            Recursive:
            $ a_{1} = 1 $
            \tab
            Closed:
            $ a_{1} = 2^{1} - 1  = 1$
            \tab \checkmark
            
            \textbf{Step 2:}
            
            $ a_{m} = 2 \cdot a_{m-1} + 1 $

            \textbf{Step 3:}
            
            $ a_{m-1} = 2^{m-1} - 1
            \tab{}
            = 2^{m} \cdot 2^{-1} - 1
            \tab{}
            = \frac{2^{m}}{2^{1}} - 1
            $

            \textbf{Step 4:}
            
            $ a_{m} = 2 \cdot a_{m-1} + 1 $
            
            $ a_{m} = 2^{1}(\frac{2^{m}}{2^{1}} - 1) + 1 $

            $ a_{m} = 2^{m} - 2 + 1 $

            $ a_{m} = 2^{m} - 1 $

        \item
            \textbf{Step 1:}
            
            Recursive: $b_{1} = 3$; \tab
            Closed: $b_{1} = 2^{2\cdot1} - 1 = 4 - 1 = 3$ \tab \checkmark
            
            \textbf{Step 2:}
            
            $b_{m} = 4 \cdot b_{m-1} + 3$
            
            \textbf{Step 3:}
            
            $ b_{m-1} = 2^{2(m-1)} - 1
            \tab = 2^{2m} \cdot 2^{-2} - 1
            \tab = \frac{ 2^{2m} }{2^{2}} - 1;$
            
            \textbf{Step 4:}
            
            $b_{m} = 4 a_{m-1} + 3$

            $b_{m} = 4 (\frac{2^{2m}}{2^{2}} - 1) + 3$

            $b_{m} = 2^{2}(\frac{2^{2m}}{2^{2}} - 1) + 3$

            $b_{m} = 2^{2m} - 4 + 3$

            $b_{m} = 2^{2m} - 1$

        \item
            \textbf{Step 1:}
            
                \begin{tabular}{l | p{4cm} | p{4cm} }
                    \textbf{ $i$ value } &
                    \textbf{ $ \sum_{i=1}^{n} (2i+4) $ } &
                    \textbf{ $n^{2} + 5n$ }
                    \\ \hline
                    $i = 1$ &
                        { $2(1)+4 = 6$ }{} &
                        { $1^{2} + 5(1) = 6$ }{} 

                    \\ 
                    $i = 2$ &
                        { $6 + 2(2)+4 = 14$ }{} &
                        { $2^{2} + 5(2) = 14$ }{} 

                    \\ 
                    $i = 3$ &
                        { $14 + 2(3) + 4 = 24$ }{} &
                        { $3^{2} + 5(3) = 9+15 = 24$ }{} 
                \end{tabular}
            
            \textbf{Step 2:}
            
                $ \sum_{i=1}^{m} (2i+4) = \sum_{i=1}^{m-1} (2i+4) + (2m+4)$

            \textbf{Step 3:}
            
                $ \sum_{i=1}^{m-1} (2i+4) = (m-1)^{2} + 5(m-1) $ ~\\
                $ \sum_{i=1}^{m-1} (2i+4) = m^{2} - 2m + 1 + 5m - 5 $ ~\\
                $ \sum_{i=1}^{m-1} (2i+4) = m^{2} + 3m - 4 $
                
            \textbf{Step 4:}
            
            $ \sum_{i=1}^{m} (2i+4) = (m^{2} + 3m - 4) + (2m+4) $ ~\\
            $ \sum_{i=1}^{m} (2i+4) = m^{2} + 5m $ ~\\
            This matches the original proposition.

        \item
            \textbf{Step 1:}
            
                \begin{tabular}{l | p{4cm} | p{4cm} }
                    \textbf{ $i$ value } &
                    \textbf{ $ \sum_{i=1}^{n} i $ } &
                    \textbf{ $ \frac{n(n+1)}{2} $ }
                    \\ \hline
                    $i = 1$ &
                        { $1$ }{} &
                        { $\frac{1(2)}{2} = 1$ }{} 

                    \\ 
                    $i = 2$ &
                        { $1 + 2 = 3$ }{} &
                        { $ \frac{2(2+1)}{2} = 3 $ }{} 

                    \\ 
                    $i = 3$ &
                        { $ 1 + 2 + 3 = 6$ }{} &
                        { $ \frac{3(3+1)}{2} = 6 $ }{} 
                \end{tabular}
            
            \textbf{Step 2:}
            
                $ \sum_{i=1}^{m} i = \sum_{i=1}^{m-1} (i) + m$

            \textbf{Step 3:}
            
                $ \sum_{i=1}^{m-1} i = \frac{(m-1)(m)}{2} $ ~\\
                $ \sum_{i=1}^{m-1} i = \frac{m^{2} - m}{2} $
                
            \textbf{Step 4:}
            
                $ \sum_{i=1}^{m} i = \frac{m^{2} - m}{2} + m $ ~\\
                $ \sum_{i=1}^{m} i = \frac{m^{2} - m}{2} + \frac{2m}{2} $ ~\\
                $ \sum_{i=1}^{m} i = \frac{m^{2} + m}{2} $ ~\\
                $ \sum_{i=1}^{m} i = \frac{m(m + 1)}{2} $ ~\\
                This matches the original proposition.
    \end{enumerate}
   
    


\end{document}

