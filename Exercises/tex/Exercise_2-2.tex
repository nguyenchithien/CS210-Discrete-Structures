\documentclass[a4paper,12pt,oneside]{book}
\usepackage[utf8]{inputenc}

\usepackage{rachwidgets}


\newcommand{\laClass}       {CS 210}
\newcommand{\laSemester}    {Spring 2018}
\newcommand{\laChapter}     {2.2}
\newcommand{\laType}        {Exercise}
\newcommand{\laPoints}      {5}
\newcommand{\laTitle}       {Proofs About Numbers}
\newcommand{\laDate}        {}
\setcounter{chapter}{2}
\setcounter{section}{2}
\addtocounter{section}{-1}
\newcounter{question}

\toggletrue{answerkey}
\togglefalse{answerkey}

\input{BASE-HEADER}

\input{INSTRUCTIONS-EXERCISE}

% ASSIGNMENT ------------------------------------ %
    %------------------------------------------------------------------%
    \section*{1. More definitions}

        \begin{intro}{Modulus}
            ``In computing, the modulo operation finds the remainder after
            division of one number by another (sometimes called modulus).

            Given two positive numbers, a (the dividend) and n (the divisor),
            a modulo n (abbreviated as a mod n) is the remainder of the Euclidean division of a by n."
            \footnote{From https://en.wikipedia.org/wiki/Modulo\_operation}

            \begin{center}
                \includegraphics[height=3cm]{images/ch2-1-division.png}

                $9$ mod $2 = 1$
            \end{center}

            If we're dividing $a$ by $b$, the result is a quotient $q$.
            If we're calculating $a$ mod $b$, the result is the remainder $r$.

            \begin{center}
                We can also write this out as: \\
                $ a = b \cdot q + r $,
                where $0 \leq r < b$, and $q$ and $r$ are the only two integers
                that will satisfy the equation.
            \end{center}
        \end{intro}

        \begin{intro}{Rational numbers}
            In mathematics, a rational number is any number that can be
            expressed as the quotient or fraction p/q of two integers,
            a numerator p and a non-zero denominator q.[1] Since q may be equal to 1,
            every integer is a rational number.
            \footnote{From https://en.wikipedia.org/wiki/Rational\_number}

            The set of rational numbers is written as $\mathbb{Q}$.
        \end{intro}

        % - QUESTION --------------------------------------------------%
        \stepcounter{question}
        \begin{questionNOGRADE}{\thequestion}
            Solve the following modulus problems.

            \paragraph{Example:} Solve 13 mod 5 \\
                \begin{answer} 13 / 5 = 2, \tab 13 mod 5 = 3, \tab $13 = 5 \cdot 2 + 3$ \end{answer}

            \begin{enumerate}
                \item[a.] 9 mod 7

                \item[b.] 5 mod 2

                \item[c.] 15 mod 3

                \item[d.] -7 mod 2

            \end{enumerate}
        \end{questionNOGRADE}

        \hrulefill

        % - QUESTION --------------------------------------------------%
        \stepcounter{question}
        \begin{questionNOGRADE}{\thequestion}
            Prove the following propositions:

            \begin{enumerate}
                \item[a.] If $a$ divides $b$ and $a$ divides $c$, then $a$ divides $b + c$.
                    \footnote{From Discrete Mathematics by Ensley and Crawley}

                    \begin{hint}{\ }
                        Start with: $b = ak$ and $c = aj$ and calculate $b+c$.
                    \end{hint}

                \item[b.] If $a$ divides $b$ and $c$ divides $d$, then $ac$ divides $bd$.
                    \footnote{From Discrete Mathematics by Ensley and Crawley}

                    \begin{hint}{\ }
                        Start with: $b = ak, d = cj$ and calculate $bd$.
                    \end{hint}

            \end{enumerate}
        \end{questionNOGRADE}


\input{BASE-FOOT}
