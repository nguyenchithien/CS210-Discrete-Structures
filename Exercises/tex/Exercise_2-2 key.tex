\documentclass[a4paper,12pt,oneside]{book}
\usepackage[utf8]{inputenc}

\usepackage{rachwidgets}


\newcommand{\laClass}       {CS 210}
\newcommand{\laSemester}    {Spring 2018}
\newcommand{\laChapter}     {2.2}
\newcommand{\laType}        {Exercise}
\newcommand{\laPoints}      {5}
\newcommand{\laTitle}       {Proofs About Numbers}
\newcommand{\laDate}        {}
\setcounter{chapter}{2}
\setcounter{section}{2}
\addtocounter{section}{-1}
\newcounter{question}

\toggletrue{answerkey}

\input{BASE-HEADER}

\input{INSTRUCTIONS-EXERCISE}

% ASSIGNMENT ------------------------------------ %

    \begin{enumerate}
        \item   
            \begin{enumerate}
                \item[a.] 9 mod 7 \\ 
                    9 mod 7 = 2; \tab $9 = 7 \cdot 1 + 2$

                \item[b.] 5 mod 2 \\ 
                    5 mod 2 = 1; \tab $5 = 2 \cdot 2 + 1$

                \item[c.] 15 mod 3 \\ 
                    15 mod 3 = 0; \tab $15 = 3 \cdot 5 + 0$

                \item[d.] -7 mod 2 \\ 
                    -7 mod 2 = 1; \tab $-7 = 2 \cdot -4 + 1$
            \end{enumerate}

        \item
            \begin{enumerate}
                \item[a.] If $a$ divides $b$ and $a$ divides $c$, then $a$ divides $b + c$.
                    \footnote{From Discrete Mathematics by Ensley and Crawley} \\
                        $b = ak, c = aj \tab => b + c = ak + aj \tab => a(k+j) $

                \item[b.] If $a$ divides $b$ and $c$ divides $d$, then $ac$ divides $bd$.
                    \footnote{From Discrete Mathematics by Ensley and Crawley} \\
                    $b = ak, d = cj \tab => bd = (ak)(cj) \tab => (ac)(kj) $
            \end{enumerate}
    \end{enumerate}



\input{BASE-FOOT}
