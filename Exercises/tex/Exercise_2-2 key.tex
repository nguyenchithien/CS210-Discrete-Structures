\documentclass[a4paper,12pt]{book}
\usepackage[utf8]{inputenc}

\usepackage{rachwidgets}


\newcommand{\laClass}       {CS 210}
\newcommand{\laSemester}    {Spring 2018}
\newcommand{\laChapter}     {2.2}
\newcommand{\laType}        {Exercise}
\newcommand{\laPoints}      {5}
\newcommand{\laTitle}       {Proofs About Numbers}
\newcommand{\laDate}        {}
\setcounter{chapter}{2}
\setcounter{section}{2}
\addtocounter{section}{-1}
\newcounter{question}

\toggletrue{answerkey}


\title{}
\author{Rachel Singh}
\date{\today}

\pagestyle{fancy}
\fancyhf{}

\lhead{\laClass, \laSemester, \laDate}

\chead{}

\rhead{\laChapter\ \laType\ \iftoggle{answerkey}{ KEY }{}}

\rfoot{\thepage\ of \pageref{LastPage}}

\lfoot{\scriptsize By Rachel Singh, last updated \today}

\renewcommand{\headrulewidth}{2pt}
\renewcommand{\footrulewidth}{1pt}

\begin{document}




\notonkey{

\footnotesize
~\\ 
\textbf{\laChapter\ \laType: } In-class exercises are meant to introduce you to a new topic
and provide some practice with the new topic. Work in a team of up to 4 people to complete this exercise.
You can work simultaneously on the problems, or work separate and then check your answers with each other.
You can take the exercise home, score will be based on the in-class quiz the following class period.
\textbf{Work out problems on your own paper} - this document just has examples and questions.

\hrulefill
\normalsize 

}{
\begin{center}
    \Large
    \textbf{Answer Key}
\end{center}
}


% ASSIGNMENT ------------------------------------ %

    \begin{enumerate}
        \item   
            \begin{enumerate}
                \item[a.] 9 mod 7 \\ 
                    9 mod 7 = 2; \tab $9 = 7 \cdot 1 + 2$

                \item[b.] 5 mod 2 \\ 
                    5 mod 2 = 1; \tab $5 = 2 \cdot 2 + 1$

                \item[c.] 15 mod 3 \\ 
                    15 mod 3 = 0; \tab $15 = 3 \cdot 5 + 0$

                \item[d.] -7 mod 2 \\ 
                    -7 mod 2 = 1; \tab $-7 = 2 \cdot -4 + 1$
            \end{enumerate}

        \item
            \begin{enumerate}
                \item[a.] If $a$ divides $b$ and $a$ divides $c$, then $a$ divides $b + c$.
                    \footnote{From Discrete Mathematics by Ensley and Crawley} \\
                        $b = ak, c = aj \tab => b + c = ak + aj \tab => a(k+j) $

                \item[b.] If $a$ divides $b$ and $c$ divides $d$, then $ac$ divides $bd$.
                    \footnote{From Discrete Mathematics by Ensley and Crawley} \\
                    $b = ak, d = cj \tab => bd = (ak)(cj) \tab => (ac)(kj) $
            \end{enumerate}
    \end{enumerate}





\end{document}

