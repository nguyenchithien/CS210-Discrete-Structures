\documentclass[a4paper,12pt]{book}
\usepackage[utf8]{inputenc}

\usepackage{rachwidgets}


\newcommand{\laClass}       {CS 210}
\newcommand{\laSemester}    {Spring 2018}
\newcommand{\laChapter}     {1.2}
\newcommand{\laType}        {Exercise}
\newcommand{\laPoints}      {5}
\newcommand{\laTitle}       {Number Puzzles and Sequences}
\newcommand{\laDate}        {Week 1}
\setcounter{chapter}{5}
\setcounter{section}{1}
\addtocounter{section}{-1}
\newcounter{question}

\toggletrue{answerkey}
\togglefalse{answerkey}


\title{}
\author{Rachel Singh}
\date{\today}

\pagestyle{fancy}
\fancyhf{}

\lhead{\laClass, \laSemester, \laDate}

\chead{}

\rhead{\laChapter\ \laType\ \iftoggle{answerkey}{ KEY }{}}

\rfoot{\thepage\ of \pageref{LastPage}}

\lfoot{\scriptsize By Rachel Singh, last updated \today}

\renewcommand{\headrulewidth}{2pt}
\renewcommand{\footrulewidth}{1pt}

\begin{document}




\notonkey{

\footnotesize
~\\ 
\textbf{\laChapter\ \laType: } In-class exercises are meant to introduce you to a new topic
and provide some practice with the new topic. Work in a team of up to 4 people to complete this exercise.
You can work simultaneously on the problems, or work separate and then check your answers with each other.
You can take the exercise home, score will be based on the in-class quiz the following class period.
\textbf{Work out problems on your own paper} - this document just has examples and questions.

\hrulefill
\normalsize 

}{
\begin{center}
    \Large
    \textbf{Answer Key}
\end{center}
}


\notonkey{
    \section{\laTitle}

    %------------------------------------------------------------------%
    \subsection{Number sequences}

    \begin{intro}{\ }
        For this section, we are analyzing sequences of numbers in order to build
        \textit{closed} and/or \textit{recursive} formulas to describe them.

        It can be a bit challenging at first to figure out the equation based
        on a list of numbers, so make sure to take note of some techniques for analyzing
        these sequences!

        Let's start off simple...
    \end{intro}

    % - QUESTION --------------------------------------------------%
    \stepcounter{question}
    \begin{questionNOGRADE}{\thequestion}
        For the given sequence of numbers: \tab 2, 4, 6, 8, 10

        \begin{enumerate}
            \item[a.] What is the next number in the sequence?  If you can tell just by looking at it, how can you tell?

            \item[b.] If we assign numbers to each of these...

            \begin{center}
                \begin{tabular}{ | c | c | c | c | c | }
                    \hline{}
                    Item 1 & Item 2 & Item 3 & Item 4 & Item 5 \\ \hline
                    2 & 4 & 6 & 8 & 10 \\ \hline
                \end{tabular}
            \end{center}

            ...How can we come up with some formula to associate the item \# to the value?

            \item[c.] If we're describing \textbf{Item 2} in terms of \textbf{Item 1}...
                Item 2 = Item 1 + ?

            \item[d.] If we're describing \textbf{Item 3} in terms of \textbf{Item 2}...
                Item 3 = ?

            \item[e.] If we want to generalize this and describe any item $n$ in terms of
                the previous item, $n-1$...
                Item $n$ = ?
        \end{enumerate}
    \end{questionNOGRADE}

    \newpage

    %------------------------------------------------------------------%
    \subsection{Sequences}
    
    \begin{intro}{\ }
        \paragraph{Definition: Recursive formula} (aka recurrence relation) \\
        In mathematics, a recurrence relation is an equation that recursively defines
        a sequence [...] of values, once one or more initial terms are given:
        each further term of the sequence [...] is defined as a function
        of the preceding terms.
        \footnote{From https://en.wikipedia.org/wiki/Recurrence\_relation}

        \paragraph{Definition: Closed formula} ~\\
        A closed formula for a sequence is a formula where each term is described
        only in relation to its position in the list.
        \footnote{From Discrete Mathematics Mathematical Reasoning and Proof with Puzzles, Patterns, and Games by Douglas E Ensley}

        \paragraph{Definition: Sequence notation}
        Sequence notation is where we have some sequence, $a$,
        and $a_{n}$ denotes the element at position $n$. On a computer,
        the subscript may be written as \texttt{a[n]}.
    \end{intro}


    % - QUESTION --------------------------------------------------%
    \stepcounter{question}
    \begin{questionNOGRADE}{\thequestion}
        Write out the first 5 elements of the following equations: \\
        \begin{itemize}
            \item[a.]   The closed formula $a_{n} = n+1$
            \item[b.]   The closed formula $a_{n} = 2n+1$
            \item[c.]   The recursive formula $a_{1} = 1, a_{n} = a_{n-1} + 2$
            \item[d.]   The resursive formula $a_{1} = 2, a_{n} = 2 a_{n-1} + 1$
        \end{itemize}
    \end{questionNOGRADE}

    \newpage
    
    \begin{hint}{Tips for finding equations}
        If it isn't immediately obvious what a sequence's function is, here are a few tips:
        
        \begin{itemize}
            \item Write out each element with its position, like
            $a_{1} = 2$, $a_{2} = 5$, $a_{3} = 10$, etc. This helps
            with trying to find a pattern between the \textbf{index} (position)
            and the \textbf{element} (value).
            
            \item Compare the difference between each element, like
            $5 - 2 = 3$, $10 - 5 = 5$, $17 - 10 = 7$. Can you find a pattern
            in the difference between the elements?
            
            \item Compare the difference between the \textit{differences}.
            Above, we can see that the difference \textit{increases}
            by 2 between each element.
        \end{itemize}
    \end{hint}

    % - QUESTION --------------------------------------------------%
    \stepcounter{question}
    \begin{questionNOGRADE}{\thequestion}
        Figure out the \textbf{closed formula} for the following sequences. \\
        For these sequences, \textit{n} will not be multiplied
        by anything, but will have something added to it.

        \begin{center}
            \begin{tabular}{p{6cm} p{6cm}}
                a. 3, 4, 5, 6, 7 &
                b. 6, 7, 8, 9, 10
            \end{tabular} 
        \end{center} 
    \end{questionNOGRADE}

    \hrulefill

    % - QUESTION --------------------------------------------------%
    \stepcounter{question}
    \begin{questionNOGRADE}{\thequestion}
        Figure out the \textbf{closed formula} for the following sequences. \\
        For these sequences, \textit{n} will have something multiplied to it.

        \begin{center}
            \begin{tabular}{p{6cm} p{6cm}}
                a. 2, 4, 6, 8, 10 &
                b. 3, 6, 9, 12, 15 \\
                c. 5, 10, 15, 20, 25 &
                d. 1, 4, 9, 16, 25
            \end{tabular}   
        \end{center}    
    \end{questionNOGRADE}

    \hrulefill

    % - QUESTION --------------------------------------------------%
    \stepcounter{question}
    \begin{questionNOGRADE}{\thequestion}
        Figure out the \textbf{closed formula} for the following sequences. \\
        For these sequences, \textit{n} will have something multipled to it
        and added to (or subtracted from) the product.

        \begin{center}
            \begin{tabular}{p{6cm} p{6cm}}
                a. 1, 3, 5, 7, 9 &
                b. 4, 7, 10, 13, 16 \\
                c. 7, 12, 17, 22, 27 &
                d. 2, 5, 10, 17, 26
            \end{tabular}
        \end{center}
    \end{questionNOGRADE}

    \newpage


    % - QUESTION --------------------------------------------------%
    \stepcounter{question}
    \begin{questionNOGRADE}{\thequestion}
        Figure out the \textbf{recursive formula} for the following sequences. \\
        For these sequences, $a_{n-1}$ will not be multiplied by anything,
        but will have something added to it.
        Be sure to specify $a_{1}$ first. It will be the first number in the sequence.

        \begin{center}
            \begin{tabular}{p{6cm} p{6cm}}
                a. 1, 3, 5, 7, 9     \tab \textit{($a_{1} = 1$)} &
                b. 1, 5, 9, 13, 17 \\
                c. 2, 4, 6, 8, 10    \tab \textit{($a_{1} = 2$)} &
                d. 2, 6, 10, 14, 18
            \end{tabular}  
        \end{center}
    \end{questionNOGRADE}

    \hrulefill

    % - QUESTION --------------------------------------------------%
    \stepcounter{question}
    \begin{questionNOGRADE}{\thequestion}
        Figure out the \textbf{recursive formula} for the following sequences. \\
        For these sequences, $a_{n-1}$ will have something multiplied to
        it, but nothing added to it.
        Be sure to specify $a_{1}$ first.

        \begin{center}
            \begin{tabular}{p{6cm} p{6cm}}
                a. 2, 4, 8, 16, 32 &
                b. 1, 3, 9, 27, 81 \\
                c. 3, 6, 12, 24, 48 &
                d. 2, 4, 16, 256, 65536
            \end{tabular}  
        \end{center}     
    \end{questionNOGRADE}

    \hrulefill

    % - QUESTION --------------------------------------------------%
    \stepcounter{question}
    \begin{questionNOGRADE}{\thequestion}
        Figure out the \textbf{recursive formula} for the following sequences. \\
        For these sequences, $a_{n-1}$ will have something multiplied
        to it and added to it.
        Be sure to specify $a_{1}$ first.

        \begin{center}
            \begin{tabular}{p{6cm} p{6cm}}
                a. 1, 3, 7, 15, 31 &
                b. 2, 5, 11, 23, 47 \\
                c. 1, 5, 17, 53, 161 &
                d. 1, 4, 10, 22, 46
            \end{tabular}
        \end{center}
    \end{questionNOGRADE}

    \newpage
    
    \subsection{Summations}
    \begin{intro}{\ }
        For a sequence of numbers (denoted $a_{k}$, where $k >= 1$,
        we can use the notation
        $$\sum_{k=1}^{n} a_{k}$$
        to denote the sum of the first $n$ terms of the sequence.
        This is called \textit{sigma notation}.

        \paragraph{Example:} Evaluate the sum $\sum_{k=1}^{3}(2k-1)$. ~\\
        First, we need to find the elements at $k=1$, $k=2$, and $k=3$:

        \begin{center}
            \begin{tabular}{| c | c | c |}
                \hline
                \textbf{ $k=1$ } & \textbf{ $k=2$ } & \textbf{ $k=3$ } \\
                \hline
                $a_{1} = (2 \cdot 1 - 1) = 1$ &
                $a_{2} = (2 \cdot 2 - 1) = 3$ &
                $a_{3} = (2 \cdot 3 - 1) = 5$
                \\
                \hline
            \end{tabular}
        \end{center}
        ~\\
        Then, we can add the values: \\
        $\sum_{k=1}^{3}(2k-1) $ \tab
        $= a_{1} + a_{2} + a_{3}$ \tab
        $= 1 + 3 + 5 $ \tab
        \fbox{ $= 9$ }
    \end{intro}


    % - QUESTION --------------------------------------------------%
    \stepcounter{question}
    \begin{questionNOGRADE}{\thequestion}
        Evaluate the following summations.

        \begin{center}
            \begin{tabular}{p{6cm} p{6cm}}
            a. $$ \sum_{k=1}^{4}(3k) $$
            &
            b. $$ \sum_{k=1}^{5}(4) $$
        \end{tabular} 
        \end{center}     
    \end{questionNOGRADE}

    
}{
    \begin{enumerate}
        \item[1a.]   12
        \item[1b.]  Item $n$ is $2 \times n$.
        \item[1c.]  Item 2 = Item 1 + 2
        \item[1d.]  Item 3 = Item 2 + 2
        \item[1e.]  Item n = Item n - 1 + 2

        \item[2a.]  2, 3, 4, 5, 6
        \item[2b.]  3, 5, 7, 9, 11
        \item[2c.]  1, 3, 5, 7, 9
        \item[2d.]  2, 5, 11, 23, 47

        \item[3a.]  $a_{n} = n+2$
        \item[3b.]  $a_{n} = n+5$
        
        \item[4a.]  $a_{n} = 2n$
        \item[4b.]  $a_{n} = 3n$
        \item[4c.]  $a_{n} = 5n$
        \item[4d.]  $a_{n} = n^{2}$

        \item[5a.]  $a_{n} = 2n-1$
        \item[5b.]  $a_{n} = 3n+1$
        \item[5c.]  $a_{n} = 5n+2$
        \item[5d.]  $a_{n} = n^{2} + 1$

        \item[6a.]  $a_{1} = 1; a_{n} = a_{n-1} + 2$
        \item[6b.]  $a_{1} = 1; a_{n} = a_{n-1} + 4$
        \item[6c.]  $a_{1} = 2; a_{n} = a_{n-1} + 2$
        \item[6d.]  $a_{1} = 2; a_{n} = a_{n-1} + 4$

        \item[7a.]  $a_{1} = 2; a_{n} = 2a_{n-1}$
        \item[7b.]  $a_{1} = 1; a_{n} = 3a_{n-1}$
        \item[7c.]  $a_{1} = 3; a_{n} = 2a_{n-1}$
        \item[7d.]  $a_{1} = 2; a_{n} = (a_{n-1})^{2}$

        \item[8a.]  $a_{1} = 1; a_{n} = 2a_{n-1} + 1$
        \item[8b.]  $a_{1} = 2; a_{n} = 2a_{n-1} + 1$
        \item[8c.]  $a_{1} = 1; a_{n} = 3a_{n-1} + 2$
        \item[8d.]  $a_{1} = 1; a_{n} = 2a_{n-1} + 2$

        \item[9a.]  $a_{1} = 3$, $a_{2} = 6$, $a_{3} = 9$, $a_{4} = 12$, sum is $30$.
        \item[9b.]  $a_{1} = 4$, $a_{2} = 4$, $a_{3} = 4$, $a_{4} = 4$, $a_{5} = 4$, sum is $20$.
    \end{enumerate}
}

    



\end{document}

